\documentclass[14pt]{extarticle} 

\usepackage{amsmath,mathtools,amsfonts,amsthm,amssymb,hyperref}
\usepackage{wasysym,geometry,latexsym,parskip,bookmark}
\usepackage{mathtools,float}
%\usepackage{bussproofs}

\newtheorem{defn}{Definition}
\newtheorem{thm}{Theorem}
\newtheorem{claim}{Claim}
\newtheorem{lemma}{Lemma}
\newcommand{\dps}{\displaystyle}
\newcommand{\fbl}{\underline{\hspace{1cm}}\,\,}

\hypersetup{colorlinks, allcolors=blue, linktoc=all}
\geometry{a4paper} 
\geometry{margin=0.5in}

\title{Chapter 1 Solutions, Susanna Epp Discrete Math 5th Edition}

\author{https://github.com/spamegg1}

\begin{document}
\maketitle
\tableofcontents

\section{Exercise Set 1.1}

{\bf In each of 1–6, fill in the blanks using a variable or variables to rewrite the given statement.}

\subsection{Problem 1}
Is there a real number whose square is $-1$?

\subsubsection{(a)}
Is there a real number $x$ such that \fbl?

\begin{proof}
Is there a real number $x$ such that \underline{$x^2 = -1$}?
\end{proof}

\subsubsection{(b)}
Does there exist \fbl such that $x^2 = -1$?

\begin{proof}
Does there exist \underline{a real number $x$} such that $x^2 = -1$?
\end{proof}

\subsection{Problem 2}
Is there an integer that has a remainder of 2 when it is divided by 5 and a remainder of 3 when it is divided by 6?

{\it Note: There are integers with this property. Can you
think of one?}

\subsubsection{(a)}
Is there an integer $n$ such that $n$ has \fbl?

\begin{proof}
Is there an integer $n$ such that $n$ has \underline{a remainder of 2 when it is divided by 5} \underline{and a remainder of 3 when it is divided by 6}?
\end{proof}

\subsubsection{(b)}
Does there exist \fbl such that if $n$ is divided by 5 the remainder is 2 and if \fbl?

\begin{proof}
Does there exist \underline{an integer $n$} such that if $n$ is divided by 5 the remainder is 2 and if \underline{$n$ is divided by 6 the remainder is 3}?
\end{proof}

\subsection{Problem 3}
Given any two distinct real numbers, there is a real number in between them.

\subsubsection{(a)}
Given any two distinct real numbers $a$ and $b$, there is a real number $c$ such that $c$ is \fbl.

\begin{proof}
Given any two distinct real numbers $a$ and $b$, there is a real number $c$ such that $c$ is \underline{between $a$ and $b$}. 
\end{proof}

\subsubsection{(b)}
For any two \fbl, \fbl such that $c$ is between $a$ and $b$.

\begin{proof}
For any two \underline{distinct real numbers $a$ and $b$}, \underline{there exists a real number $c$} such that $c$ is between $a$ and $b$.
\end{proof}

\subsection{Problem 4}
Given any real number, there is a real number that is greater.

\subsubsection{(a)}
Given any real number $r$, there is \fbl $s$ such that $s$ is \fbl.

\begin{proof}
Given any real number $r$, there is \underline{a real number} $s$ such that $s$ is \underline{greater than $r$}.
\end{proof}

\subsubsection{(b)}
For any \fbl, \fbl such that $s > r$.

\begin{proof}
For any \underline{real number $r$}, \underline{there exists a real number $s$} such that $s > r$.
\end{proof}

\subsection{Problem 5}
The reciprocal of any positive real number is positive.

\subsubsection{(a)}
Given any positive real number $r$, the reciprocal of \fbl.

\begin{proof}
Given any positive real number $r$, the reciprocal of \underline{$r$ is positive}.
\end{proof}

\subsubsection{(b)}
For any real number $r$, if $r$ is \fbl, then \fbl. 

\begin{proof}
For any real number $r$, if $r$ is \underline{positive}, then \underline{$1/r$ is positive}. 
\end{proof}

\subsubsection{(c)}
If a real number $r$ \fbl, then \fbl.

\begin{proof}
If a real number $r$ \underline{is positive}, then \underline{$1/r$ is positive}.
\end{proof}

\subsection{Problem 6}
The cube root of any negative real number is negative.

\subsubsection{(a)}
Given any negative real number s, the cube root of \fbl.

\begin{proof}
Given any negative real number $s$, the cube root of \underline{$s$ is negative}.
\end{proof}

\subsubsection{(b)}
For any real number $s$, if $s$ is \fbl, then \fbl.

\begin{proof}
For any real number $s$, if $s$ is \underline{negative}, then \underline{$\sqrt[3]{s}$ is negative}.
\end{proof}

\subsubsection{(c)}
If a real number s \fbl, then \fbl.

\begin{proof}
If a real number s \underline{is negative}, then \underline{$\sqrt[3]{s}$ is negative}.
\end{proof}

\subsection{Problem 7}
Rewrite the following statements less formally, without using variables. Determine, as best as you can, whether the statements are true or false.

\subsubsection{(a)}
There are real numbers $u$ and $v$ with the property that $u + v < u - v$.

\begin{proof}
Rewrite: There are real numbers such that their sum is less than their difference.

True: $0$ and $-1$ have this property: $-1 = 0 + (-1) < 0 - (-1) = 1$
\end{proof}

\subsubsection{(b)}
There is a real number $x$ such that $x^2 < x$.

\begin{proof}
Rewrite: there is a real number whose square is less than itself.

True: $1/2$ has this property: $\dps \frac{1}{4} = \left(\frac{1}{2}\right)^2 < \frac{1}{2}$
\end{proof}

\subsubsection{(c)}
For every positive integer $n$, $n^2 \geq n$.

\begin{proof}
Rewrite: The square of every positive integer is greater than or equal to itself.

True: if we look at the first few examples it holds: $1^2 = 1 \geq 1$, $2^2 = 4 \geq 2$, $3^2 = 9 \geq 3$ and so on. This is however not a proof. Later we'll learn methods to prove this for all positive integers.
\end{proof}

\subsubsection{(d)}
For all real numbers $a$ and $b$, $|a + b| \leq |a| + |b|$.

\begin{proof}
Rewrite: for all two real numbers, the absolute value of their sum is less than or equal to the sum of their absolute values.

True: this is known as the Triangle Inequality and it will be proved later.
\end{proof}

\subsection{Problem}

\subsubsection{(a)}

\begin{proof}
\end{proof}

\subsubsection{(b)}

\begin{proof}
\end{proof}

\subsection{Problem}

\subsubsection{(a)}

\begin{proof}
\end{proof}

\subsubsection{(b)}

\begin{proof}
\end{proof}

\end{document}
