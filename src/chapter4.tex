\documentclass[14pt]{extarticle}

\usepackage[table]{xcolor}
\usepackage{amsmath,mathtools,amsfonts,amsthm,amssymb,hyperref,wasysym,pifont}
\usepackage{parskip,geometry,latexsym,bookmark,mathtools,float,cancel,tcolorbox}

\newtheorem{defn}{Definition}
\newtheorem{thm}{Theorem}
\newtheorem{claim}{Claim}
\newtheorem{lemma}{Lemma}

\newcommand{\dps}{\displaystyle}
\newcommand{\fbl}{\underline{\hspace{1cm}}\,\,}
\newcommand{\R}{\mathbb{R}}
\newcommand{\Z}{\mathbb{Z}}
\newcommand{\from}{\leftarrow}
\newcommand{\true}{{\bf t}}
\newcommand{\false}{{\bf c}}
\newcommand{\bic}{\leftrightarrow}
\newcommand{\base}[1]{{\color{cyan}#1}}
\newcommand{\da}{\downarrow}
\newcommand{\fa}{\forall}
\newcommand{\te}{\exists}

\hypersetup{colorlinks,allcolors=blue,linktoc=all}
\geometry{a4paper}
\geometry{margin=0.42in}

\title{Chapter 4 Solutions, Susanna Epp Discrete Math 5th Edition}

\author{https://github.com/spamegg1}

\begin{document}
\maketitle
\tableofcontents

\section{Exercise Set 4.1}

\subsection{Exercise 1}
Assume that $k$ is a particular integer.

\subsubsection{(a)}
Is $-17$ an odd integer?

\begin{proof}
Yes: $-17 = 2(-9) + 1$.

\end{proof}

\subsubsection{(b)}
Is 0 neither even nor odd?

\begin{proof}
No. 0 is even because $0 = 0 \cdot 2$.
\end{proof}

\subsubsection{(c)}
Is $2k - 1$ odd?

\begin{proof}
Yes: $2k - 1 = 2(k - 1) + 1$ and $k - 1$ is an integer because it is a difference of integers.
\end{proof}

\subsection{Exercise 2}
Assume that $c$ is a particular integer.

\subsubsection{(a)}
Is $-6c$ an even integer?

\begin{proof}
Yes, because $-6c = 2 \cdot (-3c) = 2k$ where $k = -3c$ is an integer.
\end{proof}

\subsubsection{(b)}
Is $8c + 5$ an odd integer?

\begin{proof}
Yes, because $8c + 5 = 2(4c + 2) + 1$ and $k = 4k+2$ is an integer.
\end{proof}

\subsubsection{(c)}
Is $(c^2 + 1) - (c^2 - 1) - 2$ an even integer?

\begin{proof}
Yes, because it equals 0: $(c^2 + 1) - (c^2 - 1) - 2 = c^2 + 1 - c^2 + 1 - 2 = 2 - 2 = 0$.
\end{proof}

\subsection{Exercise 3}
Assume that $m$ and $n$ are particular integers.

\subsubsection{(a)}
Is $6m + 8n$ even?

\begin{proof}
Yes: $6m + 8n = 2(3m + 4n)$ and $(3m + 4n)$ is an integer because $3, 4, m$, and $n$ are integers, and products and sums of integers are integers.
\end{proof}

\subsubsection{(b)}
Is $10mn + 7$ odd?

\begin{proof}
Yes: $10mn + 7 = 2(5mn + 3) + 1$ and $5mn + 3$ is an integer because $3, 5, m$, and $n$ are integers, and products and sums of integers are integers.
\end{proof}

\subsubsection{(c)}
If $m > n > 0$, is $m^2 - n^2$ composite?

\begin{proof}
Not necessarily. For instance, if $m = 3$ and $n = 2$, then $m^2 - n^2 = 9 - 4 = 5$, which is prime. (However, $m^2 - n^2$ is composite for many values of $m$ and $n$ because of the identity $m^2 - n^2 = (m - n)(m + n)$.)
\end{proof}

\subsection{Exercise 4}
Assume that $r$ and $s$ are particular integers.

\subsubsection{(a)}
Is $4rs$ even?

\begin{proof}
Yes: $4rs = 2(2rs)$ and $2rs$ is an integer because $2, r, s$ are integers, and products of integers are integers.
\end{proof}

\subsubsection{(b)}
Is $6r + 4s^2 + 3$ odd?

\begin{proof}
Yes: $6r + 4s^2 + 3 = 2(3r + 2s^2 + 1) + 1$ and $3r + 2s^2 + 1$ is an integer because $3, r, 2, s, 1$ are integers and products and sums of integers are integers.
\end{proof}

\subsubsection{(c)}
If $r$ and $s$ are both positive, is $r^2 + 2rs + s^2$ composite?

\begin{proof}
Yes: $r^2 + 2rs + s^2 = (r+s)(r+s)$ and $r + s \geq 2$, therefore $r^2 + 2rs + s^2$ is a product of two integers both of which are greater than 1.
\end{proof}

{\bf \color{cyan} Prove the statements in 5–11.}

\subsection{Exercise 5}
There are integers $m$ and $n$ such that $m > 1$ and $n > 1$ and $\frac{1}{m} + \frac{1}{n}$ is an integer.

\begin{proof}
For example, let $m = n = 2$. Then $m$ and $n$ are integers such that $m > 1$ and $n > 1$ and $\frac{1}{m} + \frac{1}{n} = \frac{1}{2} + \frac{1}{2} = 1$ which is an integer.
\end{proof}

\subsection{Exercise 6}
There are distinct integers $m$ and $n$ such that $\frac{1}{m} + \frac{1}{n}$ is an integer.

\begin{proof}
For example, let $m = 1, n = -1$. Then $m$ and $n$ are integers such that $\frac{1}{m} + \frac{1}{n} = \frac{1}{1} - \frac{1}{1} = 0$ which is an integer.
\end{proof}

\subsection{Exercise 7}
There are real numbers $a$ and $b$ such that $\sqrt{a+b} = \sqrt{a} \sqrt{b}$.

\begin{proof}
For example, let $a = 0, b = 0$. Then $a$ and $b$ are real numbers such that 

$\sqrt{a+b} = \sqrt{0+0} = 0 = \sqrt{0} + \sqrt{0} = \sqrt{a} + \sqrt{b}$.
\end{proof}

\subsection{Exercise 8}
There is an integer $n > 5$ such that $2^n - 1$ is prime.

\begin{proof}
For example, let $n = 7$. Then $n$ is an integer such that
$n > 5$ and $2^n - 1 = 127$, which is prime.
\end{proof}

\subsection{Exercise 9}
There is a real number $x$ such that $x > 1$ and $2^x > x^{10}$.

\begin{proof}
For example, take $x = 80$. Then 

$x^{10} = 80^{10} = 8^{10} \cdot 10^{10} = (2^3)^{10} \cdot 10^{10} = 2^{30} \cdot 10^{10}$.

We have $2^{50} \approx 1,125899907 \cdot 10^{15} > 10^{10}$. So $2^{80} = 2^{30} \cdot 2^{50} > 2^{30} \cdot 10^{10} = 80^{10}$.

Therefore $x = 80$ is a real number such that $x > 1$ and $2^x > x^{10}$.
\end{proof}

\begin{tcolorbox}[colframe=cyan]
{\bf \color{cyan} Definition:} An integer $n$ is called a {\bf perfect square} if, and only if, $n = k^2$ for some integer $k$.
\end{tcolorbox}

\subsection{Exercise 10}
There is a perfect square that can be written as the sum of two other perfect squares.

\begin{proof}
For example, 25, 9, and 16 are all perfect squares, because $25 = 5^2, 9 = 3^2$, and $16 = 4^2$, and $25 = 9 + 16$. Thus 25 is a perfect square that can be written as a sum of two other perfect squares.
\end{proof}

\subsection{Exercise 11}
There is an integer $n$ such that $2n^2 - 5n + 2$ is prime.

\begin{proof}
For example, take $n = 3$. Then $2n^2 - 5n + 2 = 18 - 15 + 2 = 5$ is prime. (You can find this value of $n$ by either starting at $n = 1$ and using trial and error, or noticing that $2n^2 - 5n + 2 = (2n - 1)(n - 2)$, so, for this to be prime, one of the factors has to be 1.)
\end{proof}

{\bf \color{cyan} In $12-13$, (a) write a negation for the given statement, and (b) use a counterexample to disprove the given statement. explain how the counterexample actually shows that the given statement is false.}

\subsection{Exercise 12}
For all real numbers $a$ and $b$, if $a < b$ then $a^2 < b^2$.

\begin{proof}
a. {\it Negation for the statement:} There exist real numbers $a$ and $b$ such that $a < b$ and $a^2 \nless b^2$.

b. {\it Counterexample for the statement:} Let $a = -2$ and
$b = -1$. Then $a < b$ because $-2 < -1$, but $a^2 \nless b^2$ because $(-2)^2 = 4$ and $(-1)^2 = 1$ and $4 \nless 1$. {\it [So the hypothesis of the statement is true and its conclusion is false.]}
\end{proof}

\subsection{Exercise 13}
For every integer $n$, if $n$ is odd then $\frac{n-1}{2}$ is odd.

\begin{proof}
a. {\it Negation for the statement:} There exists an integer $n$ such that $n$ is odd and $\frac{n-1}{2}$ is not odd.

b. {\it Counterexample for the statement:} Let $n = 5$. Then $n$ is odd because $5 = 2 \cdot 2 + 1$, but $\frac{n-1}{2}$ is not odd because $\frac{5-1}{2} = 2 = 2 \cdot 1$ is even. {\it [So the hypothesis of the statement is true and its conclusion is false.]}
\end{proof}

{\bf \color{cyan} Disprove each of the statements in $14-16$ by giving a counterexample. In each case explain how the counterexample actually disproves the statement.}

\subsection{Exercise 14}
For all integers $m$ and $n$, if $2m + n$ is odd then $m$ and $n$ are both odd.

\begin{proof}
\underline{Counterexample:} Let $m = 2$ and $n = 1$. Then $2m + n = 2 \cdot 2 + 1 = 5$, which is odd. But $m$ is not odd, and so it is false that both $m$ and $n$ are odd. {\it [This is one counterexample among many.]}
\end{proof}

\subsection{Exercise 15}
For every integer $p$, if $p$ is prime then $p^2 - 1$ is even.

\begin{proof}
\underline{Counterexample:} Let $p = 2$ which is prime, but $p^2 - 1 = 2^2 - 1 = 4 - 1 = 3$ is not even. {\it [This is the only counterexample! For every other prime $p$, $p^2 - 1$ is even.]}
\end{proof}

\subsection{Exercise 16}
For every integer $n$, if $n$ is even then $n^2 + 1$ is prime.

\begin{proof}
\underline{Counterexample:} Let $n = 8$. Then $n$ is even. But $n^2 + 1 = 65 = 13 \cdot 5$ is not prime. {\it [This is one counterexample among many.]}
\end{proof}

{\bf \color{cyan} In $17-20$, determine whether the property is true for all integers, true for no integers, or true for some integers and false for other integers. Justify your answers.}

\subsection{Exercise 17}
$(a+b)^2 = a^2 + b^2$

\begin{proof}
This property is true for some integers and false for other integers. For instance, if $a = 0$ and $b = 1$, the property is true because $(0 + 1)^2 = 0^2 + 1^2$, but if $a = 1$ and $b = 1$, the property is false because $(1 + 1)^2 = 4$ and $1^2 + 1^2 = 2$ and $4 \neq 2$.
\end{proof}

\subsection{Exercise 18}
$\dps \frac{a}{b} + \frac{c}{d} = \frac{a+c}{b+d}$

\begin{proof}
True for some integers, false for others. For example, if $a = c = 0$ and $b = d = 1$ then $\dps \frac{0}{1} + \frac{0}{1} = 0 = \frac{0+0}{1+1}$ is true. But if $a = 1, b = 2, c = 3$ and $d = 4$ then 

$\dps \frac{a}{b} + \frac{c}{d} = \frac{1}{2} + \frac{3}{4} = \frac{5}{4} \neq \frac{4}{6} = \frac{1+3}{2+4} = \frac{a+c}{b+d}$.
\end{proof}

\subsection{Exercise 19}
$-a^n = (-a)^n$

{\it Hint:} This property is true for some integers and false for other integers. To justify this answer you need to find examples of both.

\begin{proof}
True for some integers: let $a = 0, n = 1$. Then $-0^1 = -0 = 0$ and $(-0)^1 = 0^1 = 0$ so the equality holds. When $a = n = 2$ it is false: $-2^2 = -4$ but $(-2)^2 = 4$ and $-4 \neq 4$.
\end{proof}

\subsection{Exercise 20}
The average of any two odd integers is odd.

\begin{proof}
True for some, false for others. For example, 3 and 7 are both odd, and their average $(3+7)/2 = 5$ is odd. But 3 and 5 are both odd, and their average is $(3 + 5) / 2 = 4$ is even.
\end{proof}

{\bf \color{cyan} Prove the statement in 21 and 22 by the method of exhaustion.}

\subsection{Exercise 21}
Every positive even integer less than 26 can be expressed as a sum of three of fewer perfect squares. (For instance, $10 = 1^2 + 3^2$ and $16 = 4^2$.)

\begin{proof}
$2 = 1^2 + 1^2$

$4 = 2^2$

$6 = 2^2 + 1^2 + 1^2$

$8 = 2^2 + 2^2$

$10 = 1^2 + 3^2$

$12 = 2^2 + 2^2 + 2^2$

$16 = 4^2$

$18 = 4^2 + 1^2 + 1^2$

$20 = 4^2 + 2^2$

$22 = 2^2 + 2^2 + 3^2$

$24 = 4^2 + 2^2 + 2^2$
\end{proof}

\subsection{Exercise 22}
For each integer $n$ with $1 \leq n \leq 10$, $n^2 - n + 11$ is a prime number.

\begin{proof}
$1^2 - 1 + 11 = 11$ is prime.

$2^2 - 2 + 11 = 13$ is prime.

$3^2 - 3 + 11 = 17$ is prime.

$4^2 - 4 + 11 = 23$ is prime.

$5^2 - 5 + 11 = 31$ is prime.

$6^2 - 6 + 11 = 41$ is prime.

$7^2 - 7 + 11 = 53$ is prime.

$8^2 - 8 + 11 = 67$ is prime.

$9^2 - 9 + 11 = 83$ is prime.

$10^2 - 10 + 11 = 101$ is prime.
\end{proof}

{\bf \color{cyan} Each of the statements in $23-26$ is true. For each, (a) rewrite the statement with the quantification implicit as If \fbl, then \fbl, and (b) write the first sentence of a proof (the “starting point”) and the last sentence of a proof (the “conclusion to be shown”). (Note that you do not need to understand the statements in order to be able to do these exercises.)}

\subsection{Exercise 23}
For every integer $m$, if $m > 1$ then $0 < \frac{1}{m} < 1$.

\begin{proof}
a. If an integer is greater than 1, then its reciprocal is
between 0 and 1.

b. {\it Start of proof:} Suppose $m$ is any integer such that $m > 1$. {\it Conclusion to be shown:} $0 < 1/m < 1$.
\end{proof}

\subsection{Exercise 24}
For every real number $x$, if $x > 1$ then $x^2 > x$.

\begin{proof}
a. If a real number is greater than 1, then its square is greater than itself.

b. {\it Start of proof:} Suppose $x$ is any real number such that $x > 1$. {\it Conclusion to be shown:} $x^2 > x$.
\end{proof}

\subsection{Exercise 25}
For all integers $m$ and $n$, if $mn = 1$ then $m = n = 1$ or $m = n = -1$.

\begin{proof}
a. If the product of two integers is 1, then either both are 1 or both are $-1$.

b. {\it Start of proof:} Suppose $m$ and $n$ are any integers with $mn = 1$. {\it Conclusion to be shown:} $m = n = 1$ or $m = n = -1$.
\end{proof}

\subsection{Exercise 26}
For every real number $x$, if $0 < x < 1$ then $x^2 < x$.

\begin{proof}
a. If a real number is strictly between 0 and 1, then its square is less than itself.

b. {\it Start of proof:} Suppose $x$ is any real number such that $0 < x < 1$. {\it Conclusion to be shown:} $x^2 < x$.
\end{proof}

\subsection{Exercise 27}
Fill in the blanks in the following proof.

{\bf Theorem:} For every odd integer $n$, $n^2$ is odd.

{\bf Proof:} Suppose $n$ is any {\color{cyan}(a)} \fbl. By definition of odd, $n = 2k + 1$ for some integer $k$. Then

\begin{center}
\begin{tabular}{rcll}
$n^2$ & = & {\color{cyan}(b)} $(\fbl)^2$ & \color{cyan} by substitution \\
& = & $4k^2 + 4k + 1$ & \color{cyan} by multiplying \\
& = & $2(2k^2 + 2k) + 1$ & \color{cyan} by factoring out a 2 \\
\end{tabular}
\end{center}

Now $2k^2 + 2k$ is an integer because it is a sum of products of integers. Therefore, $n^2$ equals $2\cdot$ (an integer) $+ 1$, and so {\color{cyan}(c)} \fbl is odd by definition of odd.

Because we have not assumed anything about $n$ except that it is an odd integer, it follows from the principle of {\color{cyan}(d)} \fbl that for every odd integer $n$, $n^2$ is odd.

\begin{proof}
(a) particular but arbitrarily chosen odd integer 
(b) $2k + 1$ (c) $n^2$ (d) universal generalization
\end{proof}

{\bf \color{cyan} In each of $28-31$: \\ 
a. Rewrite the theorem in three different ways: as $\fa$ \fbl, if \fbl, then \fbl; as $\fa$ \fbl, \fbl (without using the words if or then), and as If \fbl, then \fbl (without using an explicit universal quantifier). \\
b. Fill in the blanks in the proof of the theorem.}

\subsection{Exercise 28}
{\bf Theorem:} The sum of any two odd integers is even.

{\bf Proof:} Suppose $m$ and $n$ are any {\it [particular but arbitrarily chosen]} odd integers. 

{\it [We must show that $m + n$ is even.]}

By {\color{cyan}(a)} \fbl, $m = 2r + 1$ and $n = 2s + 1$ for some integers $r$ and $s$. Then

\begin{center}
\begin{tabular}{rcll}
$m+n$ & = & $(2r+1) + (2s+1)$ & {\color{cyan} by (b)} \fbl \\
& = & $2r + 2s + 2$ & \\
& = & $2(r+s+1)$ & \color{cyan} by algebra \\
\end{tabular}
\end{center}

Let $u = r + s + 1$. Then $u$ is an integer because $r, s$ and 1 are integers and because {\color{cyan}(c)} \fbl. 

Hence $m + n = 2u$, where $u$ is an integer, and so, by {\color{cyan}(d)} \fbl, $m + n$ is even {\it [as was to be shown]}.

\begin{proof}
a. $\fa$ integers $m$ and $n$, if $m$ and $n$ are odd then $m + n$ is even.

$\fa$ odd integers $m$ and $n$, $m + n$ is even.

If $m$ and $n$ are any odd integers, then $m + n$ is even.

b. (a) definition of odd, (b) substitution, (c) any sum of
integers is an integer, (d) definition of even
\end{proof}

\subsection{Exercise 29}
{\bf Theorem:} The negative of any even integer is even.

{\bf Proof:} Suppose $n$ is any {\it [particular but arbitrarily chosen]} even integer. 

{\it [We must show that $-n$ is even.]}

By {\color{cyan}(a)} \fbl, $n = 2k$ for some integer $k$.

Then

\begin{center}
\begin{tabular}{rcll}
$-n$ & = & $-(2k)$ & \color{cyan} by (b) \fbl \\
& = & $2(-k)$ & \color{cyan} by algebra \\
\end{tabular}
\end{center}

Let $r = -k$. Then $r$ is an integer because $-1$ and $k$ are integers and {\color{cyan}(c)} \fbl. 

Hence $-n = 2r$, where $r$ is an integer, and so $-n$ is even by {\color{cyan}(d)} \fbl {\it [as was to be shown]}.

\begin{proof}
a. $\fa$ integer $n$, if $n$ is even then $-n$ is even.

$\fa$ even integer $n$, $-n$ is even.

If $n$ is any even integer, then $-n$ is even.

b. (a) definition of even, (b) substitution, (c) any product of integers is an integer, (d) definition of even
\end{proof}

\subsection{Exercise 30}
{\bf Theorem:} The sum of any even integer and any odd integer is odd.

{\bf Proof:} Suppose $m$ is any even integer and $n$ is any {\color{cyan}(a)} \fbl. By definition of even, $m = 2r$ for some {\color{cyan}(b)} \fbl, and by definition of odd, $n = 2s + 1$ for some integer $s$. By substitution and algebra,

\begin{center}
\begin{tabular}{ccccc}
$m+n$ & = & {\color{cyan}(c)} \fbl & = & $2(r+s)+1$ \\
\end{tabular}
\end{center}

Since $r$ and $s$ are integers, so is their sum $r+s$. Hence $m+n$ has the form twice some integer plus one, and so, by {\color{cyan}(d)} \fbl by definition of odd.

\begin{proof}
a. $\fa$ integers $m$ and $n$, if $m$ is even and $n$ is odd, then $m + n$ is odd.

$\fa$ even integers $m$ and odd integers $n$, $m + n$ is odd.

If $m$ is any even integer and $n$ is any odd integer,
then $m + n$ is odd.

b. (a) any odd integer (b) integer $r$ (c) $2r + (2s + 1)$ (d) $m + n$ is odd
\end{proof}

\subsection{Exercise 31}
{\bf Theorem:} Whenever $n$ is an odd integer, $5n^2 + 7$ is even.

{\bf Proof:} Suppose $n$ is any {\it [particular but arbitrarily chosen]} odd integer. 

{\it [We must show that $5n^2 + 7$ is even.]}

By definition of odd, $n = $ {\color{cyan}(a)} \fbl, for some integer $k$. 

Then

\begin{center}
\begin{tabular}{rcll}
$5n^2 + 7$ & = & {\color{cyan} (b)} \fbl & \color{cyan} substitution \\
& = & $5(4k^2 + 4k + 1) + 7$ & \\
& = & $20k^2 + 20k + 12$ & \\
& = & $2(10k^2 + 10k + 6)$ & \color{cyan} by algebra \\
\end{tabular}
\end{center}

Let $t = $ {\color{cyan} (c)} \fbl. Then $t$ is an integer because products and sums of integers are integers. 

Hence $5n^2 + 7 = 2t$, where $t$ is an integer, and thus {\color{cyan}(d)} \fbl by definition of even {\it [as was to be shown]}.

\begin{proof}
a. $\fa$ integer $n$, if $n$ is odd, then $5n^2+7$ is even.

$\fa$ odd integer $n$, $5n^2+7$ is even.

If $n$ is any odd integer, then $5n^2+7$ is even.

b. (a) $2k+1$ (b) $5(2k+1)^2 + 7$ (c) $10k^2 + 10k + 6$ (d) $5n^2+7$ is even
\end{proof}

\section{Exercise Set 4.2}

{\bf \color{cyan} Prove the statements in $1-11$. In each case use only the definitions of the terms and the assumptions listed on page 161, not any previously established properties of odd and even integers. Follow the directions given in this section for writing proofs of universal statements.}

\subsection{Exercise 1}
For every integer $n$, if $n$ is odd then $3n + 5$ is even.

\begin{proof}
Suppose $n$ is any {\it [particular but arbitrarily chosen]} odd integer. 

{\it [We must show that $3n + 5$ is even. By definition of even, this means we must show that $3n + 5 = 2\cdot$(some integer).]}

By definition of odd, $n = 2r + 1$, for some integer $r$. 

Then

\begin{center}
\begin{tabular}{rcll}
$3n + 5$ & = & $3(2r + 1) + 5$ & \color{cyan} by substitution \\
& = & $6r + 3 + 5$ & \\
& = & $6r + 8$ & \\
& = & $2(3r + 4)$ & \color{cyan} by algebra \\
\end{tabular}
\end{center}

{\it [Idea for the rest of the proof: We want to show that $3n + 5 = 2\cdot$(some integer). At this point we know that $3n + 5 = 2(3r + 4)$. So is $3r + 4$ an integer? Yes, because products and sums of integers are integers.]}

Let $k = 3r + 4$. 

Then $k$ is an integer because products and sums of integers are integers. 

Hence $3n + 5 = 2(3r+4) = 2k$ where $k$ is an integer. Hence by definition of even $3n+5$ is even {\it [as was to be shown]}.
\end{proof}

\subsection{Exercise 2}
For every integer $m$, if $m$ is even then $3m + 5$ is odd.

\begin{proof}
Suppose $m$ is any {\it [particular but arbitrarily chosen]} even integer. 

{\it [We must show that $3m + 5$ is odd. By definition of odd, this means we must show that $3m + 5 = 2\cdot\text{(some integer)} + 1$.]}

By definition of even, $m = 2r$, for some integer $r$. 

Then

\begin{center}
\begin{tabular}{rcll}
$3m + 5$ & = & $3(2r) + 5$ & \color{cyan} by substitution \\
& = & $6r + 5$ & \\
& = & $6r + 4 + 1$ & \\
& = & $2(3r + 2) + 1$ & \color{cyan} by algebra \\
\end{tabular}
\end{center}

{\it [Idea for the rest of the proof: We want to show that $3m + 5 = 2\cdot\text{(some integer)} + 1$. At this point we know that $3m + 5 = 2(3r + 2) + 1$. So is $3r + 2$ an integer? Yes, because products and sums of integers are integers.]}

Let $k = 3r + 2$. 

Then $k$ is an integer because products and sums of integers are integers. 

Hence $3m + 5 = 2(3r+2) + 1 = 2k + 1$ where $k$ is an integer. Hence by definition of odd $3n+5$ is odd {\it [as was to be shown]}.
\end{proof}

\subsection{Exercise 3}
For every integer $n$, $2n - 1$ is odd.

\begin{proof}
Suppose $n$ is any {\it [particular but arbitrarily chosen]} integer. 

{\it [We must show that $2n - 1$ is odd. By definition of odd, this means we must show that $2n - 1 = 2 \cdot \text{(some integer)} + 1$.]}

Then

\begin{center}
\begin{tabular}{rcll}
$2n-1$ & = & $2n - 2 + 2 - 1$ & \color{cyan} because $-2 + 2 = 0$ \\
& = & $2(n-1) + 2 - 1$ & \\
& = & $2(n-1) + 1$ & \color{cyan} by algebra \\
\end{tabular}
\end{center}

Let $k = n-1$. 

Then $k$ is an integer because the difference of two integers ($n$ and $1$) is an integer. 

Hence $2n-1 = 2(n-1) + 1 = 2k + 1$ where $k$ is an integer, and thus by definition of odd $2n-1$ is odd {\it [as was to be shown]}.
\end{proof}

\subsection{Exercise 4}
The difference of any even integer minus any odd integer is odd.

\begin{proof}
{\bf Proof:} Suppose $n$ is any {\it [particular but arbitrarily chosen]} ???. 

{\it [We must show that ???.]}

By definition of ???, $n = ???$, for some ???. 

Then

\begin{center}
\begin{tabular}{rcll}
$???$ & = & $???$ & \color{cyan} by substitution \\
& = & $???$ & \\
& = & $???$ & \\
& = & $???$ & \color{cyan} by algebra \\
\end{tabular}
\end{center}

Let $k = ???$. 

Then $k$ is an integer because ??? are integers. 

Hence $??? = ???$ where $k$ is an integer, and thus by definition of ??? it is ??? {\it [as was to be shown]}.
\end{proof}

\subsection{Exercise 5}
If $a$ and $b$ are any odd integers, then $a^2 + b^2$ is even.

\begin{proof}
{\bf Proof:} Suppose $n$ is any {\it [particular but arbitrarily chosen]} ???. 

{\it [We must show that ???.]}

By definition of ???, $n = ???$, for some ???. 

Then

\begin{center}
\begin{tabular}{rcll}
$???$ & = & $???$ & \color{cyan} by substitution \\
& = & $???$ & \\
& = & $???$ & \\
& = & $???$ & \color{cyan} by algebra \\
\end{tabular}
\end{center}

Let $k = ???$. 

Then $k$ is an integer because ??? are integers. 

Hence $??? = ???$ where $k$ is an integer, and thus by definition of ??? it is ??? {\it [as was to be shown]}.
\end{proof}

\subsection{Exercise 6}
If $k$ is any odd integer and $m$ is any even integer, then $k^2 + m^2$ is odd.

\begin{proof}
{\bf Proof:} Suppose $n$ is any {\it [particular but arbitrarily chosen]} ???. 

{\it [We must show that ???.]}

By definition of ???, $n = ???$, for some ???. 

Then

\begin{center}
\begin{tabular}{rcll}
$???$ & = & $???$ & \color{cyan} by substitution \\
& = & $???$ & \\
& = & $???$ & \\
& = & $???$ & \color{cyan} by algebra \\
\end{tabular}
\end{center}

Let $k = ???$. 

Then $k$ is an integer because ??? are integers. 

Hence $??? = ???$ where $k$ is an integer, and thus by definition of ??? it is ??? {\it [as was to be shown]}.
\end{proof}

\subsection{Exercise 7}
The difference between the squares of any two consecutive integers is odd.

\begin{proof}
{\bf Proof:} Suppose $n$ is any {\it [particular but arbitrarily chosen]} ???. 

{\it [We must show that ???.]}

By definition of ???, $n = ???$, for some ???. 

Then

\begin{center}
\begin{tabular}{rcll}
$???$ & = & $???$ & \color{cyan} by substitution \\
& = & $???$ & \\
& = & $???$ & \\
& = & $???$ & \color{cyan} by algebra \\
\end{tabular}
\end{center}

Let $k = ???$. 

Then $k$ is an integer because ??? are integers. 

Hence $??? = ???$ where $k$ is an integer, and thus by definition of ??? it is ??? {\it [as was to be shown]}.
\end{proof}

\subsection{Exercise 8}
For any integers $m$ and $n$, if $m$ is even and $n$ is odd
then $5m + 3n$ is odd.

\begin{proof}
{\bf Proof:} Suppose $n$ is any {\it [particular but arbitrarily chosen]} ???. 

{\it [We must show that ???.]}

By definition of ???, $n = ???$, for some ???. 

Then

\begin{center}
\begin{tabular}{rcll}
$???$ & = & $???$ & \color{cyan} by substitution \\
& = & $???$ & \\
& = & $???$ & \\
& = & $???$ & \color{cyan} by algebra \\
\end{tabular}
\end{center}

Let $k = ???$. 

Then $k$ is an integer because ??? are integers. 

Hence $??? = ???$ where $k$ is an integer, and thus by definition of ??? it is ??? {\it [as was to be shown]}.
\end{proof}

\subsection{Exercise 9}
If an integer greater than 4 is a perfect square, then the immediately preceding integer is not prime.

\begin{proof}
{\bf Proof:} Suppose $n$ is any {\it [particular but arbitrarily chosen]} ???. 

{\it [We must show that ???.]}

By definition of ???, $n = ???$, for some ???. 

Then

\begin{center}
\begin{tabular}{rcll}
$???$ & = & $???$ & \color{cyan} by substitution \\
& = & $???$ & \\
& = & $???$ & \\
& = & $???$ & \color{cyan} by algebra \\
\end{tabular}
\end{center}

Let $k = ???$. 

Then $k$ is an integer because ??? are integers. 

Hence $??? = ???$ where $k$ is an integer, and thus by definition of ??? it is ??? {\it [as was to be shown]}.
\end{proof}

\subsection{Exercise 10}
If $n$ is any even integer, then $(-1)^n = 1$.

\begin{proof}
{\bf Proof:} Suppose $n$ is any {\it [particular but arbitrarily chosen]} ???. 

{\it [We must show that ???.]}

By definition of ???, $n = ???$, for some ???. 

Then

\begin{center}
\begin{tabular}{rcll}
$???$ & = & $???$ & \color{cyan} by substitution \\
& = & $???$ & \\
& = & $???$ & \\
& = & $???$ & \color{cyan} by algebra \\
\end{tabular}
\end{center}

Let $k = ???$. 

Then $k$ is an integer because ??? are integers. 

Hence $??? = ???$ where $k$ is an integer, and thus by definition of ??? it is ??? {\it [as was to be shown]}.
\end{proof}

\subsection{Exercise 11}
If $n$ is any odd integer, then $(-1)^n = -1$.

\begin{proof}
{\bf Proof:} Suppose $n$ is any {\it [particular but arbitrarily chosen]} ???. 

{\it [We must show that ???.]}

By definition of ???, $n = ???$, for some ???. 

Then

\begin{center}
\begin{tabular}{rcll}
$???$ & = & $???$ & \color{cyan} by substitution \\
& = & $???$ & \\
& = & $???$ & \\
& = & $???$ & \color{cyan} by algebra \\
\end{tabular}
\end{center}

Let $k = ???$. 

Then $k$ is an integer because ??? are integers. 

Hence $??? = ???$ where $k$ is an integer, and thus by definition of ??? it is ??? {\it [as was to be shown]}.
\end{proof}

{\bf \color{cyan} Prove that the statements in $12-14$ are false.}

\subsection{Exercise 12}
There exists an integer $m \geq 3$ such that $m^2 - 1$ is prime.

\begin{proof}

\end{proof}

\subsection{Exercise 13}
There exists an integer $n$ such that $6n + 27$ is prime.

\begin{proof}

\end{proof}

\subsection{Exercise 14}
There exists an integer $k \geq 4$ such that $2k^2 - 5k + 2$ is prime.

\begin{proof}

\end{proof}

{\bf \color{cyan} Find the mistakes in the “proofs” shown in $15-19$.}

\subsection{Exercise 15}

\begin{proof}

\end{proof}

\subsection{Exercise 16}

\begin{proof}

\end{proof}

\subsection{Exercise 17}

\begin{proof}

\end{proof}

\subsection{Exercise 18}

\begin{proof}

\end{proof}

\subsection{Exercise 19}

\begin{proof}

\end{proof}

\subsection{Exercise 20}

\begin{proof}

\end{proof}

\subsection{Exercise 21}

\begin{proof}

\end{proof}

\subsection{Exercise 22}

\begin{proof}

\end{proof}

\subsection{Exercise 23}

\begin{proof}

\end{proof}

\subsection{Exercise 24}

\begin{proof}

\end{proof}

\subsection{Exercise 25}

\begin{proof}

\end{proof}

\subsection{Exercise 26}

\begin{proof}

\end{proof}

\subsection{Exercise 27}

\begin{proof}

\end{proof}

\subsection{Exercise 28}

\begin{proof}

\end{proof}

\subsection{Exercise 29}

\begin{proof}

\end{proof}

\subsection{Exercise 30}

\begin{proof}

\end{proof}

\subsection{Exercise 31}

\begin{proof}

\end{proof}

\subsection{Exercise 32}

\begin{proof}

\end{proof}

\subsection{Exercise 33}

\begin{proof}

\end{proof}

\subsection{Exercise 34}

\begin{proof}

\end{proof}

\subsection{Exercise 35}

\begin{proof}

\end{proof}

\subsection{Exercise 36}

\begin{proof}

\end{proof}

\subsection{Exercise 37}

\begin{proof}

\end{proof}

\subsection{Exercise 38}

\begin{proof}

\end{proof}

\subsection{Exercise 39}

\begin{proof}

\end{proof}

\subsection{Exercise 40}

\begin{proof}

\end{proof}

\subsection{Exercise 41}

\begin{proof}

\end{proof}

\section{Exercise Set 4.3}

\subsection{Exercise 1}

\begin{proof}

\end{proof}

\subsection{Exercise 2}

\begin{proof}

\end{proof}

\subsection{Exercise 3}

\begin{proof}

\end{proof}

\subsection{Exercise 4}

\begin{proof}

\end{proof}

\subsection{Exercise 5}

\begin{proof}

\end{proof}

\subsection{Exercise 6}

\begin{proof}

\end{proof}

\subsection{Exercise 7}

\begin{proof}

\end{proof}

\subsection{Exercise 8}

\subsubsection{(a)}

\begin{proof}

\end{proof}

\subsubsection{(b)}

\begin{proof}

\end{proof}

\subsubsection{(c)}

\begin{proof}

\end{proof}

\subsection{Exercise 9}

\begin{proof}

\end{proof}

\subsection{Exercise 10}

\begin{proof}

\end{proof}

\subsection{Exercise 11}

\begin{proof}

\end{proof}

\subsection{Exercise 12}

\begin{proof}

\end{proof}

\subsection{Exercise 13}

\subsubsection{(a)}

\begin{proof}

\end{proof}

\subsubsection{(b)}

\begin{proof}

\end{proof}

\subsection{Exercise 14}

\subsubsection{(a)}

\begin{proof}

\end{proof}

\subsubsection{(b)}

\begin{proof}

\end{proof}

\subsection{Exercise 15}

\begin{proof}

\end{proof}

\subsection{Exercise 16}

\begin{proof}

\end{proof}

\subsection{Exercise 17}

\begin{proof}

\end{proof}

\subsection{Exercise 18}

\begin{proof}

\end{proof}

\subsection{Exercise 19}

\begin{proof}

\end{proof}

\subsection{Exercise 20}

\begin{proof}

\end{proof}

\subsection{Exercise 21}

\begin{proof}

\end{proof}

\subsection{Exercise 22}

\begin{proof}

\end{proof}

\subsection{Exercise 23}

\begin{proof}

\end{proof}

\subsection{Exercise 24}

\begin{proof}

\end{proof}

\subsection{Exercise 25}

\begin{proof}

\end{proof}

\subsection{Exercise 26}

\begin{proof}

\end{proof}

\subsection{Exercise 27}

\begin{proof}

\end{proof}

\subsection{Exercise 28}

\begin{proof}

\end{proof}

\subsection{Exercise 29}

\begin{proof}

\end{proof}

\subsection{Exercise 30}

\begin{proof}

\end{proof}

\subsection{Exercise 31}

\begin{proof}

\end{proof}

\subsection{Exercise 32}

\begin{proof}

\end{proof}

\subsection{Exercise 33}

\subsubsection{(a)}

\begin{proof}

\end{proof}

\subsubsection{(b)}

\begin{proof}

\end{proof}

\subsection{Exercise 34}

\subsubsection{(a)}

\begin{proof}

\end{proof}

\subsubsection{(b)}

\begin{proof}

\end{proof}

\subsection{Exercise 35}

\begin{proof}

\end{proof}

\subsection{Exercise 36}

\begin{proof}

\end{proof}

\subsection{Exercise 37}

\begin{proof}

\end{proof}

\subsection{Exercise 38}

\begin{proof}

\end{proof}

\subsection{Exercise 39}

\begin{proof}

\end{proof}

\section{Exercise Set 4.4}

\subsection{Exercise 1}

\begin{proof}

\end{proof}

\subsection{Exercise 2}

\begin{proof}

\end{proof}

\subsection{Exercise 3}

\begin{proof}

\end{proof}

\subsection{Exercise 4}

\begin{proof}

\end{proof}

\subsection{Exercise 5}

\begin{proof}

\end{proof}

\subsection{Exercise 6}

\begin{proof}

\end{proof}

\subsection{Exercise 7}

\begin{proof}

\end{proof}

\subsection{Exercise 8}

\begin{proof}

\end{proof}

\subsection{Exercise 9}

\begin{proof}

\end{proof}

\subsection{Exercise 10}

\begin{proof}

\end{proof}

\subsection{Exercise 11}

\begin{proof}

\end{proof}

\subsection{Exercise 12}

\begin{proof}

\end{proof}

\subsection{Exercise 13}

\begin{proof}

\end{proof}

\subsection{Exercise 14}

\begin{proof}

\end{proof}

\subsection{Exercise 15}

\begin{proof}

\end{proof}

\subsection{Exercise 16}

\begin{proof}

\end{proof}

\subsection{Exercise 17}

\begin{proof}

\end{proof}

\subsection{Exercise 18}

\subsubsection{(a)}

\begin{proof}

\end{proof}

\subsubsection{(b)}

\begin{proof}

\end{proof}

\subsection{Exercise 19}

\begin{proof}

\end{proof}

\subsection{Exercise 20}

\begin{proof}

\end{proof}

\subsection{Exercise 21}

\begin{proof}

\end{proof}

\subsection{Exercise 22}

\begin{proof}

\end{proof}

\subsection{Exercise 23}

\begin{proof}

\end{proof}

\subsection{Exercise 24}

\begin{proof}

\end{proof}

\subsection{Exercise 25}

\begin{proof}

\end{proof}

\subsection{Exercise 26}

\begin{proof}

\end{proof}

\subsection{Exercise 27}

\begin{proof}

\end{proof}

\subsection{Exercise 28}

\begin{proof}

\end{proof}

\subsection{Exercise 29}

\begin{proof}

\end{proof}

\subsection{Exercise 30}

\begin{proof}

\end{proof}

\subsection{Exercise 31}

\begin{proof}

\end{proof}

\subsection{Exercise 32}

\begin{proof}

\end{proof}

\subsection{Exercise 33}

\begin{proof}

\end{proof}

\subsection{Exercise 34}

\begin{proof}

\end{proof}

\subsection{Exercise 35}

\begin{proof}

\end{proof}

\subsection{Exercise 36}

\subsubsection{(a)}

\begin{proof}

\end{proof}

\subsubsection{(b)}

\begin{proof}

\end{proof}

\subsubsection{(c)}

\begin{proof}

\end{proof}

\subsubsection{(d)}

\begin{proof}

\end{proof}

\subsection{Exercise 37}

\subsubsection{(a)}

\begin{proof}

\end{proof}

\subsubsection{(b)}

\begin{proof}

\end{proof}

\subsubsection{(c)}

\begin{proof}

\end{proof}

\subsection{Exercise 38}

\subsubsection{(a)}

\begin{proof}

\end{proof}

\subsubsection{(b)}

\begin{proof}

\end{proof}

\subsubsection{(c)}

\begin{proof}

\end{proof}

\subsubsection{(d)}

\begin{proof}

\end{proof}

\subsection{Exercise 39}

\subsubsection{(a)}

\begin{proof}

\end{proof}

\subsubsection{(b)}

\begin{proof}

\end{proof}

\subsection{Exercise 40}

\subsubsection{(a)}

\begin{proof}

\end{proof}

\subsubsection{(b)}

\begin{proof}

\end{proof}

\subsection{Exercise 41}

\begin{proof}

\end{proof}

\subsection{Exercise 42}

\subsubsection{(a)}

\begin{proof}

\end{proof}

\subsubsection{(b)}

\begin{proof}

\end{proof}

\subsubsection{(c)}

\begin{proof}

\end{proof}

\subsection{Exercise 43}

\begin{proof}

\end{proof}

\subsection{Exercise 44}

\begin{proof}

\end{proof}

\subsection{Exercise 45}

\begin{proof}

\end{proof}

\subsection{Exercise 46}

\begin{proof}

\end{proof}

\subsection{Exercise 47}

\begin{proof}

\end{proof}

\subsection{Exercise 48}

\begin{proof}

\end{proof}

\subsection{Exercise 49}

\begin{proof}

\end{proof}

\subsection{Exercise 50}

\begin{proof}

\end{proof}

\section{Exercise Set 4.5}

\subsection{Exercise 1}

\begin{proof}

\end{proof}

\subsection{Exercise 2}

\begin{proof}

\end{proof}

\subsection{Exercise 3}

\begin{proof}

\end{proof}

\subsection{Exercise 4}

\begin{proof}

\end{proof}

\subsection{Exercise 5}

\begin{proof}

\end{proof}

\subsection{Exercise 6}

\begin{proof}

\end{proof}

\subsection{Exercise 7}

\subsubsection{(a)}

\begin{proof}

\end{proof}

\subsubsection{(b)}

\begin{proof}

\end{proof}

\subsection{Exercise 8}

\subsubsection{(a)}

\begin{proof}

\end{proof}

\subsubsection{(b)}

\begin{proof}

\end{proof}

\subsection{Exercise 9}

\subsubsection{(a)}

\begin{proof}

\end{proof}

\subsubsection{(b)}

\begin{proof}

\end{proof}

\subsection{Exercise 10}

\subsubsection{(a)}

\begin{proof}

\end{proof}

\subsubsection{(b)}

\begin{proof}

\end{proof}

\subsection{Exercise 11}

\subsubsection{(a)}

\begin{proof}

\end{proof}

\subsubsection{(b)}

\begin{proof}

\end{proof}

\subsubsection{(c)}

\begin{proof}

\end{proof}

\subsection{Exercise 12}

\begin{proof}

\end{proof}

\subsection{Exercise 13}

\begin{proof}

\end{proof}

\subsection{Exercise 14}

\begin{proof}

\end{proof}

\subsection{Exercise 15}

\begin{proof}

\end{proof}

\subsection{Exercise 16}

\begin{proof}

\end{proof}

\subsection{Exercise 17}

\begin{proof}

\end{proof}

\subsection{Exercise 18}

\subsubsection{(a)}

\begin{proof}

\end{proof}

\subsubsection{(b)}

\begin{proof}

\end{proof}

\subsection{Exercise 19}

\begin{proof}

\end{proof}

\subsection{Exercise 20}

\begin{proof}

\end{proof}

\subsection{Exercise 21}

\begin{proof}

\end{proof}

\subsection{Exercise 22}

\begin{proof}

\end{proof}

\subsection{Exercise 23}

\begin{proof}

\end{proof}

\subsection{Exercise 24}

\begin{proof}

\end{proof}

\subsection{Exercise 25}

\begin{proof}

\end{proof}

\subsection{Exercise 26}

\begin{proof}

\end{proof}

\subsection{Exercise 27}

\begin{proof}

\end{proof}

\subsection{Exercise 28}

\subsubsection{(a)}

\begin{proof}

\end{proof}

\subsubsection{(b)}

\begin{proof}

\end{proof}

\subsection{Exercise 29}

\subsubsection{(a)}

\begin{proof}

\end{proof}

\subsubsection{(b)}

\begin{proof}

\end{proof}

\subsection{Exercise 30}

\subsubsection{(a)}

\begin{proof}

\end{proof}

\subsubsection{(b)}

\begin{proof}

\end{proof}

\subsection{Exercise 31}

\subsubsection{(a)}

\begin{proof}

\end{proof}

\subsubsection{(b)}

\begin{proof}

\end{proof}

\subsubsection{(c)}

\begin{proof}

\end{proof}

\subsection{Exercise 32}

\begin{proof}

\end{proof}

\subsection{Exercise 33}

\begin{proof}

\end{proof}

\subsection{Exercise 34}

\begin{proof}

\end{proof}

\subsection{Exercise 35}

\begin{proof}

\end{proof}

\subsection{Exercise 36}

\begin{proof}

\end{proof}

\subsection{Exercise 37}

\begin{proof}

\end{proof}

\subsection{Exercise 38}

\begin{proof}

\end{proof}

\subsection{Exercise 39}

\begin{proof}

\end{proof}

\subsection{Exercise 40}

\begin{proof}

\end{proof}

\subsection{Exercise 41}

\begin{proof}

\end{proof}

\subsection{Exercise 42}

\begin{proof}

\end{proof}

\subsection{Exercise 43}

\begin{proof}

\end{proof}

\subsection{Exercise 44}

\subsubsection{(a)}

\begin{proof}

\end{proof}

\subsubsection{(b)}

\begin{proof}

\end{proof}

\subsubsection{(c)}

\begin{proof}

\end{proof}

\subsection{Exercise 45}

\begin{proof}

\end{proof}

\subsection{Exercise 46}

\begin{proof}

\end{proof}

\subsection{Exercise 47}

\begin{proof}

\end{proof}

\subsection{Exercise 48}

\begin{proof}

\end{proof}

\subsection{Exercise 49}

\begin{proof}

\end{proof}

\subsection{Exercise 50}

\begin{proof}

\end{proof}

\section{Exercise Set 4.6}

\subsection{Exercise 1}

\begin{proof}

\end{proof}

\subsection{Exercise 2}

\begin{proof}

\end{proof}

\subsection{Exercise 3}

\begin{proof}

\end{proof}

\subsection{Exercise 4}

\begin{proof}

\end{proof}

\subsection{Exercise 5}

\begin{proof}

\end{proof}

\subsection{Exercise 6}

\begin{proof}

\end{proof}

\subsection{Exercise 7}

\begin{proof}

\end{proof}

\subsection{Exercise 8}

\begin{proof}

\end{proof}

\subsection{Exercise 9}

\begin{proof}

\end{proof}

\subsection{Exercise 10}

\subsubsection{(a)}

\begin{proof}

\end{proof}

\subsubsection{(b)}

\begin{proof}

\end{proof}

\subsection{Exercise 11}

\begin{proof}

\end{proof}

\subsection{Exercise 12}

\begin{proof}

\end{proof}

\subsection{Exercise 13}

\begin{proof}

\end{proof}

\subsection{Exercise 14}

\begin{proof}

\end{proof}

\subsection{Exercise 15}

\begin{proof}

\end{proof}

\subsection{Exercise 16}

\begin{proof}

\end{proof}

\subsection{Exercise 17}

\begin{proof}

\end{proof}

\subsection{Exercise 18}

\begin{proof}

\end{proof}

\subsection{Exercise 19}

\begin{proof}

\end{proof}

\subsection{Exercise 20}

\begin{proof}

\end{proof}

\subsection{Exercise 21}

\begin{proof}

\end{proof}

\subsection{Exercise 22}

\begin{proof}

\end{proof}

\subsection{Exercise 23}

\begin{proof}

\end{proof}

\subsection{Exercise 24}

\begin{proof}

\end{proof}

\subsection{Exercise 25}

\begin{proof}

\end{proof}

\subsection{Exercise 26}

\begin{proof}

\end{proof}

\subsection{Exercise 27}

\begin{proof}

\end{proof}

\subsection{Exercise 28}

\begin{proof}

\end{proof}

\subsection{Exercise 29}

\begin{proof}

\end{proof}

\subsection{Exercise 30}

\begin{proof}

\end{proof}

\subsection{Exercise 31}

\begin{proof}

\end{proof}

\subsection{Exercise 32}

\begin{proof}

\end{proof}

\subsection{Exercise 33}

\begin{proof}

\end{proof}

\section{Exercise Set 4.7}

\subsection{Exercise 1}

\begin{proof}

\end{proof}

\subsection{Exercise 2}

\begin{proof}

\end{proof}

\subsection{Exercise 3}

\begin{proof}

\end{proof}

\subsection{Exercise 4}

\begin{proof}

\end{proof}

\subsection{Exercise 5}

\begin{proof}

\end{proof}

\subsection{Exercise 6}

\begin{proof}

\end{proof}

\subsection{Exercise 7}

\begin{proof}

\end{proof}

\subsection{Exercise 8}

\begin{proof}

\end{proof}

\subsection{Exercise 9}

\subsubsection{(a)}

\begin{proof}

\end{proof}

\subsubsection{(b)}

\begin{proof}

\end{proof}

\subsection{Exercise 10}

\begin{proof}

\end{proof}

\subsection{Exercise 11}

\begin{proof}

\end{proof}

\subsection{Exercise 12}

\subsubsection{(a)}

\begin{proof}

\end{proof}

\subsubsection{(b)}

\begin{proof}

\end{proof}

\subsection{Exercise 13}

\subsubsection{(a)}

\begin{proof}

\end{proof}

\subsubsection{(b)}

\begin{proof}

\end{proof}

\subsection{Exercise 14}

\subsubsection{(a)}

\begin{proof}

\end{proof}

\subsubsection{(b)}

\begin{proof}

\end{proof}

\subsection{Exercise 15}

\begin{proof}

\end{proof}

\subsection{Exercise 16}

\begin{proof}

\end{proof}

\subsection{Exercise 17}

\begin{proof}

\end{proof}

\subsection{Exercise 18}

\begin{proof}

\end{proof}

\subsection{Exercise 19}

\begin{proof}

\end{proof}

\subsection{Exercise 20}

\begin{proof}

\end{proof}

\subsection{Exercise 21}

\subsubsection{(a)}

\begin{proof}

\end{proof}

\subsubsection{(b)}

\begin{proof}

\end{proof}

\subsection{Exercise 22}

\subsubsection{(a)}

\begin{proof}

\end{proof}

\subsubsection{(b)}

\begin{proof}

\end{proof}

\subsection{Exercise 23}

\begin{proof}

\end{proof}

\subsection{Exercise 24}

\begin{proof}

\end{proof}

\subsection{Exercise 25}

\begin{proof}

\end{proof}

\subsection{Exercise 26}

\begin{proof}

\end{proof}

\subsection{Exercise 27}

\begin{proof}

\end{proof}

\subsection{Exercise 28}

\begin{proof}

\end{proof}

\subsection{Exercise 29}

\begin{proof}

\end{proof}

\subsection{Exercise 30}

\subsubsection{(a)}

\begin{proof}

\end{proof}

\subsubsection{(b)}

\begin{proof}

\end{proof}

\subsection{Exercise 31}

\subsubsection{(a)}

\begin{proof}

\end{proof}

\subsubsection{(b)}

\begin{proof}

\end{proof}

\subsubsection{(c)}

\begin{proof}

\end{proof}

\subsection{Exercise 32}

\begin{proof}

\end{proof}

\subsection{Exercise 33}

\begin{proof}

\end{proof}

\subsection{Exercise 34}

\subsubsection{(a)}

\begin{proof}

\end{proof}

\subsubsection{(b)}

\begin{proof}

\end{proof}

\subsubsection{(c)}

\begin{proof}

\end{proof}

\subsubsection{(d)}

\begin{proof}

\end{proof}

\subsection{Exercise 35}

\begin{proof}

\end{proof}

\subsection{Exercise 36}

\begin{proof}

\end{proof}

\section{Exercise Set 4.8}

\subsection{Exercise 1}

\begin{proof}

\end{proof}

\subsection{Exercise 2}

\begin{proof}

\end{proof}

\subsection{Exercise 3}

\begin{proof}

\end{proof}

\subsection{Exercise 4}

\begin{proof}

\end{proof}

\subsection{Exercise 5}

\begin{proof}

\end{proof}

\subsection{Exercise 6}

\begin{proof}

\end{proof}

\subsection{Exercise 7}

\begin{proof}

\end{proof}

\subsection{Exercise 8}

\begin{proof}

\end{proof}

\subsection{Exercise 9}

\begin{proof}

\end{proof}

\subsection{Exercise 10}

\begin{proof}

\end{proof}

\subsection{Exercise 11}

\begin{proof}

\end{proof}

\subsection{Exercise 12}

\begin{proof}

\end{proof}

\subsection{Exercise 13}

\begin{proof}

\end{proof}

\subsection{Exercise 14}

\begin{proof}

\end{proof}

\subsection{Exercise 15}

\begin{proof}

\end{proof}

\subsection{Exercise 16}

\begin{proof}

\end{proof}

\subsection{Exercise 17}

\begin{proof}

\end{proof}

\subsection{Exercise 18}

\subsubsection{(a)}

\begin{proof}

\end{proof}

\subsubsection{(b)}

\begin{proof}

\end{proof}

\subsection{Exercise 19}

\subsubsection{(a)}

\begin{proof}

\end{proof}

\subsubsection{(b)}

\begin{proof}

\end{proof}

\subsubsection{(c)}

\begin{proof}

\end{proof}

\subsection{Exercise 20}

\begin{proof}

\end{proof}

\subsection{Exercise 21}

\begin{proof}

\end{proof}

\subsection{Exercise 22}

\begin{proof}

\end{proof}

\subsection{Exercise 23}

\begin{proof}

\end{proof}

\subsection{Exercise 24}

\begin{proof}

\end{proof}

\subsection{Exercise 25}

\begin{proof}

\end{proof}

\subsection{Exercise 26}

\begin{proof}

\end{proof}

\subsection{Exercise 27}

\begin{proof}

\end{proof}

\subsection{Exercise 28}

\begin{proof}

\end{proof}

\subsection{Exercise 29}

\begin{proof}

\end{proof}

\subsection{Exercise 30}

\subsubsection{(a)}

\begin{proof}

\end{proof}

\subsubsection{(b)}

\begin{proof}

\end{proof}

\subsection{Exercise 31}

\begin{proof}

\end{proof}

\subsection{Exercise 32}

\begin{proof}

\end{proof}

\subsection{Exercise 33}

\begin{proof}

\end{proof}

\subsection{Exercise 34}

\subsubsection{(a)}

\begin{proof}

\end{proof}

\subsubsection{(b)}

\begin{proof}

\end{proof}

\subsection{Exercise 35}

\begin{proof}

\end{proof}

\subsection{Exercise 36}

\begin{proof}

\end{proof}

\subsection{Exercise 37}

\begin{proof}

\end{proof}

\subsection{Exercise 38}

\begin{proof}

\end{proof}

\section{Exercise Set 4.9}

\subsection{Exercise 1}

\begin{proof}

\end{proof}

\subsection{Exercise 2}

\begin{proof}

\end{proof}

\subsection{Exercise 3}

\begin{proof}

\end{proof}

\subsection{Exercise 4}

\begin{proof}

\end{proof}

\subsection{Exercise 5}

\begin{proof}

\end{proof}

\subsection{Exercise 6}

\begin{proof}

\end{proof}

\subsection{Exercise 7}

\begin{proof}

\end{proof}

\subsection{Exercise 8}

\begin{proof}

\end{proof}

\subsection{Exercise 9}

\begin{proof}

\end{proof}

\subsection{Exercise 10}

\begin{proof}

\end{proof}

\subsection{Exercise 11}

\begin{proof}

\end{proof}

\subsection{Exercise 12}

\begin{proof}

\end{proof}

\subsection{Exercise 13}

\begin{proof}

\end{proof}

\subsection{Exercise 14}

\subsubsection{(a)}

\begin{proof}

\end{proof}

\subsubsection{(b)}

\begin{proof}

\end{proof}

\subsection{Exercise 15}

\subsubsection{(a)}

\begin{proof}

\end{proof}

\subsubsection{(b)}

\begin{proof}

\end{proof}

\subsection{Exercise 16}

\subsubsection{(a)}

\begin{proof}

\end{proof}

\subsubsection{(b)}

\begin{proof}

\end{proof}

\subsection{Exercise 17}

\begin{proof}

\end{proof}

\subsection{Exercise 18}

\begin{proof}

\end{proof}

\subsection{Exercise 19}

\begin{proof}

\end{proof}

\subsection{Exercise 20}

\subsubsection{(a)}

\begin{proof}

\end{proof}

\subsubsection{(b)}

\begin{proof}

\end{proof}

\subsection{Exercise 21}

\subsubsection{(a)}

\begin{proof}

\end{proof}

\subsubsection{(b)}

\begin{proof}

\end{proof}

\subsubsection{(c)}

\begin{proof}

\end{proof}

\subsection{Exercise 22}

\begin{proof}

\end{proof}

\subsection{Exercise 23}

\subsubsection{(a)}

\begin{proof}

\end{proof}

\subsubsection{(b)}

\begin{proof}

\end{proof}

\subsubsection{(c)}

\begin{proof}

\end{proof}

\subsubsection{(d)}

\begin{proof}

\end{proof}

\subsubsection{(e)}

\begin{proof}

\end{proof}

\subsubsection{(f)}

\begin{proof}

\end{proof}

\subsection{Exercise 24}

\subsubsection{(a)}

\begin{proof}

\end{proof}

\subsubsection{(b)}

\begin{proof}

\end{proof}

\subsubsection{(c)}

\begin{proof}

\end{proof}

\subsubsection{(d)}

\begin{proof}

\end{proof}

\subsubsection{(e)}

\begin{proof}

\end{proof}

\subsubsection{(f)}

\begin{proof}

\end{proof}

\subsection{Exercise 25}

\begin{proof}

\end{proof}

\section{Exercise Set 4.10}

\subsection{Exercise 1}

\begin{proof}

\end{proof}

\subsection{Exercise 2}

\begin{proof}

\end{proof}

\subsection{Exercise 3}

\subsubsection{(a)}

\begin{proof}

\end{proof}

\subsubsection{(b)}

\begin{proof}

\end{proof}

\subsection{Exercise 4}

\begin{proof}

\end{proof}

\subsection{Exercise 5}

\begin{proof}

\end{proof}

\subsection{Exercise 6}

\begin{proof}

\end{proof}

\subsection{Exercise 7}

\begin{proof}

\end{proof}

\subsection{Exercise 8}

\subsubsection{(a)}

\begin{proof}

\end{proof}

\subsubsection{(b)}

\begin{proof}

\end{proof}

\subsection{Exercise 9}

\begin{proof}

\end{proof}

\subsection{Exercise 10}

\begin{proof}

\end{proof}

\subsection{Exercise 11}

\begin{proof}

\end{proof}

\subsection{Exercise 12}

\begin{proof}

\end{proof}

\subsection{Exercise 13}

\begin{proof}

\end{proof}

\subsection{Exercise 14}

\begin{proof}

\end{proof}

\subsection{Exercise 15}

\begin{proof}

\end{proof}

\subsection{Exercise 16}

\begin{proof}

\end{proof}

\subsection{Exercise 17}

\begin{proof}

\end{proof}

\subsection{Exercise 18}

\begin{proof}

\end{proof}

\subsection{Exercise 19}

\begin{proof}

\end{proof}

\subsection{Exercise 20}

\begin{proof}

\end{proof}

\subsection{Exercise 21}

\begin{proof}

\end{proof}

\subsection{Exercise 22}

\begin{proof}

\end{proof}

\subsection{Exercise 23}

\subsubsection{(a)}

\begin{proof}

\end{proof}

\subsubsection{(b)}

\begin{proof}

\end{proof}

\subsection{Exercise 24}

\begin{proof}

\end{proof}

\subsection{Exercise 25}

\subsubsection{(a)}

\begin{proof}

\end{proof}

\subsubsection{(b)}

\begin{proof}

\end{proof}

\subsection{Exercise 26}

\subsubsection{(a)}

\begin{proof}

\end{proof}

\subsubsection{(b)}

\begin{proof}

\end{proof}

\subsection{Exercise 27}

\subsubsection{(a)}

\begin{proof}

\end{proof}

\subsubsection{(b)}

\begin{proof}

\end{proof}

\subsubsection{(c)}

\begin{proof}

\end{proof}

\subsection{Exercise 28}

\subsubsection{(a)}

\begin{proof}

\end{proof}

\subsubsection{(b)}

\begin{proof}

\end{proof}

\subsubsection{(c)}

\begin{proof}

\end{proof}

\subsection{Exercise 29}

\begin{proof}

\end{proof}

\subsection{Exercise 30}

\begin{proof}

\end{proof}

\subsection{Exercise 31}

\begin{proof}

\end{proof}

\subsection{Exercise 32}

\begin{proof}

\end{proof}

\end{document}
