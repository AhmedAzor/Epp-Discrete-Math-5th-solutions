\documentclass[14pt]{extarticle}

\usepackage[table]{xcolor}
\usepackage{amsmath,mathtools,amsfonts,amsthm,amssymb,hyperref,wasysym,pifont}
\usepackage{parskip,geometry,latexsym,bookmark,mathtools,float,cancel,tcolorbox}

\newtheorem{defn}{Definition}
\newtheorem{thm}{Theorem}
\newtheorem{claim}{Claim}
\newtheorem{lemma}{Lemma}

\newcommand{\dps}{\displaystyle}
\newcommand{\fbl}{\underline{\hspace{1cm}}\,\,}
\newcommand{\R}{\mathbb{R}}
\newcommand{\Z}{\mathbb{Z}}
\newcommand{\from}{\leftarrow}
\newcommand{\true}{{\bf t}}
\newcommand{\false}{{\bf c}}
\newcommand{\bic}{\leftrightarrow}
\newcommand{\base}[1]{{\color{cyan}#1}}
\newcommand{\da}{\downarrow}
\newcommand{\fa}{\forall}
\newcommand{\te}{\exists}
\newcommand{\cy}{\color{cyan}}

\hypersetup{colorlinks,allcolors=blue,linktoc=all}
\geometry{a4paper}
\geometry{margin=0.42in}

\title{Chapter 4 Solutions, Susanna Epp Discrete Math 5th Edition}

\author{https://github.com/spamegg1}

\begin{document}
\maketitle
\tableofcontents

\section{Exercise Set 4.1}

\subsection{Exercise 1}
Assume that $k$ is a particular integer.

\subsubsection{(a)}
Is $-17$ an odd integer?

\begin{proof}
Yes: $-17 = 2(-9) + 1$.

\end{proof}

\subsubsection{(b)}
Is 0 neither even nor odd?

\begin{proof}
No. 0 is even because $0 = 0 \cdot 2$.
\end{proof}

\subsubsection{(c)}
Is $2k - 1$ odd?

\begin{proof}
Yes: $2k - 1 = 2(k - 1) + 1$ and $k - 1$ is an integer because it is a difference of integers.
\end{proof}

\subsection{Exercise 2}
Assume that $c$ is a particular integer.

\subsubsection{(a)}
Is $-6c$ an even integer?

\begin{proof}
Yes, because $-6c = 2 \cdot (-3c) = 2k$ where $k = -3c$ is an integer.
\end{proof}

\subsubsection{(b)}
Is $8c + 5$ an odd integer?

\begin{proof}
Yes, because $8c + 5 = 2(4c + 2) + 1$ and $k = 4k+2$ is an integer.
\end{proof}

\subsubsection{(c)}
Is $(c^2 + 1) - (c^2 - 1) - 2$ an even integer?

\begin{proof}
Yes, because it equals 0: $(c^2 + 1) - (c^2 - 1) - 2 = c^2 + 1 - c^2 + 1 - 2 = 2 - 2 = 0$.
\end{proof}

\subsection{Exercise 3}
Assume that $m$ and $n$ are particular integers.

\subsubsection{(a)}
Is $6m + 8n$ even?

\begin{proof}
Yes: $6m + 8n = 2(3m + 4n)$ and $(3m + 4n)$ is an integer because $3, 4, m$, and $n$ are integers, and products and sums of integers are integers.
\end{proof}

\subsubsection{(b)}
Is $10mn + 7$ odd?

\begin{proof}
Yes: $10mn + 7 = 2(5mn + 3) + 1$ and $5mn + 3$ is an integer because $3, 5, m$, and $n$ are integers, and products and sums of integers are integers.
\end{proof}

\subsubsection{(c)}
If $m > n > 0$, is $m^2 - n^2$ composite?

\begin{proof}
Not necessarily. For instance, if $m = 3$ and $n = 2$, then $m^2 - n^2 = 9 - 4 = 5$, which is prime. (However, $m^2 - n^2$ is composite for many values of $m$ and $n$ because of the identity $m^2 - n^2 = (m - n)(m + n)$.)
\end{proof}

\subsection{Exercise 4}
Assume that $r$ and $s$ are particular integers.

\subsubsection{(a)}
Is $4rs$ even?

\begin{proof}
Yes: $4rs = 2(2rs)$ and $2rs$ is an integer because $2, r, s$ are integers, and products of integers are integers.
\end{proof}

\subsubsection{(b)}
Is $6r + 4s^2 + 3$ odd?

\begin{proof}
Yes: $6r + 4s^2 + 3 = 2(3r + 2s^2 + 1) + 1$ and $3r + 2s^2 + 1$ is an integer because $3, r, 2, s, 1$ are integers and products and sums of integers are integers.
\end{proof}

\subsubsection{(c)}
If $r$ and $s$ are both positive, is $r^2 + 2rs + s^2$ composite?

\begin{proof}
Yes: $r^2 + 2rs + s^2 = (r+s)(r+s)$ and $r + s \geq 2$, therefore $r^2 + 2rs + s^2$ is a product of two integers both of which are greater than 1.
\end{proof}

{\bf \cy Prove the statements in 5–11.}

\subsection{Exercise 5}
There are integers $m$ and $n$ such that $m > 1$ and $n > 1$ and $\frac{1}{m} + \frac{1}{n}$ is an integer.

\begin{proof}
For example, let $m = n = 2$. Then $m$ and $n$ are integers such that $m > 1$ and $n > 1$ and $\frac{1}{m} + \frac{1}{n} = \frac{1}{2} + \frac{1}{2} = 1$ which is an integer.
\end{proof}

\subsection{Exercise 6}
There are distinct integers $m$ and $n$ such that $\frac{1}{m} + \frac{1}{n}$ is an integer.

\begin{proof}
For example, let $m = 1, n = -1$. Then $m$ and $n$ are integers such that $\frac{1}{m} + \frac{1}{n} = \frac{1}{1} - \frac{1}{1} = 0$ which is an integer.
\end{proof}

\subsection{Exercise 7}
There are real numbers $a$ and $b$ such that $\sqrt{a+b} = \sqrt{a} \sqrt{b}$.

\begin{proof}
For example, let $a = 0, b = 0$. Then $a$ and $b$ are real numbers such that 

$\sqrt{a+b} = \sqrt{0+0} = 0 = \sqrt{0} + \sqrt{0} = \sqrt{a} + \sqrt{b}$.
\end{proof}

\subsection{Exercise 8}
There is an integer $n > 5$ such that $2^n - 1$ is prime.

\begin{proof}
For example, let $n = 7$. Then $n$ is an integer such that
$n > 5$ and $2^n - 1 = 127$, which is prime.
\end{proof}

\subsection{Exercise 9}
There is a real number $x$ such that $x > 1$ and $2^x > x^{10}$.

\begin{proof}
For example, take $x = 80$. Then 

$x^{10} = 80^{10} = 8^{10} \cdot 10^{10} = (2^3)^{10} \cdot 10^{10} = 2^{30} \cdot 10^{10}$.

We have $2^{50} \approx 1,125899907 \cdot 10^{15} > 10^{10}$. So $2^{80} = 2^{30} \cdot 2^{50} > 2^{30} \cdot 10^{10} = 80^{10}$.

Therefore $x = 80$ is a real number such that $x > 1$ and $2^x > x^{10}$.
\end{proof}

\begin{tcolorbox}[colframe=cyan]
{\bf \cy Definition:} An integer $n$ is called a {\bf perfect square} if, and only if, $n = k^2$ for some integer $k$.
\end{tcolorbox}

\subsection{Exercise 10}
There is a perfect square that can be written as the sum of two other perfect squares.

\begin{proof}
For example, 25, 9, and 16 are all perfect squares, because $25 = 5^2, 9 = 3^2$, and $16 = 4^2$, and $25 = 9 + 16$. Thus 25 is a perfect square that can be written as a sum of two other perfect squares.
\end{proof}

\subsection{Exercise 11}
There is an integer $n$ such that $2n^2 - 5n + 2$ is prime.

\begin{proof}
For example, take $n = 3$. Then $2n^2 - 5n + 2 = 18 - 15 + 2 = 5$ is prime. (You can find this value of $n$ by either starting at $n = 1$ and using trial and error, or noticing that $2n^2 - 5n + 2 = (2n - 1)(n - 2)$, so, for this to be prime, one of the factors has to be 1.)
\end{proof}

{\bf \cy In $12-13$, (a) write a negation for the given statement, and (b) use a counterexample to disprove the given statement. explain how the counterexample actually shows that the given statement is false.}

\subsection{Exercise 12}
For all real numbers $a$ and $b$, if $a < b$ then $a^2 < b^2$.

\begin{proof}
a. {\it Negation for the statement:} There exist real numbers $a$ and $b$ such that $a < b$ and $a^2 \nless b^2$.

b. {\it Counterexample for the statement:} Let $a = -2$ and
$b = -1$. Then $a < b$ because $-2 < -1$, but $a^2 \nless b^2$ because $(-2)^2 = 4$ and $(-1)^2 = 1$ and $4 \nless 1$. {\it [So the hypothesis of the statement is true and its conclusion is false.]}
\end{proof}

\subsection{Exercise 13}
For every integer $n$, if $n$ is odd then $\frac{n-1}{2}$ is odd.

\begin{proof}
a. {\it Negation for the statement:} There exists an integer $n$ such that $n$ is odd and $\frac{n-1}{2}$ is not odd.

b. {\it Counterexample for the statement:} Let $n = 5$. Then $n$ is odd because $5 = 2 \cdot 2 + 1$, but $\frac{n-1}{2}$ is not odd because $\frac{5-1}{2} = 2 = 2 \cdot 1$ is even. {\it [So the hypothesis of the statement is true and its conclusion is false.]}
\end{proof}

{\bf \cy Disprove each of the statements in $14-16$ by giving a counterexample. In each case explain how the counterexample actually disproves the statement.}

\subsection{Exercise 14}
For all integers $m$ and $n$, if $2m + n$ is odd then $m$ and $n$ are both odd.

\begin{proof}
\underline{Counterexample:} Let $m = 2$ and $n = 1$. Then $2m + n = 2 \cdot 2 + 1 = 5$, which is odd. But $m$ is not odd, and so it is false that both $m$ and $n$ are odd. {\it [This is one counterexample among many.]}
\end{proof}

\subsection{Exercise 15}
For every integer $p$, if $p$ is prime then $p^2 - 1$ is even.

\begin{proof}
\underline{Counterexample:} Let $p = 2$ which is prime, but $p^2 - 1 = 2^2 - 1 = 4 - 1 = 3$ is not even. {\it [This is the only counterexample! For every other prime $p$, $p^2 - 1$ is even.]}
\end{proof}

\subsection{Exercise 16}
For every integer $n$, if $n$ is even then $n^2 + 1$ is prime.

\begin{proof}
\underline{Counterexample:} Let $n = 8$. Then $n$ is even. But $n^2 + 1 = 65 = 13 \cdot 5$ is not prime. {\it [This is one counterexample among many.]}
\end{proof}

{\bf \cy In $17-20$, determine whether the property is true for all integers, true for no integers, or true for some integers and false for other integers. Justify your answers.}

\subsection{Exercise 17}
$(a+b)^2 = a^2 + b^2$

\begin{proof}
This property is true for some integers and false for other integers. For instance, if $a = 0$ and $b = 1$, the property is true because $(0 + 1)^2 = 0^2 + 1^2$, but if $a = 1$ and $b = 1$, the property is false because $(1 + 1)^2 = 4$ and $1^2 + 1^2 = 2$ and $4 \neq 2$.
\end{proof}

\subsection{Exercise 18}
$\dps \frac{a}{b} + \frac{c}{d} = \frac{a+c}{b+d}$

\begin{proof}
True for some integers, false for others. For example, if $a = c = 0$ and $b = d = 1$ then $\dps \frac{0}{1} + \frac{0}{1} = 0 = \frac{0+0}{1+1}$ is true. But if $a = 1, b = 2, c = 3$ and $d = 4$ then 

$\dps \frac{a}{b} + \frac{c}{d} = \frac{1}{2} + \frac{3}{4} = \frac{5}{4} \neq \frac{4}{6} = \frac{1+3}{2+4} = \frac{a+c}{b+d}$.
\end{proof}

\subsection{Exercise 19}
$-a^n = (-a)^n$

{\it Hint:} This property is true for some integers and false for other integers. To justify this answer you need to find examples of both.

\begin{proof}
True for some integers: let $a = 0, n = 1$. Then $-0^1 = -0 = 0$ and $(-0)^1 = 0^1 = 0$ so the equality holds. When $a = n = 2$ it is false: $-2^2 = -4$ but $(-2)^2 = 4$ and $-4 \neq 4$.
\end{proof}

\subsection{Exercise 20}
The average of any two odd integers is odd.

\begin{proof}
True for some, false for others. For example, 3 and 7 are both odd, and their average $(3+7)/2 = 5$ is odd. But 3 and 5 are both odd, and their average is $(3 + 5) / 2 = 4$ is even.
\end{proof}

{\bf \cy Prove the statement in 21 and 22 by the method of exhaustion.}

\subsection{Exercise 21}
Every positive even integer less than 26 can be expressed as a sum of three of fewer perfect squares. (For instance, $10 = 1^2 + 3^2$ and $16 = 4^2$.)

\begin{proof}
$2 = 1^2 + 1^2$

$4 = 2^2$

$6 = 2^2 + 1^2 + 1^2$

$8 = 2^2 + 2^2$

$10 = 1^2 + 3^2$

$12 = 2^2 + 2^2 + 2^2$

$16 = 4^2$

$18 = 4^2 + 1^2 + 1^2$

$20 = 4^2 + 2^2$

$22 = 2^2 + 2^2 + 3^2$

$24 = 4^2 + 2^2 + 2^2$
\end{proof}

\subsection{Exercise 22}
For each integer $n$ with $1 \leq n \leq 10$, $n^2 - n + 11$ is a prime number.

\begin{proof}
$1^2 - 1 + 11 = 11$ is prime.

$2^2 - 2 + 11 = 13$ is prime.

$3^2 - 3 + 11 = 17$ is prime.

$4^2 - 4 + 11 = 23$ is prime.

$5^2 - 5 + 11 = 31$ is prime.

$6^2 - 6 + 11 = 41$ is prime.

$7^2 - 7 + 11 = 53$ is prime.

$8^2 - 8 + 11 = 67$ is prime.

$9^2 - 9 + 11 = 83$ is prime.

$10^2 - 10 + 11 = 101$ is prime.
\end{proof}

{\bf \cy Each of the statements in $23-26$ is true. For each, (a) rewrite the statement with the quantification implicit as If \fbl, then \fbl, and (b) write the first sentence of a proof (the “starting point”) and the last sentence of a proof (the “conclusion to be shown”). (Note that you do not need to understand the statements in order to be able to do these exercises.)}

\subsection{Exercise 23}
For every integer $m$, if $m > 1$ then $0 < \frac{1}{m} < 1$.

\begin{proof}
a. If an integer is greater than 1, then its reciprocal is
between 0 and 1.

b. {\it Start of proof:} Suppose $m$ is any integer such that $m > 1$. {\it Conclusion to be shown:} $0 < 1/m < 1$.
\end{proof}

\subsection{Exercise 24}
For every real number $x$, if $x > 1$ then $x^2 > x$.

\begin{proof}
a. If a real number is greater than 1, then its square is greater than itself.

b. {\it Start of proof:} Suppose $x$ is any real number such that $x > 1$. {\it Conclusion to be shown:} $x^2 > x$.
\end{proof}

\subsection{Exercise 25}
For all integers $m$ and $n$, if $mn = 1$ then $m = n = 1$ or $m = n = -1$.

\begin{proof}
a. If the product of two integers is 1, then either both are 1 or both are $-1$.

b. {\it Start of proof:} Suppose $m$ and $n$ are any integers with $mn = 1$. {\it Conclusion to be shown:} $m = n = 1$ or $m = n = -1$.
\end{proof}

\subsection{Exercise 26}
For every real number $x$, if $0 < x < 1$ then $x^2 < x$.

\begin{proof}
a. If a real number is strictly between 0 and 1, then its square is less than itself.

b. {\it Start of proof:} Suppose $x$ is any real number such that $0 < x < 1$. {\it Conclusion to be shown:} $x^2 < x$.
\end{proof}

\subsection{Exercise 27}
Fill in the blanks in the following proof.

{\bf Theorem:} For every odd integer $n$, $n^2$ is odd.

{\bf Proof:} Suppose $n$ is any {\cy (a)} \fbl. By definition of odd, $n = 2k + 1$ for some integer $k$. Then

\begin{center}
\begin{tabular}{rcll}
$n^2$ & = & {\cy (b)} $(\fbl)^2$ & \cy by substitution \\
& = & $4k^2 + 4k + 1$ & \cy by multiplying \\
& = & $2(2k^2 + 2k) + 1$ & \cy by factoring out a 2 \\
\end{tabular}
\end{center}

Now $2k^2 + 2k$ is an integer because it is a sum of products of integers. Therefore, $n^2$ equals $2\cdot$ (an integer) $+ 1$, and so {\cy (c)} \fbl is odd by definition of odd.

Because we have not assumed anything about $n$ except that it is an odd integer, it follows from the principle of {\cy (d)} \fbl that for every odd integer $n$, $n^2$ is odd.

\begin{proof}
(a) particular but arbitrarily chosen odd integer 
(b) $2k + 1$ (c) $n^2$ (d) universal generalization
\end{proof}

{\bf \cy In each of $28-31$: \\ 
a. Rewrite the theorem in three different ways: as $\fa$ \fbl, if \fbl, then \fbl; as $\fa$ \fbl, \fbl (without using the words if or then), and as If \fbl, then \fbl (without using an explicit universal quantifier). \\
b. Fill in the blanks in the proof of the theorem.}

\subsection{Exercise 28}
{\bf Theorem:} The sum of any two odd integers is even.

{\bf Proof:} Suppose $m$ and $n$ are any {\it [particular but arbitrarily chosen]} odd integers. 

{\it [We must show that $m + n$ is even.]}

By {\cy (a)} \fbl, $m = 2r + 1$ and $n = 2s + 1$ for some integers $r$ and $s$. Then

\begin{center}
\begin{tabular}{rcll}
$m+n$ & = & $(2r+1) + (2s+1)$ & {\cy by (b)} \fbl \\
& = & $2r + 2s + 2$ & \\
& = & $2(r+s+1)$ & \cy by algebra \\
\end{tabular}
\end{center}

Let $u = r + s + 1$. Then $u$ is an integer because $r, s$ and 1 are integers and because {\cy (c)} \fbl. 

Hence $m + n = 2u$, where $u$ is an integer, and so, by {\cy (d)} \fbl, $m + n$ is even {\it [as was to be shown]}.

\begin{proof}
a. $\fa$ integers $m$ and $n$, if $m$ and $n$ are odd then $m + n$ is even.

$\fa$ odd integers $m$ and $n$, $m + n$ is even.

If $m$ and $n$ are any odd integers, then $m + n$ is even.

b. (a) definition of odd, (b) substitution, (c) any sum of
integers is an integer, (d) definition of even
\end{proof}

\subsection{Exercise 29}
{\bf Theorem:} The negative of any even integer is even.

{\bf Proof:} Suppose $n$ is any {\it [particular but arbitrarily chosen]} even integer. 

{\it [We must show that $-n$ is even.]}

By {\cy (a)} \fbl, $n = 2k$ for some integer $k$.

Then

\begin{center}
\begin{tabular}{rcll}
$-n$ & = & $-(2k)$ & \cy by (b) \fbl \\
& = & $2(-k)$ & \cy by algebra \\
\end{tabular}
\end{center}

Let $r = -k$. Then $r$ is an integer because $-1$ and $k$ are integers and {\cy (c)} \fbl. 

Hence $-n = 2r$, where $r$ is an integer, and so $-n$ is even by {\cy (d)} \fbl {\it [as was to be shown]}.

\begin{proof}
a. $\fa$ integer $n$, if $n$ is even then $-n$ is even.

$\fa$ even integer $n$, $-n$ is even.

If $n$ is any even integer, then $-n$ is even.

b. (a) definition of even, (b) substitution, (c) any product of integers is an integer, (d) definition of even
\end{proof}

\subsection{Exercise 30}
{\bf Theorem:} The sum of any even integer and any odd integer is odd.

{\bf Proof:} Suppose $m$ is any even integer and $n$ is any {\cy (a)} \fbl. By definition of even, $m = 2r$ for some {\cy (b)} \fbl, and by definition of odd, $n = 2s + 1$ for some integer $s$. By substitution and algebra,

\begin{center}
\begin{tabular}{ccccc}
$m+n$ & = & {\cy (c)} \fbl & = & $2(r+s)+1$ \\
\end{tabular}
\end{center}

Since $r$ and $s$ are integers, so is their sum $r+s$. Hence $m+n$ has the form twice some integer plus one, and so, by {\cy (d)} \fbl by definition of odd.

\begin{proof}
a. $\fa$ integers $m$ and $n$, if $m$ is even and $n$ is odd, then $m + n$ is odd.

$\fa$ even integers $m$ and odd integers $n$, $m + n$ is odd.

If $m$ is any even integer and $n$ is any odd integer,
then $m + n$ is odd.

b. (a) any odd integer (b) integer $r$ (c) $2r + (2s + 1)$ (d) $m + n$ is odd
\end{proof}

\subsection{Exercise 31}
{\bf Theorem:} Whenever $n$ is an odd integer, $5n^2 + 7$ is even.

{\bf Proof:} Suppose $n$ is any {\it [particular but arbitrarily chosen]} odd integer. 

{\it [We must show that $5n^2 + 7$ is even.]}

By definition of odd, $n = $ {\cy (a)} \fbl, for some integer $k$. 

Then

\begin{center}
\begin{tabular}{rcll}
$5n^2 + 7$ & = & {\cy (b)} \fbl & \cy substitution \\
& = & $5(4k^2 + 4k + 1) + 7$ & \\
& = & $20k^2 + 20k + 12$ & \\
& = & $2(10k^2 + 10k + 6)$ & \cy by algebra \\
\end{tabular}
\end{center}

Let $t = $ {\cy (c)} \fbl. Then $t$ is an integer because products and sums of integers are integers. 

Hence $5n^2 + 7 = 2t$, where $t$ is an integer, and thus {\cy (d)} \fbl by definition of even {\it [as was to be shown]}.

\begin{proof}
a. $\fa$ integer $n$, if $n$ is odd, then $5n^2+7$ is even.

$\fa$ odd integer $n$, $5n^2+7$ is even.

If $n$ is any odd integer, then $5n^2+7$ is even.

b. (a) $2k+1$ (b) $5(2k+1)^2 + 7$ (c) $10k^2 + 10k + 6$ (d) $5n^2+7$ is even
\end{proof}

\section{Exercise Set 4.2}

{\bf \cy Prove the statements in $1-11$. In each case use only the definitions of the terms and the assumptions listed on page 161, not any previously established properties of odd and even integers. Follow the directions given in this section for writing proofs of universal statements.}

\subsection{Exercise 1}
For every integer $n$, if $n$ is odd then $3n + 5$ is even.

\begin{proof}
Suppose $n$ is any {\it [particular but arbitrarily chosen]} odd integer. 

{\it [We must show that $3n + 5$ is even. By definition of even, this means we must show that $3n + 5 = 2\cdot$(some integer).]}

By definition of odd, $n = 2r + 1$, for some integer $r$. 

Then

\begin{center}
\begin{tabular}{rcll}
$3n + 5$ & = & $3(2r + 1) + 5$ & \cy by substitution \\
& = & $6r + 3 + 5$ & \\
& = & $6r + 8$ & \\
& = & $2(3r + 4)$ & \cy by algebra \\
\end{tabular}
\end{center}

{\it [Idea for the rest of the proof: We want to show that $3n + 5 = 2\cdot$(some integer). At this point we know that $3n + 5 = 2(3r + 4)$. So is $3r + 4$ an integer? Yes, because products and sums of integers are integers.]}

Let $k = 3r + 4$. 

Then $k$ is an integer because products and sums of integers are integers. 

Hence $3n + 5 = 2(3r+4) = 2k$ where $k$ is an integer. Hence by definition of even $3n+5$ is even {\it [as was to be shown]}.
\end{proof}

\subsection{Exercise 2}
For every integer $m$, if $m$ is even then $3m + 5$ is odd.

\begin{proof}
Suppose $m$ is any {\it [particular but arbitrarily chosen]} even integer. 

{\it [We must show that $3m + 5$ is odd. By definition of odd, this means we must show that $3m + 5 = 2\cdot\text{(some integer)} + 1$.]}

By definition of even, $m = 2r$, for some integer $r$. 

Then

\begin{center}
\begin{tabular}{rcll}
$3m + 5$ & = & $3(2r) + 5$ & \cy by substitution \\
& = & $6r + 5$ & \\
& = & $6r + 4 + 1$ & \\
& = & $2(3r + 2) + 1$ & \cy by algebra \\
\end{tabular}
\end{center}

{\it [Idea for the rest of the proof: We want to show that $3m + 5 = 2\cdot\text{(some integer)} + 1$. At this point we know that $3m + 5 = 2(3r + 2) + 1$. So is $3r + 2$ an integer? Yes, because products and sums of integers are integers.]}

Let $k = 3r + 2$. 

Then $k$ is an integer because products and sums of integers are integers. 

Hence $3m + 5 = 2(3r+2) + 1 = 2k + 1$ where $k$ is an integer. Hence by definition of odd $3n+5$ is odd {\it [as was to be shown]}.
\end{proof}

\subsection{Exercise 3}
For every integer $n$, $2n - 1$ is odd.

\begin{proof}
Suppose $n$ is any {\it [particular but arbitrarily chosen]} integer. 

{\it [We must show that $2n - 1$ is odd. By definition of odd, this means we must show that $2n - 1 = 2 \cdot \text{(some integer)} + 1$.]}

Then

\begin{center}
\begin{tabular}{rcll}
$2n-1$ & = & $2n - 2 + 2 - 1$ & \cy because $-2 + 2 = 0$ \\
& = & $2(n-1) + 2 - 1$ & \\
& = & $2(n-1) + 1$ & \cy by algebra \\
\end{tabular}
\end{center}

Let $k = n-1$. 

Then $k$ is an integer because the difference of two integers ($n$ and $1$) is an integer. 

Hence $2n-1 = 2(n-1) + 1 = 2k + 1$ where $k$ is an integer, and thus by definition of odd $2n-1$ is odd {\it [as was to be shown]}.
\end{proof}

\subsection{Exercise 4}
The difference of any even integer minus any odd integer is odd.

\begin{proof}
Suppose $a$ is any even integer and $b$ is any odd integer. 
{\it [We must show that $a - b$ is odd.]} By definition of even and odd, $a = 2r$ and $b = 2s + 1$, for some integers $r, s$. By substitution and algebra,
$$
a-b = 2r - (2s+1) = 2r-2s-1 = 2r-2s-2+2-1 = 2(r-s-1)+1
$$


Let $t = r-s-1$. Then $t$ is an integer because differences of integers are integers. 

Thus $a-b = 2t+1$ where $t$ is an integer, and so by definition of odd $a-b$ is odd {\it [as was to be shown]}.
\end{proof}

\subsection{Exercise 5}
If $a$ and $b$ are any odd integers, then $a^2 + b^2$ is even.

\begin{proof}
Suppose $a,b$ are any {\it [particular but arbitrarily chosen]} odd integers. 

{\it [We must show that $a^2+b^2$ is even.]}

By definition of odd, $a = 2r+1$ and $b = 2s+1$, for some integers $r,s$. 

Then

\begin{center}
\begin{tabular}{rcll}
$a^2+b^2$ & = & $(2r+1)^2 + (2s+1)^2$ & \cy by substitution \\
& = & $(4r^2 + 4r + 1) + (4s^2 + 4s + 1)$ & \cy by multiplying \\
& = & $4r^2 + 4r + 4s^2 + 4s + 2$ & \cy by adding \\
& = & $2(2r^2+2r+2s^2+2s+1)$ & \cy by factoring out \\
\end{tabular}
\end{center}

Let $k = 2r^2+2r+2s^2+2s+1$. 

Then $k$ is an integer because squares, products and sums of integers are integers. 

Hence $a^2+b^2 = 2k$ where $k$ is an integer, and thus by definition of even $a^2+b^2$ is even {\it [as was to be shown]}.
\end{proof}

\subsection{Exercise 6}
If $k$ is any odd integer and $m$ is any even integer, then $k^2 + m^2$ is odd.

\begin{proof}
Suppose $k$ is any odd integer and $m$ is any even integer. 

{\it [We must show that $k^2 + m^2$ is odd.]}

By definition of odd and even, $k = 2a+1$ and $m = 2b$, for some integers $a, b$. Then

\begin{center}
\begin{tabular}{rcll}
$k^2+m^2$ & = & $(2a+1)^2+(2b)^2$ & \cy by substitution \\
& = & $4a^2+4a+1+4b^2$ & \\
& = & $4(a^2+a+b^2)+1$ & \\
& = & $2(2a^2+2a+2b^2)+1$ & \cy by algebra \\
\end{tabular}
\end{center}

But $2a^2+2a+2b^2$ is an integer because it is a sum of products of integers. Thus $k^2+m^2$ is twice an integer plus 1, and so $k^2+m^2$ is odd {\it [as was to be shown]}.
\end{proof}

\subsection{Exercise 7}
The difference between the squares of any two consecutive integers is odd.

\begin{proof}
Suppose $m$ and $n$ are any {\it [particular but arbitrarily chosen]} two consecutive integers. 

{\it [We must show that $m^2-n^2$ (or $n^2-m^2$) is odd.]}

By definition of consecutive, $m = k$ and $n = k+1$, for some integer $k$. 

Then

\begin{center}
\begin{tabular}{rcll}
$m^2-n^2$ & = & $k^2 - (k+1)^2$ & \cy by substitution \\
& = & $k^2 - (k^2 + 2k + 1)$ & \\
& = & $k^2 - k^2 - 2k - 1$ & \\
& = & $-2k-1$ & \\
& = & $-2k-2+2-1$ & \\
& = & $2(-k-1)+1$ & \cy by algebra \\
\end{tabular}
\end{center}

Let $r = -k-1$. Then $r$ is an integer because it is a difference of integers. 

Hence $m^2 - n^2 = 2r+1$ where $r$ is an integer, and thus by definition of odd $m^2-n^2$ is odd {\it [as was to be shown]}.
\end{proof}

\subsection{Exercise 8}
For any integers $m$ and $n$, if $m$ is even and $n$ is odd
then $5m + 3n$ is odd.

\begin{proof}
Suppose $m$ is any even integer and $n$ is any odd integer.  

{\it [We must show that $5m+3n$ is odd.]}

By definition of even and odd, $m = 2r$ and $n = 2s+1$, for some integers $r, s$. 

Then

\begin{center}
\begin{tabular}{rcll}
$5m+3n$ & = & $5(2r)+3(2s+1)$ & \cy by substitution \\
& = & $10r+6s+3$ & \\
& = & $10r+6s+2+1$ & \\
& = & $2(5r+3s+1)+1$ & \cy by algebra \\
\end{tabular}
\end{center}

Let $k = 5r+3s+1$. 

Then $k$ is an integer because it is a sum of products of integers. 

Hence $5m+3n = 2k+1$ where $k$ is an integer, and thus by definition of odd $5m+3n$ is odd {\it [as was to be shown]}.
\end{proof}

\subsection{Exercise 9}
If an integer greater than 4 is a perfect square, then the immediately preceding integer is not prime.

\begin{proof}
Suppose $n$ is any integer greater than 4 that is a perfect square. 

{\it [We must show that $n-1$ is not prime, in other words, $n-1$ is composite.]}

By definition of perfect square, $n = k^2$, for some integer $k$. 

Without loss of generality, we may assume $k > 0$, because $n = k^2 = (-k)^2$, and if $k$ is negative, we can replace it with $-k$ which is positive.

Since $n > 4$ we have $n - 4 > 0$. So $k^2 - 4 > 0$. So $(k-2)(k+2) > 0$. So either $k-2$ and $k+2$ are both negative, or they are both positive. Since $k > 0$, $k+2>2>0$, so they have to be both positive. Therefore, $k - 2 > 0$ so $k > 2$.

Then

\begin{center}
\begin{tabular}{rcll}
$n-1$ & = & $k^2-1$ & \cy by substitution \\
& = & $(k-1)(k+1)$ & \cy by algebra \\
\end{tabular}
\end{center}

Since $k>2$ we have both $k-1>1$ and $k+1>3>1$.

Hence $n-1$ is a product of two positive integers both greater than 1, and thus by definition of composite $n-1$ is composite {\it [as was to be shown]}.
\end{proof}

\subsection{Exercise 10}
If $n$ is any even integer, then $(-1)^n = 1$.

\begin{proof}
Suppose $n$ is any even integer. {\it [We must show that $(-1)^n = 1$.]}

By definition of even, $n = 2k$, for some integer $k$. 

Then by the laws of exponents from algebra $(-1)^n = (-1)^{2k} = ((-1)^2)^k = 1^k = 1$, {\it [as was to be shown]}.
\end{proof}

\subsection{Exercise 11}
If $n$ is any odd integer, then $(-1)^n = -1$.

\begin{proof}
Suppose $n$ is any odd integer. {\it [We must show that $(-1)^n = -1$.]}

By definition of even, $n = 2k+1$, for some integer $k$. 

Then by the laws of exponents from algebra 
$$
(-1)^n = (-1)^{2k+1} = (-1)^{2k}\cdot(-1) = ((-1)^2)^k\cdot(-1) = 1^k\cdot(-1) = 1\cdot(-1) = -1
$$ 
{\it [as was to be shown]}.
\end{proof}

{\bf \cy Prove that the statements in $12-14$ are false.}

\subsection{Exercise 12}
There exists an integer $m \geq 3$ such that $m^2 - 1$ is prime.

\begin{proof}
To prove the given statement is false, we prove that its
negation is true.

The negation of the statement is “For every integer $m \geq 3$, $m^2 - 1$ is not prime.”

{\it Proof of the negation:} Suppose $m$ is any integer with $m \geq 3$. 

By basic algebra, $m^2 - 1 = (m - 1)(m + 1)$. 

Because $m \geq 3$, both $m - 1$ and $m + 1$ are positive integers greater than 1, and each is smaller than $m^2 - 1$. 

So $m^2 - 1$ is a product of two smaller positive integers, each greater than 1, and hence $m^2 - 1$ is not prime.
\end{proof}

\subsection{Exercise 13}
There exists an integer $n$ such that $6n + 27$ is prime.

\begin{proof}
To prove the given statement is false, we prove that its
negation is true.

The negation of the statement is “For every integer $n$, $6n+27$ is not prime. In other words, $6n+27$ is composite.”

{\it Proof of the negation:} Suppose $n$ is any integer. 
By basic algebra, $6n + 27 = 3(2n + 9)$. 

Hence $6n+27$ is the product of two integers greater than 1. Therefore by definition of composite, $6n+27$ is composite.
\end{proof}

\subsection{Exercise 14}
There exists an integer $k \geq 4$ such that $2k^2 - 5k + 2$ is prime.

\begin{proof}
To prove the given statement is false, we prove that its
negation is true.

The negation of the statement is “For every integer $k \geq 4$, $2k^2 - 5k + 2$ is composite.”

{\it Proof of the negation:} Suppose $k$ is any integer. 

By basic algebra, $2k^2 - 5k + 2 = (2k-1)(k-2)$. 

Because $k \geq 4$, $2k - 1 \geq 7$ and $k - 2 \geq 2$. So both $2k-1$ and $k-2$ are integers greater than 1. 

Hence $2k^2 - 5k + 2$ is the product of two integers greater than 1. Therefore by definition of composite, $2k^2 - 5k + 2$ is composite.
\end{proof}

{\bf \cy Find the mistakes in the “proofs” shown in $15-19$.}

\subsection{Exercise 15}
{\bf Theorem:} For every integer $k$, if $k > 0$ then $k^2 + 2k + 1$ is composite.

“{\bf Proof:} For $k = 2$, $k > 0$ and $k^2 + 2k + 1 = 2^2 + 2\cdot2 + 1 = 9$. And since $9 = 3\cdot3$, then 9 is composite. Hence the theorem is true.”

\begin{proof}
The incorrect proof just shows the theorem to be true in
the one case where $k = 2$. A real proof must show that it
is true for {\it every} integer $k > 0$.
\end{proof}

\subsection{Exercise 16}
{\bf Theorem:} The difference between any odd integer and any even integer is odd.

“{\bf Proof:} Suppose $n$ is any odd integer, and $m$ is any even integer. By definition of odd, $n = 2k + 1$ where $k$ is an integer, and by definition of even, $m = 2k$ where $k$ is an integer. Then $n - m = (2k + 1) - 2k = 1$, and 1 is odd. Therefore, the difference between any odd integer and any even integer is odd.”

\begin{proof}
The mistake in the “proof” is that the same symbol, $k$, is used to represent two different quantities. By setting $m = 2k$ and $n = 2k + 1$, the proof implies that $n = m + 1$, and thus it deduces the conclusion only for this one situation. When $m = 4$ and $n = 17$, for instance, the computations in the proof indicate that $n - m = 1$, but actually $n - m = 13$. In other words, the proof does not deduce the conclusion for an arbitrarily chosen even integer $m$ and odd integer $n$, and hence it is invalid.
\end{proof}

\subsection{Exercise 17}
{\bf Theorem:} For every integer $k$, if $k > 0$ then $k^2 + 2k + 1$ is composite.

{\bf Proof:} Suppose $k$ is any integer such that $k > 0$. If $k^2 + 2k + 1$ is composite, then $k^2 + 2k + 1 = rs$
for some integers $r$ and $s$ such that 

$1 < r < k^2 + 2k + 1$ and $1 < s < k^2 + 2k + 1$.

Since $k^2 + 2k + 1 = rs$ and both $r$ and $s$ are strictly between 1 and $k^2 + 2k + 1$, then $k^2 + 2k + 1$ is not prime. So $k^2 + 2k + 1$ is composite as was to be shown.

\begin{proof}
This incorrect proof assumes what is to be proved. The word since in the third sentence is completely unjustified. The second sentence tells only what happens if $k^2 + 2k = 1$ is composite. But at that point in the proof, it has not been established that $k^2 + 2k + 1$ is composite. In fact, that is exactly what is to be proved.
\end{proof}

\subsection{Exercise 18}
{\bf Theorem:} The product of any even integer and any odd integer is even.

“{\bf Proof:} Suppose $m$ is any even integer and $n$ is any odd integer. If $m\cdot n$ is even, then by definition of even there exists an integer $r$ such that $m\cdot n = 2r$. 

Also since $m$ is even, there exists an integer $p$ such that $m = 2p$, and since $n$ is odd there exists an integer $q$ such that $n = 2q + 1$. 

Thus $mn = (2p)(2q + 1) = 2r$, where $r$ is an integer. By definition of even, then, $m\cdot n$ is even, as was to be shown.”

\begin{proof}
The issue is just like in Exercise 17. The proof uses the $r$ value without establishing the existence of $r$ first. 

``If $m \cdot n$ is even...'' has an unjustified assumption because we haven't proved that $m \cdot n$ is even yet (that's what we are {\it trying to prove}), so its conclusion ``...$m\cdot n = 2r$'' has not been proven.

Therefore the part ``$mn = (2p)(2q + 1) = 2r$, where $r$ is an integer'' is unjustified as well.

{\bf Discussion:}

This is a fairly common form of circular reasoning: assuming what we have to prove. It happens because, at the beginning of the proof, we want to mention to the reader what we want to prove. 

What we want to prove has a short, condensed definition (in this case ``being even''), so we write out the full definition of what it is that we are {\it trying to prove} (in this case ``the existence of an integer $r$ such that $\ldots = 2r$''). Again, the purpose of this is to articulate to the reader what we are {\it trying to prove}.

But then we forget that and continue as if that was already an established fact. The act of writing out the full definition of what we are trying to prove is not the same as actually having proved it.

Using ``if $m\cdot n$ is even...'' in this case is the problem; it has the {\it feeling} of using modus ponens on an already established implication with an established premise. But we are just writing out the full definition, instead of using modus ponens.

So it would be better to write: ``{\it We want to prove that $m\cdot n$ is even. In other words, we want to prove that there is an integer $r$ such that $m \cdot n = 2r$.}'' Using the words ``We want to prove that...'' instead of ``If...'' goes a long way to avoid this common mistake. This way we can ``unpack'' the definition of what we are trying to prove without assuming it.

Another related problem is to first unpack the definition of what we are trying to prove, then try to ``prove backwards''. Say we want to prove $A$, and we unpack the definition to B. So we have to prove $B$. But instead, we start by assuming $B$ is true, and apply some algebra or logic to it, to arrive at something else, say $E$, that is true:
$$
A \to \text{unpack definition} \to B \to \text{middle steps} \to C \to D \to E = \text{something true!}
$$
But this would only prove that $B$ implies $E$. In order to establish the truth of $B$ (and hence of $A$), we would actually have to prove that $E$ implies $B$! So all the ``steps'' from $B$ to $E$ would have to be ``reversible'', in other words, logical equivalences (biconditionals):
$$
A \bic \text{unpack definition} \bic B \bic \text{middle steps} \bic C \bic D \bic E = \text{something true!}
$$
But that is rarely the case!
\end{proof}

\subsection{Exercise 19}
{\bf Theorem:} The sum of any two even integers equals $4k$ for some integer $k$.

“{\bf Proof:} Suppose $m$ and $n$ are any two even integers. By definition of even, $m = 2k$ for some integer $k$ and $n = 2k$ for some integer $k$. By substitution, 
$$
m + n = 2k + 2k = 4k.
$$
This is what was to be shown.”

\begin{proof}
The problem here is the same as in Exercise 16. The mistake in the “proof” is that the same symbol, $k$, is used to represent two different quantities. By setting $m = 2k$ and $n = 2k$, the proof implies that $n = m$, and thus it deduces the conclusion only for this one situation. 

When $m = 4$ and $n = 20$, for instance, the proof indicates that $n = m = 4$, but actually $n = 20$. In other words, the proof does not deduce the conclusion for an arbitrarily chosen even integer $m$ and an arbitrarily chosen even integer $n$, and hence it is invalid.
\end{proof}

{\bf \cy In $20-38$ determine whether the statement is true or false. Justify your answer with a proof or a counterexample, as appropriate. In each case use only the definitions of the terms and the assumptions listed on page 161, not any previously established properties.}

\subsection{Exercise 20}
The product of any two odd integers is odd.

\begin{proof}
True. Suppose $m$ and $n$ are any odd integers. {\it [We must show that $mn$ is odd.]} By definition of odd, $n = 2r + 1$ and $m = 2s + 1$ for some integers $r$ and $s$.

Then

\begin{center}
\begin{tabular}{rcll}
$mn$ & = & $(2r+1)(2s+1)$ & \cy by substitution \\
& = & $4rs + 2r + 2s + 1$ & \\
& = & $2(2rs + r + s) + 1$ & \cy by algebra \\
\end{tabular}
\end{center}

Now $2rs + r + s$ is an integer because products and sums
of integers are integers and $2$, $r$, and $s$ are all integers. Hence $mn = 2\cdot \text{(some integer)} + 1$, and so, by definition of odd, $mn$ is odd.
\end{proof}

\subsection{Exercise 21}
The negative of any odd integer is odd.

\begin{proof}
True. Assume $n$ is any odd integer. {\it We want to prove $-n$ is odd.}

By definition of odd, $n = 2r + 1$ for some integer $r$. 

Then $-n = -(2r+1) = -2r-1 = -2r-2+2-1 = 2(-r-1) + 1$.

Let $k = -r-1$. Then $k$ is an integer because it is the difference of two integers.

Therefore $-n = 2k+1$ where $k$ is an integer, hence by definition of odd, $-n$ is odd.
\end{proof}

\subsection{Exercise 22}
For all integers $a$ and $b$, $4a + 5b + 3$ is even.

\begin{proof}
False. \underline{Counterexample:} Let $a = 1$ and $b = 0$. 

Then $4a + 5b + 3 = 4\cdot 1 + 5\cdot 0 + 3 = 7$, which is odd. 

{\it [This is one counterexample among many. Can you find a way to characterize all counterexamples?]}
\end{proof}

\subsection{Exercise 23}
The product of any even integer and any integer is even.

\begin{proof}
True. Suppose $m$ is any even integer and $n$ is any integer. {\it [We want to prove $m \cdot n$ is even.]}

By definition of even, $m = 2k$ for some integer $k$.

Then, $m \cdot n = (2k) \cdot n = 2kn = 2(kn)$.

Let $r = kn$. Then $r$ is an integer because it is the product of two integers.

Therefore $m\cdot n = 2r$ where $r$ is an integer. So by definition of even, $m \cdot n$ is even.
\end{proof}

\subsection{Exercise 24}
If a sum of two integers is even, then one of the summands is even. (In the expression $a + b$, $a$ and $b$ are called {\bf summands}.)

\begin{proof}
False. \underline{Counterexample:} Let $m = 1$ and $n = 3$. 

Then $m + n = 4$ is even, but neither summand $m$ nor summand $n$ is even.
\end{proof}

\subsection{Exercise 25}
The difference of any two even integers is even.

\begin{proof}
True. Assume $m$ and $n$ are any two even integers. {\it [We want to prove $m-n$ is even.]}

By definition of even, $m = 2r$ and $n = 2s$ for some integers $r, s$.

Then $m - n = 2r - 2s = 2(r-s)$. Let $k = r-s$. Then $k$ is an integer because it is the difference of two integers.

Therefore $m - n = 2k$ where $k$ is an integer. So $m-n$ is even by definition of even.
\end{proof}

\subsection{Exercise 26}
For all integers $a, b$, and $c$, if $a, b$, and $c$ are consecutive, then $a + b + c$ is even.

\begin{proof}
False. \underline{Counterexample:} Let $a = 2, b = 3, c = 4$. They are consecutive integers but $a+b+c = 9$ which is not even.
\end{proof}

\subsection{Exercise 27}
The difference of any two odd integers is even.

\begin{proof}
True. Assume $m, n$ are any two odd integers. {\it [We want to prove $m - n$ is even.]}

By definition of odd, $m = 2r+1, n = 2s+1$ for some integers $r,s$.

Then $m-n = 2r+1 - (2s+1) = 2r+1 - 2s - 1 = 2r - 2s = 2(r-s)$.

Let $k = r-s$. Then $k$ is an integer because it is the difference of two integers.

So $m-n = 2k$ where $k$ is an integer. Hence $m-n$ is even by definition of even.
\end{proof}

\subsection{Exercise 28}
For all integers $n$ and $m$, if $n - m$ is even then $n^3 - m^3$ is even.

\begin{proof}
True. Assume $n, m$ are any integers such that $n - m$ is even. {\it [Want to prove that $n^3 - m^3$ is even.]}

By definition of even, $n-m = 2r$ for some integer $r$. By algebra, 

$n^3 - m^3 = (n-m)(n^2 + nm + m^2) = 2r(n^2 + nm + m^2) = 2(r(n^2 + nm + m^2))$.

Let $k = r(n^2 + nm + m^2)$. Then $r$ is an integer because it is a sum and product of integers.

So $n^3 - m^3 = 2k$ where $k$ is an integer. So by definition of even, $n^3 - m^3$ is even.
\end{proof}

\subsection{Exercise 29}
For every integer $n$, if $n$ is prime then $(-1)^n = -1$.

\begin{proof}
False. \underline{Counterexample:} Let $n=2$. 

Then $n$ is prime, but $(-1)^n = (-1)^2 = 1 \neq -1$.
\end{proof}

\subsection{Exercise 30}
For every integer $m$, if $m > 2$ then $m^2 - 4$ is composite.

\begin{proof}
False. \underline{Counterexample:} Let $m = 3$. Then $m^2 - 4 = 3^2 - 4 = 9 - 4 = 5$ is prime. not composite.
\end{proof}

\subsection{Exercise 31}
For every integer $n$, $n^2 - n + 11$ is a prime number.

\begin{proof}
False. Let $n = 11$. Then $n^2 - n + 11 = 11^2 - 11 + 11 = 11^2$ is not prime.
\end{proof}

\subsection{Exercise 32}
For every integer $n$, $4(n^2 + n + 1) - 3n^2$ is a perfect square.

\begin{proof}
True. Suppose $n$ is any integer. Then by algebra
$$
4(n^2 + n + 1) - 3n^2 = 4n^2 + 4n + 4 - 3n^2 = n^2 + 4n + 4 = (n + 2)^2
$$
Now $(n + 2)^2$ is a perfect square because $n + 2$ is an integer (being a sum of $n$ and $2$). Hence $4(n^2 + n + 1) - 3n^2$ is a perfect square, as was to be shown.
\end{proof}

\subsection{Exercise 33}
Every positive integer can be expressed as a sum of three or fewer perfect squares.

\begin{proof}
False. \underline{Counterexample:} 7 cannot be written as a sum of three of fewer perfect squares: $7 = 2^2 + 1^2 + 1^2 + 1^2$.
\end{proof}

\subsection{Exercise 34}
(Two integers are {\bf consecutive} if, and only if, one is one more than the other.) Any product of four consecutive integers is one less than a perfect square.

\begin{proof}
True. Suppose $a, b, c, d$ are any four consecutive integers. {\it [Want to prove: there is an integer $k$ such that $abcd = k^2 - 1$.]}

By definition of consecutive, there is an integer $n$ such that $a = n, b = n+1, c = n+2, d = n+3$. Then
$$
abcd = n(n+1)(n+2)(n+3) = n(n+3)(n+1)(n+2) = (n^2+3n)(n^2+3n+2)
$$
{\it [Here we notice a pattern. The two factors differ by 2. So it is reminiscent of $(x-1)(x+1) = x^2 - 1^2$ isn't it?]}

By some more algebra,
$$
(n^2+3n)(n^2+3n+2) = (n^2+3n+1-1)(n^2+3n+1+1) = (n^2+3n+1)^2-1^2
$$
Let $k = n^2+3n+1$. Then $k$ is an integer because it is a sum and product of integers. Therefore $abcd = k^2 - 1$ where $k$ is an integer, {\it [as was to be shown].}
\end{proof}

\subsection{Exercise 35}
If $m$ and $n$ are any positive integers and $mn$ is a perfect square, then $m$ and $n$ are perfect squares. 

\begin{proof}
False. \underline{Counterexample:} let $m = n = 2$. Then $mn = 2^2$ is a perfect square. But neither $m$ nor $n$ is a perfect square.
\end{proof}

\subsection{Exercise 36}
The difference of the squares of any two consecutive integers is odd.

\begin{proof}
True. Assume $a,b$ are any two consecutive integers. 

By definition of consecutive, $a = n$ and $b = n+1$ for some integer $n$.

Then $b^2 - a^2 = (n+1)^2 - n^2 = n^2+2n+1-n^2 = 2n+1$.

So $b^2 - a^2 = 2k+1$ where $n$ is an integer. Therefore by definition of odd, $b^2-a^2$ is odd.
\end{proof}

\subsection{Exercise 37}
For all nonnegative real numbers $a$ and $b$, $\sqrt{ab} = \sqrt{a}\sqrt{b}$. (Note that if $x$ is a nonnegative real number, then there is a unique nonnegative real number $y$, denoted $\sqrt{x}$, such that $y^2 = x$.)

\begin{proof}
True. Assume $a$ and $b$ are any two nonnegative real numbers. By the information given to us in the parentheses: 

1. There is a unique nonnegative real number denoted $\sqrt{ab}$ such that $(\sqrt{ab})^2 = ab$.

2. There is a unique nonnegative real number denoted $\sqrt{a}$ such that $(\sqrt{a})^2 = a$.

3. There is a unique nonnegative real number denoted $\sqrt{b}$ such that $(\sqrt{b})^2 = b$.

Since $ab = a \cdot b$, we have by substitution: $(\sqrt{ab})^2 = (\sqrt{a})^2 \cdot (\sqrt{b})^2$.

By algebra, $(\sqrt{ab})^2 = [(\sqrt{a}) \cdot (\sqrt{b})]^2 = (\sqrt{a}\sqrt{b})^2$. Therefore $(\sqrt{ab})^2 - (\sqrt{a}\sqrt{b})^2 = 0$.

By factoring we get $(\sqrt{ab} - \sqrt{a}\sqrt{b})(\sqrt{ab} + \sqrt{a}\sqrt{b}) = 0$.

So: either $\sqrt{ab} - \sqrt{a}\sqrt{b} = 0$, or $\sqrt{ab} + \sqrt{a}\sqrt{b} = 0$ (by the Zero Product Property T11).

If $\sqrt{ab} - \sqrt{a}\sqrt{b} = 0$, then $\sqrt{ab} = \sqrt{a}\sqrt{b}$ {\it [as was to be shown.]}

If $\sqrt{ab} + \sqrt{a}\sqrt{b} = 0$, then since both $\sqrt{ab}$ and $\sqrt{a}\sqrt{b}$ are nonnegative, they must be both 0, hence $\sqrt{ab} = \sqrt{a}\sqrt{b}$ again {\it [as was to be shown.]}
\end{proof}

\subsection{Exercise 38}
For all nonnegative real numbers $a$ and $b$, $\sqrt{a + b} = \sqrt{a} + \sqrt{b}$.

\begin{proof}
False. \underline{Counterexample:} Let $a = b = 1$. Then 
$$
\sqrt{a+b} = \sqrt{1+1} = \sqrt{2} \neq 2 = 1 + 1 = \sqrt{1} + \sqrt{1} = \sqrt{a} + \sqrt{b}
$$
\end{proof}

\subsection{Exercise 39}
Suppose that integers $m$ and $n$ are perfect squares. Then $m + n + 2\sqrt{mn}$ is also a perfect square. Why?

\begin{proof}
Assume $m$ and $n$ are perfect squares (so they are nonnegative real numbers). By definition of perfect square, $m = r^2$ and $n = s^2$ for some integers $r, s$. Using Exercise 37 $\sqrt{mn} = \sqrt{m}\sqrt{n}$, we get:

$m + n + 2\sqrt{mn} = r^2 + s^2 + 2\sqrt{m}\sqrt{n} = r^2 + s^2 + 2rs = (r+s)^2$.

Let $k = r+s$. $k$ is an integer because it is a sum of integers. So $m + n + 2\sqrt{mn} = k^2$ where $k$ is an integer, therefore $m + n + 2\sqrt{mn}$ is a perfect square by definition.

\end{proof}

\subsection{Exercise 40}
If $p$ is a prime number, must $2^p - 1$ also be prime? Prove or give a counterexample.

\begin{proof}
No. \underline{Counterexample:} $p = 11$ is prime, but $2^p - 1 = 2^{11} - 1 = 2047 = 13 \cdot 89$ is not prime.
\end{proof}

\subsection{Exercise 41}
If $n$ is a nonnegative integer, must $2^{2n} + 1$ be prime? Prove or give a counterexample.

\begin{proof}
No. \underline{Counterexample:} Let $n = 3$. Then $2^{2n} + 1 = 2^{6} + 1 = 65 = 13 \cdot 5$ is not prime.
\end{proof}

\section{Exercise Set 4.3}

{\bf \cy The numbers in $1-7$ are all rational. Write each number as a ratio of two integers.}

\subsection{Exercise 1}
$\dps-\frac{35}{6}$

\begin{proof}
$\frac{-35}{6} = \frac{-35}{6}$
\end{proof}

\subsection{Exercise 2}
$4.6037$

\begin{proof}
$4.6037 = \frac{46037}{10000}$
\end{proof}

\subsection{Exercise 3}
$\dps\frac{4}{5} + \frac{2}{9}$

\begin{proof}
$\dps\frac{4}{5} + \frac{2}{9} = \frac{4 \cdot 9 + 5 \cdot 2}{5 \cdot 9} = \frac{46}{45}$
\end{proof}

\subsection{Exercise 4}
$0.37373737\ldots$

\begin{proof}
Let $x = 0.373737\ldots$

Then $100x = 37.373737\ldots$, so $100x - x = 37.373737\ldots - 0.373737\ldots = 37$. 

Thus $99x = 37$, and hence $x = \frac{37}{99}$.
\end{proof}

\subsection{Exercise 5}
$0.56565656\ldots$

\begin{proof}
Let $x = 0.565656\ldots$ 

Then $100x = 56.565656\ldots$ and so $100x - x = 56.565656\ldots - 0.565656\ldots = 56$. 

Thus $99x = 56$, and hence $x = \frac{56}{99}$.
\end{proof}

\subsection{Exercise 6}
$320.5492492492\ldots$

\begin{proof}
Let $x = 320.5492492492\ldots$. 

Then $10000x = 3205492.492492\ldots$ and $10x = 3205.492492\ldots$, so 

$10000x - 10x = 3205492.492492 \ldots - 3205.492492\ldots = 3205492 - 3205 = 3202287$. 

Thus $9990x = 3202287$, and hence $x = \frac{3202287}{9990}$.
\end{proof}

\subsection{Exercise 7}
$52.4672167216721\ldots$

\begin{proof}
Let $x = 52.467216721\ldots$

Then $100000x = 5246721.67216721\ldots$ and $10x = 524.67216721\ldots$, so 

$100000x - 10x = 5246721.67216721 \ldots - 524.67216721 \ldots = 5246721 - 524 = 5246197$. 

Thus $99990x = 5246197$, and hence $x = \frac{5246197}{99990}$.
\end{proof}

\subsection{Exercise 8}
The zero product property, says that if a product of two real numbers is 0, then one of the numbers must be 0.

\subsubsection{(a)}
Write this property formally using quantifiers and variables.

\begin{proof}
$\fa$ real numbers $x, y$, if $xy = 0$ then $x = 0$ or $y = 0$.
\end{proof}

\subsubsection{(b)}
Write the contrapositive of your answer to part (a).

\begin{proof}
$\fa$ real numbers $x, y$, if $x \neq 0$ and $y \neq 0$ then $xy \neq 0$.
\end{proof}

\subsubsection{(c)}
Write an informal version (without quantifier symbols or variables) for your answer to part (b).

\begin{proof}
The product of two nonzero real numbers is nonzero.
\end{proof}

\subsection{Exercise 9}
Assume that $a$ and $b$ are both integers and that $a \neq 0$ and $b \neq 0$. Explain why $(b - a)/(ab^2)$ must be a rational number.

\begin{proof}
Given that $a$ and $b$ are integers, both $b - a$ and $ab^2$ are integers (since differences and products of integers are integers). Also, by the zero product property, $ab^2 \neq 0$ because neither $a$ nor $b$ is zero. Hence $(b - a)/(ab^2)$ is a quotient of two integers with a nonzero denominator, and so it is rational.
\end{proof}

\subsection{Exercise 10}
Assume that $m$ and $n$ are both integers and that $n \neq 0$. Explain why $(5m - 12n)/(4n)$ must be a rational number.

\begin{proof}
Given that $m$ and $n$ are integers, both $5m - 12n$ and $4n$ are integers (since differences and products of integers are integers). Also, by the zero product property, $4n \neq 0$ because neither $4$ nor $n$ is zero. Hence $(5m - 12n)/(4n)$ is a quotient of two integers with a nonzero denominator, and so it is rational.
\end{proof}

\subsection{Exercise 11}
Prove that every integer is a rational number.

\begin{proof}
Suppose $n$ is any {\it[particular but arbitrarily chosen]}
integer. Then $n = n\cdot 1$, and so $n = n/1$ by dividing both sides by 1. Now $n$ and $1$ are both integers, and $1 \neq 0$. Hence $n$ can be written as a quotient of integers with a nonzero denominator, and so $n$ is rational.
\end{proof}

\subsection{Exercise 12}
Let $S$ be the statement “The square of any rational number is rational.” A formal version of $S$ is “For every rational number $r$, $r^2$ is rational.” Fill in the blanks in the proof for $S$.

{\bf Proof:} Suppose that $r$ is (a) \fbl. By definition of rational, $r = a/b$ for some (b) \fbl with $b \neq 0$. By substitution,
$$
r^2 = (c) \fbl = a^2/b^2.
$$
Since $a$ and $b$ are both integers, so are the products $a^2$ and (d) \fbl. Also $b^2 \neq 0$ by the (e) \fbl. Hence $r^2$ is a ratio of two integers with a nonzero denominator, and so (f) \fbl by definition of rational.

\begin{proof}
(a) any {\it [particular but arbitrarily chosen]} rational number

(b) integers $a$ and $b$

(c) $(a/b)^2$

(d) $b^2$

(e) zero product property

(f) $r^2$ is rational
\end{proof}

\subsection{Exercise 13}
Consider the following statement: The negative of any rational number is rational.

\subsubsection{(a)}
Write the statement formally using a quantifier and a variable.

\begin{proof}
$\fa$ real number $r$, if $r$ is rational then $-r$ is rational.

Or: $\fa r$, if $r$ is a rational number then $-r$ is rational.

Or: $\fa$ rational number $r$, $-r$ is rational.
\end{proof}

\subsubsection{(b)}
Determine whether the statement is true or false and justify your answer.

\begin{proof}
The statement is true. Suppose $r$ is a {\it [particular but arbitrarily chosen]} rational number. {\it [We must show that $-r$ is rational.]} By definition of rational, $r = a/b$ for some integers $a$ and $b$ with $b \neq 0$. Then by substitution and algebra,
$$
-r = -\frac{a}{b} = \frac{-a}{b}
$$
Now since $a$ is an integer, so is $-a$ (being the product of $-1$ and $a$). Hence $-r$ is a quotient of integers with a nonzero denominator, and so $-r$ is rational {\it [as was to be shown]}.
\end{proof}

\subsection{Exercise 14}
Consider the statement: The cube of any rational number is a rational number.

\subsubsection{(a)}
Write the statement formally using a quantifier and a variable.

\begin{proof}
$\fa$ rational $r$, $r^3$ is rational.
\end{proof}

\subsubsection{(b)}
Determine whether the statement is true or false and justify your answer.

\begin{proof}
The statement is true. Suppose $r$ is a {\it [particular but arbitrarily chosen]} rational number. {\it [We must show that $r^3$ is rational.]} By definition of rational, $r = a/b$ for some integers $a$ and $b$ with $b \neq 0$. Then by substitution and algebra,
$$
r^3 = \left(\frac{a}{b}\right)^3 = \frac{a^3}{b^3}
$$
Now since $a, b$ are integers, so are $a^3$ and $b^3$ (being the products of $a$ and $b$). Moreover, since $b \neq 0$, by the Zero Product Property, $b^3 \neq 0$. 

Hence $r^3$ is a quotient of integers with a nonzero denominator, and so $r^3$ is rational {\it [as was to be shown]}.
\end{proof}

{\bf \cy Determine which of the statements in $15-19$ are true and which are false. prove each true statement directly from the definitions, and give a counterexample for each false statement. For a statement that is false, determine whether a small change would make it true. If so, make the change and prove the new statement. Follow the directions for writing proofs on page 173.}

\subsection{Exercise 15}
The product of any two rational numbers is a rational number.

\begin{proof}
Suppose $r$ and $s$ are rational numbers. By definition of rational, $r = a/b$ and $s = c/d$ for some integers $a, b, c$, and $d$ with $b \neq 0$ and $d \neq 0$. Then by substitution and algebra,
$$
rs = \frac{a}{b} \cdot \frac{c}{d} = \frac{ac}{bd}
$$
Now $ac$ and $bd$ are both integers (being products of integers) and $bd \neq 0$ (by the zero product property). Hence $rs$ is a quotient of integers with a nonzero denominator, and so, by definition of rational, $rs$ is rational.
\end{proof}

\subsection{Exercise 16}
The quotient of any two rational numbers is a rational number.

\begin{proof}
\underline{Counterexample:} Let $r$ be any rational number and $s = 0$. Then $r$ and $s$ are both rational, but the quotient of $r$ divided by $s$ is not a real number and therefore is not a rational number.

{\it Revised statement to be proved:} For all rational numbers $r$ and $s$, if $s \neq 0$ then $r/s$ is rational.

Suppose $r,s$ are rational numbers such that $s \neq 0$. {\it[Want to prove $r/s$ is rational.]}

By definition of rational, $r = a/b$ and $s = c/d$ for some integers $a,b,c,d$ where $b \neq 0, d \neq 0$. Since $s \neq 0$ we also have $c \neq 0$. Then by algebra
$$
\frac{r}{s} = \frac{a/b}{c/d} = \frac{ad}{bc}
$$
Now $ad$ and $bc$ are integers because they are products of integers. Since $b \neq 0$ and $c \neq 0$, by Zero Product Property $bc \neq 0$. 

Let $m = ad$ and $n = bc$. So $m$ and $n$ are integers with $n \neq 0$, and $r / s = m / n$. Therefore by definition of rational, $r/s$ is rational.
\end{proof}

\subsection{Exercise 17}
The difference of any two rational numbers is a rational number.

\begin{proof}
True. Suppose $r,s$ are rational numbers. {\it[Want to prove $r-s$ is rational.]}

By definition of rational, $r = a/b$ and $s = c/d$ for some integers $a,b,c,d$ where $b \neq 0, d \neq 0$. Then by algebra
$$
r-s = \frac{a}{b}-\frac{c}{d} = \frac{ad-bc}{bd}
$$
Now $ad-bc$ and $bd$ are integers because they are products and differences of integers. Since $b \neq 0$ and $d \neq 0$, by Zero Product Property $bd \neq 0$. 

Let $m = ad-bc$ and $n = bd$. So $m$ and $n$ are integers with $n \neq 0$, and $r - s = m / n$. Therefore by definition of rational, $r-s$ is rational.
\end{proof}

\subsection{Exercise 18}
If $r$ and $s$ are any two rational numbers, then $\frac{r+s}{2}$ is rational.

\begin{proof}
True. The proof is very similar to Exercises 16 and 17. The crucial steps are
$$
\frac{r+s}{2} = \frac{\frac{a}{b} + \frac{c}{d}}{2} = \frac{(ad+bc)/bd}{2} = \frac{ad+bc}{2bd}
$$
and noticing that $ad+bc$ and $2bd$ are integers, and $2bd \neq 0$.
\end{proof}

\subsection{Exercise 19}
For all real numbers $a$ and $b$, if $a < b$ then $a < \frac{a+b}{2} < b$. (You may use the properties of inequalities in T17-T27 of Appendix A.)

\begin{proof}
True. Suppose $a,b$ are any two real numbers such that $a<b$. {\it[We need to prove two inequalities: $a < \frac{a+b}{2}$ and $\frac{a+b}{2} < b$.]}

Since $a<b$ we have $a+a < a+b$ by T19. So $2a < a+b$. Then $a < \frac{a+b}{2}$ by T20. This proves the first inequality.

Since $a<b$ we have $a+b < b+b$ by T19. So $a+b < 2b$. Then $\frac{a+b}{2} < b$ by T20. This proves the second inequality.
\end{proof}

\subsection{Exercise 20}
Use the results of exercises 18 and 19 to prove that given any two rational numbers $r$ and $s$ with $r < s$, there is another rational number between $r$ and $s$. An important consequence is that there are infinitely many rational numbers in between any two distinct rational numbers. See Section 7.4.

\begin{proof}
Assume $r$ and $s$ are any two rational numbers with $r < s$. {\it [Want to prove: $r < t < s$ for some rational number $t$.]}

Let $t = \frac{a+b}{2}$. By Exercise 18, $t$ is a rational number. By Exercise 19, $r < t < s$, {\it [as was to be shown.]}
\end{proof}

{\bf \cy Use the properties of even and odd integers that are listed in Example 4.3.3 to do exercises $21-23$. Indicate which properties you use to justify your reasoning.}

\subsection{Exercise 21}
True or false? If $m$ is any even integer and $n$ is any
odd integer, then $m^2 + 3n$ is odd. Explain.

\begin{proof}
True. 

$m$ is even. An even integer times an even integer is even, therefore $m^2$ is even. 

$3$ and $n$ are both odd. An odd integer times an odd integer is odd, therefore $3n$ is odd. 

$m^2$ is even. $3n$ is odd. An even integer plus an odd integer is odd, therefore $m^2 + 3n$ is odd.
\end{proof}

\subsection{Exercise 22}
True or false? If $a$ is any odd integer, then $a^2 + a$ is
even. Explain.

\begin{proof}
True. $a^2 + a = a(a+1)$. Since $a$ is odd, $a+1$ is even. Odd times even is even, therefore $a(a+1)$ is even. So $a^2+a$ is even.
\end{proof}

\subsection{Exercise 23}
True or false? If $k$ is any even integer and $m$ is any
odd integer, then $(k + 2)^2 - (m - 1)^2$ is even. Explain.

\begin{proof}
True. 

$k$ is even, so $k+2$ is even. Even squared is even, so $(k + 2)^2$ is even.

$m$ is odd, so $m-1$ is even. Even squared is even, so $(m - 1)^2$ is even.

Even minus even is even, so $(k + 2)^2 - (m - 1)^2$ is even.

{\it Another solution.} By algebra:

$(k + 2)^2 - (m - 1)^2 = (k+2-(m-1))(k+2+m-1) = (k-m+3)(k+m+1)$

Now $k-m+3$ is even $-$ odd $+$ odd = even. Even times anything is even, therefore $(k-m+3)(k+m+1)$ is even. So $(k + 2)^2 - (m - 1)^2$ is even.
\end{proof}

{\bf \cy Derive the statements in $24-26$ as corollaries of theorems 4.3.1, 4.3.2, and the results of exercises 12, 13, 14, 15, and 17.}

\subsection{Exercise 24}
For any rational numbers $r$ and $s$, $2r + 3s$ is rational.

\begin{proof}
Suppose $r$ and $s$ are any rational numbers. By Theorem 4.3.1, both 2 and 3 are rational, and so, by Exercise 15, both $2r$ and $3s$ are rational. Hence, by Theorem 4.3.2, $2r + 3s$ is rational. 
\end{proof}

\subsection{Exercise 25}
If $r$ is any rational number, then $3r^2 - 2r + 4$ is
rational.

\begin{proof}
Suppose $r$ is any rational number. By Exercise 12, $r^2$ is rational. By Theorem 4.3.1, $2, 3,4$ are all rational. By Exercise 15, $3r^2$ and $2r$ are rational. By Exercise 17, $3r^2-2r$ is rational. So by Theorem 4.3.2, $3r^2-2r+4$ is rational.
\end{proof}

\subsection{Exercise 26}
For any rational number $s$, $5s^3 + 8s^2 - 7$ is rational.

\begin{proof}
Assume $s$ is any rational number. By Theorem 4.3.1, $5, 8, 7$ are rational, and by Exercise 13, $-7$ is rational. By Exercise 14, $s^3$ is rational. By Exercise 12, $s^2$ is rational. By Exercise 15, $5s^3$ and $8s^2$ are rational. Therefore by Theorem 4.3.2, $5s^3 + 8s^2 - 7$ is rational. 
\end{proof}

\subsection{Exercise 27}
It is a fact that if $n$ is any nonnegative integer, then
$$
1 + \frac{1}{2} + \frac{1}{2^2} + \frac{1}{2^3} + \cdots + \frac{1}{2^n} = \frac{1-(1/2^{n+1})}{1-(1/2)}
$$
(A more general form of this statement is proved in Section 5.2.) Is the right-hand side of this equation rational? If so, express it as a ratio of two integers.

\begin{proof}
$$
x = \frac{1-\dps\frac{1}{2^{n+1}}}{1-\dps\frac{1}{2}} =
\frac{\dps\frac{2^{n+1}-1}{2^{n+1}}}{\dps\frac{1}{2}} =
\dps\frac{2^{n+1}-1}{2^{n+1}}\cdot{\dps\frac{2}{1}} = 
\dps\frac{2^{n+1}-1}{2^{n}}
$$
Now $2^{n+1} - 1$ and $2^n$ are both integers (since $n$ is a nonnegative integer) and $2^n \neq 0$ by the zero product
property. Therefore, $x$ is rational.
\end{proof}

\subsection{Exercise 28}
Suppose $a, b, c$, and $d$ are integers and $a \neq c$. Suppose also that $x$ is a real number that satisfies the
equation
$$
\frac{ax+b}{cx+d} = 1.
$$
Must $x$ be rational? If so, express $x$ as a ratio of two integers.

\begin{proof}
$$
\frac{ax+b}{cx+d} = 1 \to ax+b = cx+d \to ax-cx+b-d = 0 \to x(a-c) = d-b \to x = \frac{d-b}{a-c}
$$
$x$ is rational because both $d-b$ and $a-c$ are rational, and because $a-c \neq 0$ (since $a \neq c$).
\end{proof}

\subsection{Exercise 29}
Suppose $a, b$, and $c$ are integers and $x, y$, and $z$ are nonzero real numbers that satisfy the following equations:
$$
\frac{xy}{x+y} = a, \frac{xz}{x+z} = b, \frac{yz}{y+z} = c
$$
Is $x$ rational? If so, express it as ratio of two integers.

\begin{proof}
Taking the reciprocals of both sides of the first equation:
$$
\frac{x+y}{xy} = \frac{1}{a}
$$
Now we split the first fraction into two, and simplify:
$$
\frac{x+y}{xy} = \frac{x}{xy} + \frac{y}{xy} = \frac{1}{y} + \frac{1}{x}
$$
Therefore $\dps\frac{1}{y} + \frac{1}{x} = \frac{1}{a}$. By performing the same steps on the other two equations, we see that $\dps\frac{1}{z} + \frac{1}{x} = \frac{1}{b}$ and $\dps\frac{1}{z} + \frac{1}{y} = \frac{1}{c}$.

For the sake of simplicity let's do some renaming: let 
$$
X = 1/x, \,\, Y = 1/y, \,\, Z = 1/z, \,\, A = 1/a, \,\, B = 1/b, \,\, C = 1/c
$$
So the three new equations we derived above become:
$$
\begin{array}{ccccccc}
Y&+&X&=&A&&(1)\\
Z&+&X&=&B&&(2)\\
Z&+&Y&=&C&&(3)\\
\end{array}
$$
From (1) we get (4): $Y = A - X$ and from (2) we get (5): $Z = B - X$.

Substituting (4) and (5) back into (3) we get: $(B-X) + (A-X) = C$.

So $B+A-2X = C$, then $B+A-C = 2X$ and $\frac{1}{2}(B+A-C) = X$. Using our old variable names, we get:
$$
\frac{1}{2}\left(\frac{1}{b}+\frac{1}{a}-\frac{1}{c}\right) = \frac{1}{x}
$$
Rewriting:
$$
\frac{1}{2b}+\frac{1}{2a}-\frac{1}{2c} = \frac{1}{x}
$$
Getting a common denominator, then adding:
$$
\frac{ac}{2abc}+\frac{bc}{2abc}-\frac{ab}{2abc} = \frac{ac + bc - ac}{2abc} = \frac{1}{x}
$$
Finally, taking reciprocals of both sides, we get $x$ as a ratio of two integers:
$$
\frac{2abc}{ac + bc - ac} = x
$$
\end{proof}

\subsection{Exercise 30}
Prove that if one solution for a quadratic equation of the form $x^2 + bx + c = 0$ is rational (where $b$ and $c$ are rational), then the other solution is also rational. (Use the fact that if the solutions of the equation are $r$ and $s$, then $x^2 + bx + c = (x - r)(x - s)$.)

\begin{proof}
Assume $x^2 + bx + c = 0$ has two solutions $r,s$ where one of them, $r$, is rational. {\it [Want to prove: $s$ is also rational.]}

We are given the fact that $x^2 + bx + c = (x - r)(x - s)$. This holds true for all real numbers $x$. Solving for $s$, we get
$$
\frac{x^2 + bx + c}{x-r} = x - s \implies \frac{x^2 + bx + c}{x-r} -x = - s \implies -\frac{x^2 + bx + c}{x-r} + x = s
$$
This equality is true for all real $x$ except $x = r$ (because then division by $x-r$ would be illegal). So, let's substitute an $x$ value that is rational and different than $r$, say $r+1$. Then we get:
$$
-\frac{(r+1)^2 + b(r+1) + c}{r+1-r} + r+1 = s \implies -(r+1)^2 + b(r+1) + c + r+1 = s
$$
Now we use the facts established in the Exercises that sums, negatives, products and squares of rational numbers are rational. Since $r, b, c, 1$ are all rational, this implies that $s$ is rational.
\end{proof}

\subsection{Exercise 31}
Prove that if a real number $c$ satisfies a polynomial equation of the form
$$
r_3 x^3 + r_2 x^2 + r_1 x + r_0 = 0
$$
where $r_0, r_1, r_2, r_3$ are rational numbers, then $c$ satisfies an equation of the form
$$
n_3 x^3 + n_2 x^2 + n_1 x + n_0 = 0
$$
where $n_0, n_1, n_2, n_3$ are integers.

\begin{proof}
Suppose $c$ is a real number such that $r_3 c^3 + r_2 c^2 + r_1 c + r_0 = 0$, where $r_0, r_1, r_2, r_3$ are rational numbers.

By definition of rational, $r_0 = a_0/b_0, r_1 = a_1/b_1, r_2 = a_2/b_2, r_3 = a_3/b_3$ for some integers $a_0, a_1, a_2, a_3$ and some nonzero integers $b_0, b_1, b_2, b_3$. By substitution,
$$
\begin{array}{rcl}
r_3 c^3 + r_2 c^2 + r_1 c + r_0 & = & \dps\frac{a_3}{b_3}c^3 + \frac{a_2}{b_2}c^2 + \frac{a_1}{b_1}c + \frac{a_0}{b_0} \vspace{0.3cm}\\
& = & \dps\frac{b_0b_1b_2a_3}{b_0b_1b_2b_3}c^3 + \frac{b_0b_1b_3a_2}{b_0b_1b_2b_3}c^2 + \frac{b_0b_2b_3a_1}{b_0b_1b_2b_3}c + \frac{b_1b_2b_3a_0}{b_0b_1b_2b_3} \\
& = & 0.
\end{array}
$$
Multiplying both sides by $b_0b_1b_2b_3$ gives
$$
b_0b_1b_2a_3 \cdot c^3 + b_0b_1b_3a_2 \cdot c^2 + b_0b_2b_3a_1 \cdot c + b_1b_2b_3a_0 = 0
$$
Let $n_3 = b_0b_1b_2a_3, n_2 = b_0b_1b_3a_2, n_1 = b_0b_2b_3a_1, n_0 = b_1b_2b_3a_0$. Then $n_3, n_2, n_1, n_0$ are all integers (being products of integers). Hence $c$ satisfies the equation
$$
n_3 \cdot c^3 + n_2 \cdot c^2 + n_1 \cdot c + n_0 = 0
$$
where $n_3, n_2, n_1, n_0$ are all integers, {\it [as was to be shown.]}
\end{proof}

\begin{tcolorbox}[colframe=cyan]
{\bf \cy Definition:} A number $c$ is called a {\bf root} of a polynomial $p(x)$ if, and only if, $p(c) = 0$.
\end{tcolorbox}

\subsection{Exercise 32}
Prove that for every real number $c$, if $c$ is a root of a polynomial with rational coefficients, then $c$ is a root of a polynomial with integer coefficients.

\begin{proof}
The proof is extremely similar to Exercise 31. Assume $p(c) = 0$ where: 

$c$ is any real number, and $p(x) = \dps\frac{a_n}{b_n}x^n + \cdots + \frac{a_1}{b_1}x + \frac{a_0}{b_0}$ is a polynomial with rational coefficients (so $a_0, \ldots, a_n$ are all integers and $b_0, \ldots, b_n$ are all nonzero integers).

So $c$ satisfies the equation
$$
\frac{a_n}{b_n}c^n + \cdots + \frac{a_1}{b_1}c + \frac{a_0}{b_0} = 0
$$
Multiply both sides by $L = b_0b_1\cdots b_{n-1}b_n$:
$$
\frac{a_n L}{b_n}c^n + \cdots + \frac{a_1 L}{b_1}c + \frac{a_0 L}{b_0} = 0
$$
Notice that $\dps\frac{a_nL}{b_n}, \ldots, \frac{a_0L}{b_0}$ are all integers (because all the denominators can be cancelled out with one of the factors of $L$). So $c$ is the root of a polynomial
$$
q(x) = \frac{a_n L}{b_n}x^n + \cdots + \frac{a_1 L}{b_1}x + \frac{a_0 L}{b_0}
$$
where $q(x)$ has all integer coefficients, {\it [as was to be shown.]}
\end{proof}

{\bf \cy Use the properties of even and odd integers that are listed in example 4.3.3 to do exercises 33 and 34.}

\subsection{Exercise 33}
When expressions of the form $(x - r)(x - s)$ are multiplied out, a quadratic polynomial is obtained. For instance, $(x - 2)(x - (-7)) = (x - 2)(x + 7) = x^2 + 5x - 14$.

\subsubsection{(a)}
What can be said about the coefficients of the polynomial obtained by multiplying out $(x - r)(x - s)$ when both $r$ and $s$ are odd integers? When both $r$ and $s$ are even integers? When one of $r$ and $s$ is even and the other is odd?

\begin{proof}
Note that $(x - r)(x - s) = x^2 - (r + s)x + rs$. 

If both $r$ and $s$ are odd, then $r + s$ is even and $rs$ is odd. So the coefficient of $x^2$ is 1 (odd), the coefficient of $x$ is even, and the constant coefficient, $rs$, is odd.

If both $r$ and $s$ are even, then $r + s$ is even and $rs$ is even. So the coefficient of $x^2$ is 1 (odd), the coefficient of $x$ is even, and the constant coefficient, $rs$, is even.

If one of $r$ and $s$ is even and the other is odd, then $r + s$ is odd and $rs$ is even. So the coefficient of $x^2$ is 1 (odd), the coefficient of $x$ is odd, and the constant coefficient, $rs$, is even.
\end{proof}

\subsubsection{(b)}
It follows from part (a) that $x^2 - 1253x + 255$ cannot be written as a product of two polynomials with integer coefficients. Explain why this is so.

\begin{proof}
Assume $x^2 - 1253x + 255 = (x-r)(x-s)$ where $r,s$ are real numbers. So $r+s = 1253$ and $rs = 255$. If $r,s$ are both integers, then since $rs = 255$, $r$ and $s$ must be both odd. But this is impossible, because then $r+s$ would be even, but 1253 is not even! Therefore $r,s$ cannot be both integers.
\end{proof}

\subsection{Exercise 34}
Observe that
$$
(x-r)(x-s)(x-t) = x^3 - (r+s+t)x^2 + (rs+rt+st)x - rst
$$
\subsubsection{(a)}
Derive a result for cubic polynomials similar to the result in part (a) of exercise 33 for quadratic polynomials.

\begin{proof}
If $r,s,t$ are all odd, then the constant coefficient is odd, the coefficient of $x$ is odd, the coefficient of $x^2$ is odd, and the coefficient of $x^3$ is odd (it's 1).

If exactly one of $r,s,t$ is even, then the constant coefficient is even, the coefficient of $x$ is odd, the coefficient of $x^2$ is even, and the coefficient of $x^3$ is odd (it's 1).

If exactly two of $r,s,t$ are even, then the constant coefficient is even, the coefficient of $x$ is even, the coefficient of $x^2$ is odd, and the coefficient of $x^3$ is odd (it's 1).

If $r,s,t$ are all even, then the constant coefficient is even, the coefficient of $x$ is even, the coefficient of $x^2$ is even, and the coefficient of $x^3$ is odd (it's 1).
\end{proof}

\subsubsection{(b)}
Can $x^3 + 7x^2 - 8x - 27$ be written as a product of three polynomials with integer coefficients? Explain.

\begin{proof}
Assume $x^3 + 7x^2 - 8x - 27 = (x-r)(x-s)(x-t)$ where $r,s,t$ are integers.

The coefficient of $x^2$ is 7, which is odd. So by part (a), either $r,s,t$ are all odd, or exactly two of them are even.

They can't be all odd, because then the coefficient of $x$ would be odd, but it's $-8$ which is even.

If exactly two of them are even, then the constant coefficient would have to be even, but it's $-27$ which is odd.

So it's impossible for $r,s,t$ to be all integers.
\end{proof}

{\bf \cy In $35-39$ find the mistakes in the “proofs” that the sum of any two rational numbers is a rational number.}

\subsection{Exercise 35}
“{\bf Proof:} Any two rational numbers produce a rational number when added together. So if $r$ and $s$ are particular but arbitrarily chosen rational numbers, then $r + s$ is rational.”

\begin{proof}
This “proof” assumes what is to be proved.
\end{proof}

\subsection{Exercise 36}
“{\bf Proof:} Let rational numbers $r = \frac{1}{4}$ and $s = \frac{1}{2}$ be given. Then $r + s = \frac{1}{4} + \frac{1}{2} = \frac{3}{4}$, which is a rational number. This is what was to be shown.”

\begin{proof}
This “proof” argues from a single example. It does not establish the result for {\it[arbitrarily chosen]} rational numbers.
\end{proof}

\subsection{Exercise 37}
“{\bf Proof:} Suppose $r$ and $s$ are rational numbers. By
definition of rational, $r = a/b$ for some integers $a$ and $b$ with $b \neq 0$, and $s = a/b$ for some integers $a$ and $b$ with $b \neq 0$. Then $r+s = \dps\frac{a}{b} + \frac{a}{b} = \frac{2a}{b}$.

Let $p = 2a$. Then p is an integer since it is a product of integers. Hence $r + s = p/b$, where $p$ and $b$ are integers and $b \neq 0$. Thus $r + s$ is a rational number by definition of rational. This is what was to be shown.”

\begin{proof}
By setting both $r$ and $s$ equal to $a/b$, this incorrect proof violates the requirement that $r$ and $s$ be arbitrarily chosen rational numbers. If both $r$ and $s$ equal $a/b$, then $r=s$.
\end{proof}

\subsection{Exercise 38}
“{\bf Proof:} Suppose $r$ and $s$ are rational numbers. Then $r = a/b$ and $s = c/d$ for some integers $a, b, c$, and $d$ with $b \neq 0$ and $d \neq 0$ (by definition of rational). Then
$$
r = \frac{a}{b}+\frac{c}{d}
$$
But this is a sum of two fractions, which is a fraction. So $r + s$ is a rational number since a rational number is a fraction.”

\begin{proof}
This ``proof'' does not establish that ``the sum of two fractions is a fraction''. Also, ``a rational number is a fraction'' is ambiguous. A rational number is a quotient of two integers (with nonzero denominator). But ``a fraction'' could be a fraction of non-integer numbers too. 
\end{proof}

\subsection{Exercise 39}
“{\bf Proof:} Suppose $r$ and $s$ are rational numbers. If $r + s$ is rational, then by definition of rational $r + s = a/b$ for some integers $a$ and $b$ with $b \neq 0$. 

Also since $r$ and $s$ are rational, $r = i/j$ and $s = m/n$ for some integers $i, j, m$, and $n$ with $j \neq 0$ and $n \neq 0$. It follows that
$$
r+s = \frac{i}{j} + \frac{m}{n} = \frac{a}{b}
$$
which is a quotient of two integers with a nonzero denominator. Hence it is a rational number. This is what was to be shown.”

\begin{proof}
This ``proof'' assumes what is to be proved.
\end{proof}

\section{Exercise Set 4.4}

{\bf \cy Give a reason for your answer in each of $1-13$. assume that all variables represent integers.}

\subsection{Exercise 1}
Is 52 divisible by 13?

\begin{proof}
Yes, $52 = 13 \cdot 4$
\end{proof}

\subsection{Exercise 2}
Does $7 \mid 56$?

\begin{proof}
Yes, $56 = 7 \cdot 8$
\end{proof}

\subsection{Exercise 3}
Does $5 \mid 0$?

\begin{proof}
Yes, $0 = 5 \cdot 0$
\end{proof}

\subsection{Exercise 4}
Does 3 divide $(3k + 1)(3k + 2)(3k + 3)$?

\begin{proof}
Yes, $(3k + 1)(3k + 2)(3k + 3) = 3[(3k + 1)(3k + 2)
(k + 1)]$, and $(3k + 1)(3k + 2)(k + 1)$ is an integer because $k$ is an integer and sums and products of integers
are integers.
\end{proof}

\subsection{Exercise 5}
Is $6m(2m + 10)$ divisible by 4?

\begin{proof}
Yes: $6m(2m + 10) = (2(3m))(2(m + 5)) = 4(3m)(m + 5)$, and $(3m)(m + 5)$ is an integer because $m$ is an integer and sums and products of integers are integers.
\end{proof}

\subsection{Exercise 6}
Is 29 a multiple of 3?

\begin{proof}
No, $29/3 \approx 9.67$, which is not an integer.
\end{proof}

\subsection{Exercise 7}
Is $-3$ a factor of 66?

\begin{proof}
Yes, $66 = (-3)(-22)$.
\end{proof}

\subsection{Exercise 8}
Is $6a(a + b)$ a multiple of $3a$?

\begin{proof}
Yes, $6a(a + b) = 3a[2(a + b)]$, and $2(a + b)$ is an integer because $a$ and $b$ are integers and sums and products of integers are integers.
\end{proof}

\subsection{Exercise 9}
Is 4 a factor of $2a\cdot 34b$?

\begin{proof}
Yes: $2a\cdot 34b = 2a \cdot (2\cdot(17b)) = 4(a\cdot(17b))$
\end{proof}

\subsection{Exercise 10}
Does $7 \mid 34$?

\begin{proof}
No, $34/7 \approx 4.86$, which is not an integer.
\end{proof}

\subsection{Exercise 11}
Does $13 \mid 73$?

\begin{proof}
No, $73/13 \approx 5.61$, which is not an integer.
\end{proof}

\subsection{Exercise 12}
If $n = 4k + 1$, does 8 divide $n^2 - 1$?

\begin{proof}
Yes, $n^2 - 1 = (4k + 1)^2 - 1 = (16k^2 + 8k + 1) - 1 =
16k^2 + 8k = 8(2k^2 + k)$, and $2k^2 + k$ is an integer because $k$ is an integer and sums and products of integers are integers.
\end{proof}

\subsection{Exercise 13}
If $n = 4k + 3$, does 8 divide $n^2 - 1$?

\begin{proof}
Yes, $n^2 - 1 = (4k + 3)^2 - 1 = (16k^2 + 24k + 9) - 1 =
16k^2 + 24k + 8 = 8(2k^2 + 3k + 1)$, and $2k^2 + 3k + 1$ is an integer because $k$ is an integer and sums and products of integers are integers.
\end{proof}

\subsection{Exercise 14}
Fill in the blanks in the following proof that for all integers $a$ and $b$, if $a \mid b$ then $a \mid (-b)$.

{\bf Proof:} Suppose $a$ and $b$ are any integers such that
{\cy (a)} \fbl. By definition of divisibility, there exists an integer $r$ such that {\cy (b)} \fbl. By substitution, 
$$
-b = -(ar) = a(-r).
$$
Let $t = $ {\cy(c)} \fbl. Then $t$ is an integer because $t = (-1)\cdot r$, and both $-1$ and $r$ are integers. Thus, by substitution, $-b = at$, where $t$ is an integer, and so by definition of divisibility, {\cy (d)} \fbl, as was to be shown.

\begin{proof}
(a) $a \mid b$ (b) $b = a \cdot r$ (c) $-r$ (d) $a \mid (-b)$
\end{proof}

{\bf \cy Prove statements $15-17$ directly from the definition of divisibility.}

\subsection{Exercise 15}
For all integers $a, b$, and $c$, if $a \mid b$ and $a \mid c$ then $a \mid (b + c)$.

\begin{proof}
Suppose $a, b$, and $c$ are any integers such that $a\mid b$ and $a \mid c$. {\it [We must show that $a \mid (b + c)$.]} 

By definition of divides, $b = ar$ and $c = as$ for some integers $r$ and $s$. Then $b + c = ar + as = a(r + s)$ by algebra.

Let $t = r + s$. Then $t$ is an integer (being a sum of integers), and thus $b + c = at$ where $t$ is an integer. By definition of divides, then, $a \mid (b + c)$ {\it [as was to be shown]}.
\end{proof}

\subsection{Exercise 16}
For all integers $a, b$, and $c$, if $a \mid b$ and $a \mid c$ then $a \mid (b - c)$.

\begin{proof}
Suppose $a, b$, and $c$ are any integers such that $a\mid b$ and $a \mid c$. {\it [We must show that $a \mid (b - c)$.]} 

By definition of divides, $b = ar$ and $c = as$ for some integers $r$ and $s$. Then $b - c = ar - as = a(r - s)$ by algebra.

Let $t = r - s$. Then $t$ is an integer (being a sum of integers), and thus $b - c = at$ where $t$ is an integer. By definition of divides, then, $a \mid (b - c)$ {\it [as was to be shown]}.
\end{proof}

\subsection{Exercise 17}
For all integers $a, b, c$, and $d$, if $a \mid c$ and $b \mid d$ then $ab \mid cd$.

\begin{proof}
Suppose $a, b, c$, and $d$ are any integers such that $a \mid c$ and $b \mid d$. {\it [We must show that $ab \mid cd $.]} 

By definition of divides, $c = ar$ and $d = bs$ for some integers $r$ and $s$. Then $cd = (ar)(bs) = ab(rs)$ by algebra.

Let $t = rs$. Then $t$ is an integer (being a product of integers), and thus $cd = abt$ where $t$ is an integer. By definition of divides, then, $ab \mid cd)$ {\it [as was to be shown]}.
\end{proof}

\subsection{Exercise 18}
Consider the following statement: The negative of any multiple of 3 is a multiple of 3.

\subsubsection{(a)}
Write the statement formally using a quantifier and a variable.

\begin{proof}
$\fa$ integers $n$ if $n$ is a multiple of 3 then $-n$ is a multiple of 3.
\end{proof}

\subsubsection{(b)}
Determine whether the statement is true or false and justify your answer.

\begin{proof}
The statement is true. Suppose $n$ is any integer that is a multiple of 3. {\it [We must show that $-n$ is a multiple of 3.]} 

By definition of multiple, $n = 3k$ for some integer $k$. Then $-n = -(3k) = 3(-k)$ by substitution and by algebra.

Now $-k$ is an integer because $k$ is. Hence, by definition of multiple, $-n$ is a multiple of 3 {\it [as was to be shown]}.
\end{proof}

\subsection{Exercise 19}
Show that the following statement is false: For all integers $a$ and $b$, if $3\mid (a + b)$ then $3 \mid (a - b)$.

\begin{proof}
\underline{Counterexample:} Let $a = 2$ and $b = 1$. Then
$a + b = 2 + 1 = 3$, and so $3 \mid (a + b)$ because $3 = 3\cdot 1$.

On the other hand, $a - b = 2 - 1 = 1$, and $3 \nmid 1$ because $1/3$ is not an integer. Thus $3 \nmid (a - b)$. {\it [So the hypothesis of the statement is true and its conclusion is false.]}
\end{proof}

{\bf \cy For each statement in $20-31$, determine whether the statement is true or false. prove the statement directly from the definitions if it is true, and give a counterexample if it is false.}

\subsection{Exercise 20}
The sum of any three consecutive integers is divisible by 3.

\begin{proof}
True. Assume $a,b,c$ are any three consecutive integers. By definition of consecutive, $a = n, b = n+1, c = n+2$ for some integer $n$. Then $a+b+c = n+n+1+n+2 = 3n+6 = 3(n+2)$. Let $t = n+2$. Then $t$ is an integer (being a sum of integers). So $a+b+c = 3t$ where $t$ is an integer. So by definition of divisibility, $a+b+c$ is divisible by 3.
\end{proof}

\subsection{Exercise 21}
The product of any two even integers is a multiple of 4.

\begin{proof}
True. Assume $a,b$ are any two integers. By definition of even, $a = 2r$ and $b = 2s$ for some integers $r,s$. Then $ab = (2r)(2s) = 4rs$. Let $t = rs$. Then $t$ is an integer (being a product of integers). So $ab = 4t$ where $t$ is an integer. So $ab$ is a multiple of 4.
\end{proof}

\subsection{Exercise 22}
A necessary condition for an integer to be divisible by 6 is that it be divisible by 2.

\begin{proof}
Rewriting the statement, we get:

$\fa$ integers $n$, if $n$ is divisible by 6 then $n$ is divisible by 2.

True. Assume $n$ is any integer divisible by 6. By definition of divisibility, $n = 6m$ for some integer $m$. Then $n = 6m = 2(3m)$. Let $t = 3m$. Then $t$ is an integer because it is a product of integers. So $n = 2t$ where $t$ is an integer, therefore $n$ is divisible by 2 by the definition of divisibility.
\end{proof}

\subsection{Exercise 23}
A sufficient condition for an integer to be divisible by 8 is that it be divisible by 16.

\begin{proof}
Rewriting the statement, we get:

$\fa$ integers $n$, if $n$ is divisible by 16 then $n$ is divisible by 8.

True. Assume $n$ is any integer divisible by 16. By definition of divisibility, $n = 16m$ for some integer $m$. Then $n = 16m = 8(2m)$. Let $t = 2m$. Then $t$ is an integer because it is a product of integers. So $n = 8t$ where $t$ is an integer, therefore $n$ is divisible by 8 by the definition of divisibility.
\end{proof}

\subsection{Exercise 24}
For all integers $a, b$, and $c$, if $a \mid b$ and $a \mid c$ then $a \mid (2b - 3c)$.

\begin{proof}
The statement is true. Suppose $a, b$, and $c$ are any integers such that $a\mid b$ and $a\mid c$. {\it [We must show that $a\mid (2b - 3c)$.]} 

By definition of divisibility, we know that $b = am$ and $c = an$ for some integers $m$ and $n$. 

It follows that $2b - 3c = 2(am) - 3(an)$ (by substitution) $= a(2m - 3n)$ (by basic algebra). 

Let $t = 2m - 3n$. Then $t$ is an integer because it is a difference of products of integers. Hence $2b - 3c = at$, where $t$ is an integer, and so $a\mid (2b - 3c)$ by definition of divisibility {\it [as was to be shown]}.
\end{proof}

\subsection{Exercise 25}
For all integers $a, b$, and $c$, if $a$ is a factor of $c$ and $b$ is a factor of $c$ then $ab$ is a factor of $c$.

\begin{proof}
The statement is false. \underline{Counterexample:} Let
$a = 2, b = 8$, and $c = 8$. Then $a$ is a factor of $c$ because $8 = 2\cdot 4$ and $b$ is a factor of $c$ because $8 = 1\cdot8$, but $ab = 16$ and 16 is not a factor of 8 because $8 \neq 16\cdot k$ for any integer $k$ since $8/16 = 1/2$.
\end{proof}

\subsection{Exercise 26}
For all integers $a, b$, and $c$, if $ab \mid c$ then $a \mid c$ and $b \mid c$.

\begin{proof}
True. Assume $a,b,c$ are any three integers such that $ab \mid c$. By definition of divides, $abm = c$ for some integer $m$. Let $t = bm$ and $s = am$. Then $t$ and $s$ are integers (being products of integers). So $c = at$ and $c = bs$ where $t$ and $s$ are integers. Therefore by definition of divides, $a \mid c$ and $b \mid c$.
\end{proof}

\subsection{Exercise 27}
For all integers $a, b$, and $c$, if $a\mid (b + c)$ then $a \mid b$ or $a\mid c$.

\begin{proof}
False \underline{Counterexample:} Let $a=6,b=2,c=4$. Then $b+c = 6$ and so $b+c = 6 = 6 \cdot 1 = a \cdot 1$ therefore $a \mid (b+c)$. However, $a \nmid b$ because $6 \nmid 2$ because 2/6 is not an integer; similarly $a \nmid c$ because $6 \nmid 4$ because 4/6 is not an integer.
\end{proof}

\subsection{Exercise 28}
For all integers $a, b$, and $c$, if $a\mid bc$ then $a\mid b$ or $a\mid c$.

\begin{proof}
False \underline{Counterexample:} Let $a=6,b=2,c=3$. Then $bc = 6$ and so $bc = 6 = 6 \cdot 1 = a \cdot 1$ therefore $a \mid bc$. However, $a \nmid b$ because $6 \nmid 2$ because 2/6 is not an integer; similarly $a \nmid c$ because $6 \nmid 3$ because 3/6 is not an integer.
\end{proof}

\subsection{Exercise 29}
For all integers $a$ and $b$, if $a\mid b$ then $a^2 \mid b^2$.

\begin{proof}
True. Suppose $a,b$ are any two integers such that $a \mid b$. By definition of divides, $b = ac$ for some integer $c$. Then $b^2 = (ac)^2 = a^2c^2$. Let $t = c^2$. Then $t$ is an integer (being the square of an integer). So $b^2 = a^2 \cdot t$ where $t$ is an integer. So by definition of divides, $a^2 \mid b^2$.
\end{proof}

\subsection{Exercise 30}
For all integers $a$ and $n$, if $a\mid n^2$ and $a \leq n$ then $a \mid n$.

\begin{proof}
False. \underline{Counterexample:} Let $a = 4, n = 6$. Then $n^2 = 36$, so $a \mid n^2$ because $4 \mid 36$ because $36 = 4 \cdot 9$. But $a \nmid n$ because $4 \nmid 6$ because 6/4 is not an integer.
\end{proof}

\subsection{Exercise 31}
For all integers $a$ and $b$, if $a\mid 10b$ then $a\mid 10$ or $a\mid b$.

\begin{proof}
False. \underline{Counterexample:} Let $a = 15, b = 3$. Then $10b = 30$, so $a \mid 10b$ because $15 \mid 30$ because $30 = 15 \cdot 2$. But $a \nmid 10$ because $15 \nmid 10$ because 10/15 is not an integer; similarly $a \nmid b$ because $15 \nmid 3$ because 3/15 is not an integer.
\end{proof}

\subsection{Exercise 32}
A fast-food chain has a contest in which a card with numbers on it is given to each customer who makes a purchase. If some of the numbers on the card add up to 100, then the customer wins \$100. A certain customer receives a card containing the numbers 
\begin{center}
72, 21, 15, 36, 69, 81, 9, 27, 42, and 63. 
\end{center}
Will the customer win \$100? Why or why not?

\begin{proof}
Each of these numbers is divisible by 3, and so their sum is also divisible by 3. But 100 is not divisible by 3. Thus the sum cannot equal \$100.
\end{proof}

\subsection{Exercise 33}
Is it possible to have a combination of nickels, dimes, and quarters that add up to \$4.72? Explain.

\begin{proof}
Impossible. Nickels are 5 cents, dimes are 10 cents, quarters are 25 cents. These are all divisible by 5. Therefore the cents value any combination of nickels, dimes, and quarters is also divisible by 5. But the cents value of \$4.72 is 472, which is not divisible by 5 (it does not end in 0 or 5).
\end{proof}

\subsection{Exercise 34}
Consider a string consisting of $a$’s, $b$’s, and $c$’s where the number of $b$’s is three times the number of $a$’s and the number of $c$’s is five times the number of $a$’s. Prove that the length of the string is divisible by 3.

\begin{proof}
Suppose that the number of $a$'s in the string $S$ is $n$ where $n$ is some integer. Then the number of $b$'s is $3n$, and the number of $c$'s is $5n$.

The total length of the string is: number of $a$'s + number of $b$'s + number of $c$'s = $n + 3n + 5n = 9n$. Since $9n = 3(3n)$ and $3n$ is an integer, $9n$ is divisible by 3.
\end{proof}

\subsection{Exercise 35}
Two athletes run a circular track at a steady pace so that the first completes one round in 8 minutes and the second in 10 minutes. If they both start from the same spot at 4 p.m., when will be the first time they return to the start
together?

\begin{proof}
The length of time needed is the least common multiple of 8 and 10, which is 40 minutes. So they will be back at the start together again for the first time at 4:40 p.m.
\end{proof}

\subsection{Exercise 36}
It can be shown (see exercises $44-48$) that an integer is divisible by 3 if, and only if, the sum of its digits is divisible by 3; an integer is divisible by 9 if, and only if, the sum of its digits is divisible by 9; an integer is divisible by 5 if, and only if, its right-most digit is a 5 or a 0; and an integer is divisible by 4 if, and only if, the number formed by its right-most two digits is divisible by 4. Check the following integers for divisibility by 3, 4, 5, and 9.

\subsubsection{(a)}
637,425,403,705,125

\begin{proof}
The sum of the digits is 54, which is divisible by 9. Therefore, 637,425,403,705,125 is divisible by 9 and hence also divisible by 3 (by transitivity of divisibility). Because the rightmost digit is 5, then 637,425,403,705,125 is divisible by 5. And because the two rightmost digits are 25, which is not divisible by 4, then 637,425,403,705,125 is not divisible by 4.
\end{proof}

\subsubsection{(b)}
12,858,306,120,312

\begin{proof}
$1+2+8+5+8+3+0+6+1+2+0+3+1+2=42$. It is divisible by 3 (because $42 / 3 = 14$ is an integer), not divisible by 9 (because $42/9 \approx 4.67$ which is not an integer).

Last 2 digits are 12, so it's divisible by 4 (because $12/4 = 3 $ is an integer) and not divisible by 5 (because $2 \neq 0$ or $5$).
\end{proof}

\subsubsection{(c)}
517,924,440,926,512

\begin{proof}
Last two digits are 12, so divisible by 4, not divisible by 5. Sum of digits is 61, so not divisible by 3 or 9.
\end{proof}

\subsubsection{(d)}
14,328,083,360,232

\begin{proof}
Last two digits are 32, so divisible by 4, not divisible by 5. Sum of digits is 45, so divisible by 3 and 9.
\end{proof}

\subsection{Exercise 37}
Use the unique factorization theorem to write the following integers in standard factored form.

\subsubsection{(a)}
1,176

\begin{proof}
$1,176 = 2^3 \cdot 3 \cdot 7^2$
\end{proof}

\subsubsection{(b)}
5,733

\begin{proof}
$5,733 = 3^2 \cdot 7^2 \cdot 13$
\end{proof}

\subsubsection{(c)}
3,675

\begin{proof}
$3,675 = 3 \cdot 5^2 \cdot 7^2$
\end{proof}

\subsection{Exercise 38}
Let $n = 8,424$.

\subsubsection{(a)}
Write the prime factorization for $n$.

\begin{proof}
$8,424 = 2^3 \cdot 3^4 \cdot 13$
\end{proof}

\subsubsection{(b)}
Write the prime factorization for $n^5$.

\begin{proof}
$8,424^5 = (2^3 \cdot 3^4 \cdot 13)^5 = 2^{15} \cdot 3^{20} \cdot 13^5$
\end{proof}

\subsubsection{(c)}
Is $n^5$ divisible by 20? Explain.

\begin{proof}
The answer is no. $20 = 2^2 \cdot 5$. So if $n^5$ is divisible by 20, it is divisible by 5. But 5 is not a factor of $n^5$, so it is not divisible by 5.
\end{proof}

\subsubsection{(d)}
What is the least positive integer $m$ so that $8,424\cdot m$ is a perfect square?

\begin{proof}
The answer is 26. 

In order for $8,424 \cdot m$ to be a perfect square, each prime factor must be raised to an even power.

$8,424 = 2^3 \cdot 3^4 \cdot 13$. The prime factor $3$ already has an even power of 4. The prime factor $2$ has an odd power of 3, the closest even power is 4. The prime factor $13$ has an odd power of 1, the closest even power is 2. 

So let $m = 2 \cdot 13 = 26$. Then $8,424 \cdot m = 2^3 \cdot 3^4 \cdot 13 \cdot (2 \cdot 13) = 2^4 \cdot 3^4 \cdot 13^2 = (2^2 \cdot 3^2 \cdot 13^1)^2$ is a perfect square. 

$m = 26$ is the least positive integer for this; any positive integer smaller than 26 will not result in $8,424 \cdot m$ being a perfect square, since we need to increase the powers of both prime factors 2 and 13 by at least 1.
\end{proof}

\subsection{Exercise 39}
Suppose that in standard factored form $a = p_1^{e_1} p_2^{e_2} \ldots p_k^{e_k}$, where $k$ is a positive integer; $p_1, p_2, \ldots, p_k$ are prime numbers; and $e_1, e_2, \ldots , e_k$ are positive integers.

\subsubsection{(a)}
What is the standard factored form for $a^3$?

\begin{proof}
$a^3 = (p_1^{e_1} p_2^{e_2} \ldots p_k^{e_k})^3 = p_1^{3e_1} p_2^{3e_2} \ldots p_k^{3e_k}$
\end{proof}

\subsubsection{(b)}
Find the least positive integer $k$ such that $2^4 \cdot 3^5 \cdot 7 \cdot 11^2 \cdot k$ is a perfect cube (that is, it equals an integer to the third power). Write the resulting product as a perfect cube.

\begin{proof}
Every prime factor must be raised to a power that is a multiple of 3.

So $2^4 \cdot 3^5 \cdot 7 \cdot 11^2$ should be turned into: $2^6 \cdot 3^6 \cdot 7^3 \cdot 11^3$ (because those powers are the closest multiples of 3).

Therefore, let $k = 2^2 \cdot 3^1 \cdot 7^2 \cdot 11^1$. Now 

$2^4 \cdot 3^5 \cdot 7 \cdot 11^2 \cdot k = 2^4 \cdot 3^5 \cdot 7 \cdot 11^2 \cdot (2^2 \cdot 3^1 \cdot 7^2 \cdot 11^1) = 2^6 \cdot 3^6 \cdot 7^3 \cdot 11^3 = (2^2 \cdot 3^2 \cdot 7^1 \cdot 11^1)^3$ is a perfect cube.
\end{proof}

\subsection{Exercise 40}
\subsubsection{(a)}
If $a$ and $b$ are integers and $12a = 25b$, does $12 \mid b$? does $25 \mid a$? Explain.

\begin{proof}
Because $12a = 25b$, the unique factorization theorem guarantees that the standard factored forms of $12a$ and $25b$ must be the same. Thus $25b$ contains the factors $2^2 \cdot 3 (=12)$. But since neither 2 nor 3 divides 25, the factors $2^2 \cdot 3$ must all occur in $b$, and hence $12\mid b$. Similarly, $12a$ contains the factors $5^2 = 25$, and since 5 is not a factor of 12, the factors $5^2$ must occur in $a$. So $25\mid a$.
\end{proof}

\subsubsection{(b)}
If $x$ and $y$ are integers and $10x = 9y$, does $10 \mid y$? does $9 \mid x$? Explain.

\begin{proof}
Because $10x = 9y$, the unique factorization theorem guarantees that the standard factored forms of $10x$ and $9y$ must be the same. Thus $9y$ contains the factors $2 \cdot 5 (=10)$. But since neither 2 nor 5 divides 9, the factors $2 \cdot 5$ must all occur in $y$, and hence $10\mid y$. Similarly, $10x$ contains the factors $3^2 = 9$, and since 3 is not a factor of 10, the factors $3^2$ must occur in $x$. So $9\mid x$.
\end{proof}

\subsection{Exercise 41}
How many zeros are at the end of $45^8 \cdot 88^5$? Explain how you can answer this question without actually computing the number. (Hint: $10 = 2\cdot 5$.)

\begin{proof}
$45^8 \cdot 88^5 =(3^2 \cdot 5)^8 \cdot (2^3 \cdot 11)^5 = 3^{16} \cdot 5^8 \cdot 2^{15} \cdot 11^5$. 

Since $10 = 2 \cdot 5$, there are 8 factors of 10 in this number, because 8 is the smaller power between the powers of 2 and 5. 

$5^8 \cdot 2^{15}$ can be combined as $(2\cdot5)^8 \cdot 2^7 = 10^8 \cdot 2^7$. So 7 powers of 2 are ``left over'' and they do not contribute to powers of 10 (in other words, zeros at the end of the number). 
\end{proof}

\subsection{Exercise 42}
If $n$ is an integer and $n > 1$, then $n!$ is the product of $n$ and every other positive integer that is less than $n$. For example, $5! = 5\cdot4\cdot3\cdot2\cdot1$.

\subsubsection{(a)}
Write $6!$ in standard factored form.

\begin{proof}
$6! = 6 \cdot 5 \cdot 4 \cdot 3 \cdot 2 \cdot 1 = 2 \cdot 3 \cdot 5 \cdot 2 \cdot 2 \cdot 3 \cdot 2 = 2^4 \cdot 3^2  \cdot 5$
\end{proof}

\subsubsection{(b)}
Write $20!$ in standard factored form.

\begin{proof}
$(20)! = 20 \cdot 19 \cdot 18 \cdot 17 \cdot 16 \cdot 15 \cdot 14 \cdot 13 \cdot 12 \cdot 11 \cdot 10 \cdot 9 \cdot 8 \cdot 7 \cdot 6 \cdot 5 \cdot 4 \cdot 3 \cdot 2$

$= (2^2 \cdot 5) \cdot 19 \cdot (2 \cdot 3^2) \cdot 17 \cdot (2^4) \cdot (3 \cdot 5) \cdot (2 \cdot 7) \cdot 13 \cdot (2^2 \cdot 3) \cdot 11 \cdot (2 \cdot 5) \cdot (3^2) \cdot (2^3) \cdot 7 \cdot (2 \cdot 3) \cdot 5 \cdot (2^2) \cdot 3 \cdot 2$

$= 2^{18} \cdot 3^{8} \cdot 5^4 \cdot 7^2 \cdot 11 \cdot 13 \cdot 17 \cdot 19$
\end{proof}

\subsubsection{(c)}
Without computing the value of $(20!)^2$ determine how many zeros are at the end of this number when it is written in decimal form. Justify your answer.

\begin{proof}
By part (b), $(20!) = 2^{18} \cdot 3^{8} \cdot 5^4 \cdot 7^2 \cdot 11 \cdot 13 \cdot 17 \cdot 19$.

So $(20!)^2 = (2^{18} \cdot 3^{8} \cdot 5^4 \cdot 7^2 \cdot 11 \cdot 13 \cdot 17 \cdot 19)^2 = 2^{36} \cdot 3^{16} \cdot 5^8 \cdot 7^4 \cdot 11^2 \cdot 13^2 \cdot 17^2 \cdot 19^2$.

To find the number of zeros at the end of this number, we look for the largest power of 10 in it. The largest power of 10 is the smaller of the powers of 2 and 5 (36 and 8), which is 8. So there are 8 zeros at the end.
\end{proof}

\subsection{Exercise 43}
At a certain university $2/3$ of the mathematics students and $3/5$ of the computer science students have taken a discrete mathematics course. The number of mathematics students who have taken the course equals the number of computer science students who have taken the course. If there are at least 100 mathematics students at the university, what are the least possible number of mathematics students and the least possible number of computer science students at the university?

\begin{proof}
Let $m$ and $c$ denote the number of mathematics and computer science students, respectively.

We are given that $2m/3 = 3c/5$. So $10m = 9c$. By Exercise 40 part (b), we have $10 \mid c$ and $9 \mid m$.

The smallest possible value for $m$ such that $9 \mid m$ and $m \geq 100$ is $m = 108$. (99 is divisible by 9, then 108, then 117...)

Putting this back into the equation we get $10(108) = 9c$ and solving for $c$ we get $c = 10(108)/9 = 120$.
\end{proof}

\begin{tcolorbox}[colframe=cyan]
{\bf \cy Definition:} Given any nonnegative integer $n$, the {\bf decimal representation} of $n$ is an expression of the form 
$$
d_k d_{k-1} \cdots d_2d_1d_0,
$$ 
where $k$ is a nonnegative integer, $d_0, d_1, d_2, \ldots, d_k$ (called the {\bf decimal digits} of $n$) are integers from 0 to 9 inclusive, $d_k \neq 0$ unless $n = 0$ and $k = 0$, and 
$$
n = d_k \cdot 10^k + d_{k-1} \cdot 10^{k-1} + \cdots + d_2 \cdot 10^2 + d_1 \cdot 10 + d_0. 
$$
(For example, $2,503 = 2\cdot 10^3 + 5\cdot 10^2 + 0\cdot 10 + 3$.)
\end{tcolorbox}

\subsection{Exercise 44}
Prove that if $n$ is any nonnegative integer whose decimal representation ends in 0, then $5\mid n$. (Hint: If the decimal representation of a nonnegative integer $n$ ends in $d_0$, then $n = 10m + d_0$ for some integer $m$.)

\begin{proof}
Suppose $n$ is a nonnegative integer whose decimal representation ends in 0. Then $n = 10m + 0 = 10m$ for
some integer $m$. Factoring out a 5 yields $n = 10m = 5(2m)$, and $2m$ is an integer since $m$ is an integer. Hence $10m$ is divisible by 5, which is what was to be shown.
\end{proof}

\subsection{Exercise 45}
Prove that if $n$ is any nonnegative integer whose decimal representation ends in $5$, then $5\mid n$.

\begin{proof}
Assume that if $n$ is any nonnegative integer whose decimal representation ends in $5$. Then by definition,
$$
n = d_k \cdot 10^k + d_{k-1} \cdot 10^{k-1} + \cdots + d_2 \cdot 10^2 + d_1 \cdot 10 + 5
$$
for some nonnegative integer $k$ and integers $d_1, \ldots, d_k$ from 0 to 9. If $k = 0$ then $n = 5$ which is divisible by 5. So we can assume $k \geq 1$. Now
$$
n = 10d_k \cdot 10^{k-1} + 10d_{k-1} \cdot 10^{k-2} + \cdots + 10d_2 \cdot 10^{2-1} + 10 d_1 + 5
$$
Now factoring out a 5 we get
$$
n = 5(2d_k \cdot 10^{k-1} + 2d_{k-1} \cdot 10^{k-2} + \cdots + 2d_2 \cdot 10 + 2 d_1 + 1)
$$
Let $t = 2d_k \cdot 10^{k-1} + 2d_{k-1} \cdot 10^{k-2} + \cdots + 2d_2 \cdot 10 + 2 d_1 + 1$. Then $t$ is an integer because it is a sum of products of integers. So $5 \mid n$ by definition of divides.
\end{proof}

\subsection{Exercise 46}
Prove that if the decimal representation of a nonnegative integer $n$ ends in $d_1d_0$ and if $4 \mid (10d_1 + d0)$, then $4\mid n$. (Hint: If the decimal representation of a nonnegative integer $n$ ends in $d_1d_0$, then there is an integer $s$ such that $n = 100s + 10d_1 + d_0$.)

\begin{proof}
Assume the decimal representation of a nonnegative integer $n$ ends in $d_1d_0$ and assume $4 \mid (10d_1 + d0)$. Let
$$
n = d_k \cdot 10^k + d_{k-1} \cdot 10^{k-1} + \cdots + d_2 \cdot 10^2 + d_1 \cdot 10 + d_0. 
$$
be the decimal representation of $n$. Since the representation ends in $d_1d_0$, $k$ must be at least 2 or greater. Factoring out $10^2$ from the first $k-1$ terms, and leaving the last 2 terms alone, we get
$$
n = 10^2(d_k \cdot 10^{k-2} + d_{k-1} \cdot 10^{k-3} + \cdots + d_2 \cdot 10^{2-2}) + d_1 \cdot 10 + d_0. 
$$
Let $s = d_k \cdot 10^{k-2} + d_{k-1} \cdot 10^{k-3} + \cdots + d_2 \cdot 10^{2-2}$, which is an integer (being a sum of products of integers).

So $n = 100s + 10d_1 + d_0$. Since $4 \mid (10d_1 + d0)$, we have $10d_1 + d0 = 4t$ for some integer $t$. Therefore $n = 100s + 10d_1 + d_0 = 100s + 4t = 4(25s+t)$.

Let $r = 25s+t$, which is an integer (being a sum of products of integers). So $n = 4r$ where $r$ is an integer. So $4 \mid n$ by definition of divides.
\end{proof}

\subsection{Exercise 47}
Observe that
$$
\begin{array}{rcl}
7,524 &=& 7 \cdot 1,000 + 5 \cdot 100 + 2 \cdot 10 + 4\\
&=& 7(999 + 1) + 5(99 + 1) + 2(9 + 1) + 4\\
&=& (7 \cdot 999 + 7) + (5 \cdot 99 + 5) + (2 \cdot 9 + 2) + 4\\
&=& (7 \cdot 999 + 5 \cdot 99 + 2 \cdot 9) + (7 + 5 + 2 + 4)\\
&=& (7 \cdot 111 \cdot 9 + 5 \cdot 11 \cdot 9 + 2 \cdot 9) + (7 + 5 + 2 + 4)\\
&=& (7 \cdot 111 + 5 \cdot 11 + 2) \cdot 9 + (7 + 5 + 2 + 4)\\
&=& \text{(an integer divisible by 9) + (the sum of the digits of 7,524)}.
\end{array}
$$
Since the sum of the digits of 7,524 is divisible by 9, 7,524 can be written as a sum of two integers each of which is divisible by 9. It follows from exercise 15 that 7,524 is divisible by 9. Generalize the argument given in this example to any nonnegative integer $n$. In other words, prove that for any nonnegative integer $n$, if the sum of the digits of $n$ is divisible by 9, then $n$ is divisible by 9.

Hint: You may take it as a fact that for any positive
integer $k$, 
$$
10^k = 9\cdot10^{k-1} + 9\cdot10^{k-2} + \ldots + 9\cdot10^1 + 9\cdot10^0 + 1.
$$
\begin{proof}
First let's talk about the Hint. It tells us that, for any positive integer $k$, we have $10^k = 9(10^{k-1} +10^{k-2} + \ldots + 10^1 + 10^0) + 1$. (For example, if $k = 5$ then $10^5 = 100000 = 99999 + 1 = 9 \cdot 11111 + 1$.)

In other words, every power of 10 can be written as: (some integer divisible by 9) + 1, or equivalently as: ($9 \cdot$ some integer) + 1.

Assume $n$ is any nonnegative integer such that the sum of the digits of $n$ is divisible by 9. Let
$$
n = d_k \cdot 10^k + d_{k-1} \cdot 10^{k-1} + \cdots + d_2 \cdot 10^2 + d_1 \cdot 10 + d_0. 
$$
be the decimal representation of $n$. By the Hint, 
$$
n = d_k \cdot (9a_k + 1) + d_{k-1} \cdot (9a_{k-1}+1) + \cdots + d_2 \cdot (9a_2+1) + d_1 \cdot (9a_1+1) + d_0
$$
for some integers $a_1, \ldots, a_k$. Multiplying out, we get:
$$
n = (9d_ka_k + d_k) + (9d_{k-1}a_{k-1} + d_{k-1}) + \cdots + (9d_2 a_2 + d_2) + (9d_1a_1 + d_1) + d_0
$$
Reorganizing, we get
$$
n = 9(d_ka_k + d_{k-1}a_{k-1} + \cdots + d_2 a_2 + d_1a_1) + (d_k + d_{k-1} + \cdots + d_2 + d_1 + d_0)
$$
The first term is divisible by 9. Therefore, $n$ is divisible by 9 if, and only if, the second term $(d_k + d_{k-1} + \cdots + d_2 + d_1 + d_0)$, in other words the sum of $n$'s digits, is also divisible by 9.
\end{proof}

\subsection{Exercise 48}
Prove that for any nonnegative integer $n$, if the sum of the digits of $n$ is divisible by 3, then $n$ is divisible by 3.

\begin{proof}
This follows from the proof of Exercise 47. We can simply repeat the same proof, and at the end, since $9 = 3 \cdot 3$, we have
$$
n = (3 \cdot 3) \cdot(d_ka_k + d_{k-1}a_{k-1} + \cdots + d_2 a_2 + d_1a_1) + (d_k + d_{k-1} + \cdots + d_2 + d_1 + d_0)
$$
so the first term is divisible by 3, therefore $n$ is divisible by 3 if and only if the second term (sum of its digits) is also divisible by 3.
\end{proof}

\subsection{Exercise 49}
Given a positive integer $n$ written in decimal form, the alternating sum of the digits of $n$ is obtained by starting with the right-most digit, subtracting the digit immediately to its left, adding the next digit to the left, subtracting the next digit, and so forth. For example, the alternating sum of the digits of 180,928 is $8 - 2 + 9 - 0 + 8 - 1 = 22$. Justify the fact that for any nonnegative integer $n$, if the alternating sum of the digits of $n$ is divisible by 11, then $n$ is divisible by 11.

\begin{proof}
The idea is the same as in Exercises 47 and 48. This time, we have to notice that for even powers of 10, like $100$ for example, we can write it as $100 = 99 + 1 = 11 \cdot 9 + 1$; but for odd powers of 10, like $1000$ for example, we can write $1000 = 1001 - 1 = 11 \cdot 91 - 1$.

Taking a look at more examples of even powers of 10, we see $10,000 = 9,999 + 1 = 11 \cdot 909 + 1$, and $1,000,000 = 999,999 + 1 = 11 \cdot 90,909 + 1$ and so on.

Taking a look at more examples of odd powers of 10, we see $100,000 = 100,001 - 1 = 11 \cdot 9091 - 1$, and $10,000,000 = 10,000,001 - 1 = 11 \cdot 909,091 - 1$ and so on.

In general, if $k$ is a nonnegative integer, then we have $10^{2k} = 11 \cdot a + 1$ and $10^{2k+1} = 11 \cdot b - 1$ for some integers $a, b$. Putting these into one formula we get:
$$
\fa k \geq 0, \te a_k \geq 0 \text{ such that } 10^{k} = 11 \cdot a_k + (-1)^k.
$$
Now we can repeat the proof in Exercise 47 but using the above equation instead. We'll end up with:
$$
n = 11 \cdot(d_ka_k +\cdots + d_2 a_2 + d_1a_1) + ((-1)^{k}d_k +  \cdots + (-1)^{2}d_2 + (-1)^{1}d_1 + (-1)^{0}d_0)
$$
The first term on the right-hand side is divisible by 11. So $n$ is divisible by 11 if, and only if, the second term (the alternating sum of its digits) is divisible by 11.
\end{proof}

\subsection{Exercise 50}
The integer 123,123 has the form $abc,abc$, where $a, b$, and $c$ are integers from 0 through 9. Consider all six-digit integers of this form. Which prime numbers divide every one of these integers? Prove your answer.

\begin{proof}
11 divides every one of these integers. Every one of these integers has an alternating sum of digits equal to 0, because: $a - b + c - a + b - c = 0$. Since $11 \mid 0$, by Exercise 49, 11 divides every one of these integers.
\end{proof}

\section{Exercise Set 4.5}

{\bf \cy For each of the values of $n$ and $d$ given in $1-6$, find integers $q$ and $r$ such that $n = dq + r$ and $0 \leq r < d$.}

\subsection{Exercise 1}
$n = 70, d = 9$

\begin{proof}
$q = 7, r = 7$
\end{proof}

\subsection{Exercise 2}
$n = 62, d = 7$

\begin{proof}
$q = 8, r = 6$
\end{proof}

\subsection{Exercise 3}
$n = 36, d = 40$

\begin{proof}
$q = 0, r = 36$
\end{proof}

\subsection{Exercise 4}
$n = 3, d = 11$

\begin{proof}
$q = 0, r = 3$
\end{proof}

\subsection{Exercise 5}
$n = -45, d = 11$

\begin{proof}
$q = -5, r = 10$
\end{proof}

\subsection{Exercise 6}
$n = -27, d = 8$

\begin{proof}
$q = -4, r = 5$
\end{proof}

{\bf \cy Evaluate the expressions in $7-10$.}

\subsection{Exercise 7}

\subsubsection{(a)}
43 div 9

\begin{proof}
4
\end{proof}

\subsubsection{(b)}
$43 \mod 9$

\begin{proof}
7
\end{proof}

\subsection{Exercise 8}

\subsubsection{(a)}
50 div 7

\begin{proof}
7
\end{proof}

\subsubsection{(b)}
$50 \mod 7$

\begin{proof}
1
\end{proof}

\subsection{Exercise 9}

\subsubsection{(a)}
28 div 5

\begin{proof}
5
\end{proof}

\subsubsection{(b)}
$28 \mod 5$

\begin{proof}
3
\end{proof}

\subsection{Exercise 10}
30 div 2

\subsubsection{(a)}

\begin{proof}
15
\end{proof}

\subsubsection{(b)}
$30 \mod 2$

\begin{proof}
0
\end{proof}

\subsection{Exercise 11}
Check the correctness of formula (4.5.1) given in Example 4.5.3 for the following values of $DayT$ and $N$.

\subsubsection{(a)}
$DayT = 6$ (Saturday) and $N = 15$

\begin{proof}
When today is Saturday, 15 days from today is two weeks (which is Saturday) plus one day (which is Sunday). Hence $DayN$ should be 0. According to the formula, when today is Saturday, $DayT = 6$, and so when $N = 15$,
$$
\begin{array}{rcl}
DayN &=& (DayT + N) \mod 7 \\
&=& (6 + 15) \mod 7 \\
&=& 21 \mod 7 \\
&=& 0,
\end{array}
$$
which agrees.
\end{proof}

\subsubsection{(b)}
$DayT = 0$ (Sunday) and $N = 7$

\begin{proof}
When today is Sunday, 7 days from today is one week (which is Sunday). Hence $DayN$ should be 0. According to the formula, when today is Sunday, $DayT = 0$, and so when $N = 7$,
$$
\begin{array}{rcl}
DayN &=& (DayT + N) \mod 7 \\
&=& (0 + 7) \mod 7 \\
&=& 7 \mod 7 \\
&=& 0,
\end{array}
$$
which agrees.
\end{proof}

\subsubsection{(c)}
$DayT = 4$ (Thursday) and $N = 12$

\begin{proof}
When today is Thursday, 12 days from today is one week (which is Thursday) plus 5 days (which is Tuesday). Hence $DayN$ should be 2. According to the formula, when today is Thursday, $DayT = 4$, and so when $N = 12$,
$$
\begin{array}{rcl}
DayN &=& (DayT + N) \mod 7 \\
&=& (4 + 12) \mod 7 \\
&=& 16 \mod 7 \\
&=& 2,
\end{array}
$$
which agrees.
\end{proof}

\subsection{Exercise 12}
Justify formula (4.5.1) for general values of $DayT$ and $N$.

Formula 4.5.1 says: if $DayT$ is the day of the week today and $DayN$ is the day of the week in $N$ days, then $DayN = (DayT + N) \mod 7$, where Sunday = 0, Monday = 1, $\ldots$, Saturday = 6.

\begin{proof}
By the quotient-remainder theorem, $N = 7q + r$ for some integers $q$ and $r$ where $0 \leq r < 7$.

If we start counting from Monday (assuming that today is Monday), then $N$ days later, the day of the week will be $r$ days after Monday.

If, instead, we start counting from another day of the week (assume today is $DayT$ days after Monday, then $N$ days later, the day of the week will be $r+DayT$ days away from Monday.

Since $0 \leq r \leq 6$ and $0 \leq DayT \leq 6$, we have $0 \leq r + DayT \leq 12$. So, to find the day of the week that is $r+DayT$ days after Monday, we have to take its remainder mod 7. (If $r+DayT \geq 7$ then we would have to subtract 7.)

Therefore $DayN = (r+DayT) \mod 7 = (N + DayT) \mod 7$ (because $N = r \mod 7$).
\end{proof}

\subsection{Exercise 13}
On a Monday a friend says he will meet you again in 30 days. What day of the week will that be?

\begin{proof}
Solution 1: $30 = 4\cdot 7 + 2$. Hence the answer is two days after Monday, which is Wednesday.

Solution 2: By the formula, the answer is $(1 + 30) \mod
7 = 31 \mod 7 = 3$, which is Wednesday.
\end{proof}

\subsection{Exercise 14}
If today is Tuesday, what day of the week will it be 1,000 days from today?

{\it Hint:} There are two ways to solve this problem. One is to find that $1,000 = 7\cdot 142 + 6$ and note that if today is Tuesday, then 1,000 days from today is 142 weeks plus 6 days from today. 

The other way is to use the formula $DayN = (DayT + N) \mod 7$, with $DayT = 2$ (Tuesday) and $N = 1,000$.

\begin{proof}
For Tuesday, $DayT = 2$. According to the formula, when $N  = 1000$, $DayN = (DayT + N) \mod 7 = 1002 \mod 7 = 1$, which is Monday.
\end{proof}

\subsection{Exercise 15}
January 1, 2000, was a Saturday, and 2000 was a leap year. What day of the week will January 1, 2050, be?

\begin{proof}
There are 13 leap years 2000-2050: 2000, 2004, 2008, $\ldots$, 2044, 2048. These 13 years contribute 366 days, one more than the usual 365. There are 50 years. So the total number of days between January 1, 2000 and January 1, 2050 is $365 \cdot 50 + 13 = 18263$.

By the formula, where $DayT = 6$ for Saturday and $N = 18263$, we have $DayN = (DayT + N) \mod 7 = (6 + 18263) \mod 7$ $= 18269 \mod 7 = 6$ so it will be Saturday.

(Another way to approach this problem is as follows: there are 52 weeks in every year, which add up to $52 \cdot 7 = 364$ days, which is 1 less than 365. So every non-leap year advances the day of the week by only 1 day, and every leap year advances the day of the week by 2 days, because they have 366 days.

There are 37 non-leap years, which advance the day of the week by $37 \cdot 1 = 37$ days, and there are 13 leap years which advance the day of the week by $13 \cdot 2 = 26$ days. So the 50 years will advance the day of the week by $37 + 26 = 63$ days. Then we can use the formula with $N = 63$ instead: $DayN = (DayT + 63) \mod 7 = (6 + 63) \mod 7 = 6$, which is Saturday, as before.)
\end{proof}

\subsection{Exercise 16}
Suppose $d$ is a positive and $n$ is any integer. If $d \mid n$, what is the remainder obtained when the quotient-remainder theorem is applied to $n$ with divisor $d$?

\begin{proof}
Because $d\mid n$, $n = dq + 0$ for some integer $q$. Thus the remainder is 0.
\end{proof}

\subsection{Exercise 17}
Prove directly from the definitions that for every integer $n$, $n^2-n + 3$ is odd. Use division into two cases: $n$ is even and $n$ is odd.

\begin{proof}
Assume $n$ is any integer. {\it [Want to prove $n^2-n + 3$ is odd.]}

{\bf Case 1:} $n$ is even.

By definition of even, $n = 2k$ for some integer $k$. 

Then $n^2-n + 3 = (2k)^2 - 2k + 3 = 4k^2 - 2k + 2 + 1 = 2(2k^2 - k + 1) + 1$. 

Let $t = 2k^2 - k + 1$, which is an integer because it is a sum and product of integers. So $n^2 - n + 3 = 2t+1$ where $t$ is an integer. So by definition of odd, $n^2 - n + 3$ is odd.

{\bf Case 2:} $n$ is odd.

By definition of even, $n = 2k + 1$ for some integer $k$. 

Then $n^2-n + 3 = (2k+1)^2 - 2(k+1) + 3 = 4k^2 + 4k + 4 - 2k - 2 + 3 = 4k^2 + 2k + 5 = 2(2k^2 + k + 2) + 1$. 

Let $t = 2k^2 + k + 2$, which is an integer because it is a sum and product of integers. So $n^2 - n + 3 = 2t+1$ where $t$ is an integer. So by definition of odd, $n^2 - n + 3$ is odd.
\end{proof}

\subsection{Exercise 18}

\subsubsection{(a)}
Prove that the product of any two consecutive integers is even.

\begin{proof}
Assume $a,b$ are any two consecutive integers. By definition of consecutive, $a = n, b = n+1$ for some integer $n$. Then $ab = n(n+1)$.

{\bf Case 1:} $n$ is even.

By definition of even, $n = 2k$ for some integer $k$. Then $ab = n(n+1) = (2k)(2k+1) = 2(k^2+k)$, where $k^2+k$ is an integer (sum and product of integers). So by definition of even, $ab$ is even.

{\bf Case 2:} $n$ is odd.

By definition of odd, $n = 2k+1$ for some integer $k$. Then $ab = n(n+1) = (2k+1)(2k+2) = (2k+1)(2(k+1)) = 2(2k+1)(k+1) = 2(2k^2 + 3k + 1)$, where $2k^2 + 3k + 1$ is an integer (sum and product of integers). So by definition of even, $ab$ is even.
\end{proof}

\subsubsection{(b)}
The result of part (a) suggests that the second approach in the discussion of Example 4.5.7 might be possible after all. Write a new proof of Theorem 4.5.3 based on this observation.

{\bf Theorem 4.5.3} says: ``The square of any odd integer has the form $8m + 1$ for some integer $m$.''

\begin{proof}
Suppose $n$ is any odd integer. By definition of odd, $n = 2q + 1$ for some integer $q$. Then $n^2 = (2q + 1)^2 = 4q^2 + 4q + 1 = 4(q^2 + q) + 1 = 4q(q + 1) + 1$. By the result of part (a), the product $q(q + 1)$ is even, so $q(q + 1) = 2m$ for some integer $m$. Then, by substitution, $n^2 = 4\cdot2m + 1 = 8m + 1$.
\end{proof}

\subsection{Exercise 19}
Prove directly from the definitions that for all integers $m$ and $n$, if $m$ and $n$ have the same parity, then $5m + 7n$ is even. Divide into two cases: $m$ and $n$ are both even and $m$ and $n$ are both odd.

\begin{proof}
Assume $m, n$ are any two integers that have the same parity. {\it [Want to prove $5m+7n$ is even.]}

{\bf Case 1:} $m,n$ are both even.

By definition of even, $m = 2r, n = 2s$ for some integers $r, s$.

Then $5m+7n = 5(2r) + 7(2s) = 2(5r + 7s)$ where $5r + 7s$ is an integer (sum and product of integers). So by definition of even, $5m+7n$ is even.

{\bf Case 2:} $m,n$ are both odd.

By definition of even, $m = 2r+1, n = 2s+1$ for some integers $r, s$.

Then $5m+7n = 5(2r+1) + 7(2s+1) = 10r+5+14s+7 = 2(5r+7s+6)$ where $5r+7s+6$ is an integer (sum and product of integers). So by definition of even, $5m+7n$ is even.
\end{proof}

\subsection{Exercise 20}
Suppose $a$ is any integer. If $a \mod 7 = 4$, what is $5a \mod 7$? In other words, if division of $a$ by $7$ gives a remainder of 4, what is the remainder when $5a$ is divided by 7? Your solution should show that you obtain the same answer no matter what integer you start with.

\begin{proof}
Because $a \mod 7 = 4$, the remainder obtained when $a$ is divided by 7 is 4, and so $a = 7q + 4$ for some integer $q$. Multiplying this equation through by 5 gives that $5a = 35q + 20 = 35q + 14 + 6 = 7(5q + 2) + 6$. Because $q$ is an integer, $5q + 2$ is also an integer, and so $5a = 7\cdot$(an integer) $+ 6$. Thus, because $0 \leq 6 < 7$, the remainder obtained when $5a$ is divided by 7 is 6, and so $5a \mod 7 = 6$.
\end{proof}

\subsection{Exercise 21}
Suppose $b$ is any integer. If $b \mod 12 = 5$, what is $8b \mod 12$? In other words, if division of $b$ by 12 gives a remainder of 5, what is the remainder when $8b$ is divided by 12? Your solution should show that you obtain the same answer no matter what integer you start with.

\begin{proof}
Because $b \mod 12 = 5$, the remainder obtained when $b$ is divided by 12 is 5, and so $b = 12q + 5$ for some integer $q$. Multiplying this equation through by 8 gives that $8b = 96q + 40 = 96q + 36 + 4 = 12(8q + 3) + 4$. Because $q$ is an integer, $8q + 3$ is also an integer, and so $8b = 12\cdot$(an integer) $+ 4$. Thus, because $0 \leq 4 < 12$, the remainder obtained when $8b$ is divided by 12 is 4, and so $8b \mod 12 = 4$.
\end{proof}

\subsection{Exercise 22}
Suppose $c$ is any integer. If $c \mod 15 = 3$, what is $10c \mod 15$? In other words, if division of $c$ by 15 gives a remainder of 3, what is the remainder when $10c$ is divided by 15? Your solution should show that you obtain the same answer no matter what integer you start with.

\begin{proof}
Because $c \mod 15 = 3$, the remainder obtained when $c$ is divided by 15 is 3, and so $c = 15q + 3$ for some integer $q$. Multiplying this equation through by 10 gives that $10c = 150q + 30 = 15(10q + 2) + 0$. Because $q$ is an integer, $10q + 2$ is also an integer, and so $10c = 15\cdot$(an integer) $+ 0$. Thus, because $0 \leq 0 < 15$, the remainder obtained when $10c$ is divided by 15 is 0, and so $10c \mod 15 = 0$.
\end{proof}

\subsection{Exercise 23}
Prove that for every integer $n$, if $n \mod 5 = 3$ then
$n^2 \mod 5 = 4$.

\begin{proof}
Suppose $n$ is any {\it [particular but arbitrarily chosen]} integer such that $n \mod 5 = 3$. Then the remainder obtained when $n$ is divided by 5 is 3, and so $n = 5q + 3$ for some integer $q$. By substitution, 

$n^2 = (5q + 3)^2 = 25q^2 + 30q + 9 = 25q^2 + 30q + 5 + 4 = 5(5q^2 + 6q + 1) + 4$.

Because products and sums of integers are integers, $5q^2 + 6q + 1$ is an integer, and hence $n^2 = 5\cdot$ (an integer) $+ 4$. Thus, since $0 \leq 4 < 5$, the remainder obtained when $n^2$ is divided by 5 is 4, and so $n^2 \mod 5 = 4$.
\end{proof}

\subsection{Exercise 24}
Prove that for all integers $m$ and $n$, if $m \mod 5 = 2$
and $n \mod 5 = 1$ then $mn \mod 5 = 2$.

\begin{proof}
Suppose $m,n$ are any {\it [particular but arbitrarily chosen]} integers such that $m \mod 5 = 2$ and $n \mod 5 = 1$. 

Then $m = 5q + 2$ and $n = 5s + 1$ for some integers $q,s$. By substitution, 

$mn = (5q + 2)(5s+1) = 25qs + 5q + 10s + 2 = 5(5qs + q + 2s) + 2$.

Because products and sums of integers are integers, $5qs + q + 2s$ is an integer, and hence $mn = 5\cdot$ (an integer) $+ 2$. Thus, since $0 \leq 2 < 5$, the remainder obtained when $mn$ is divided by 5 is 2, and so $mn \mod 5 = 2$.
\end{proof}

\subsection{Exercise 25}
Prove that for all integers $a$ and $b$, if $a \mod 7 = 5$ and $b \mod 7 = 6$ then $ab \mod 7 = 2$.

\begin{proof}
Suppose $a,b$ are any {\it [particular but arbitrarily chosen]} integers such that $a \mod 7 = 5$ and $b \mod 7 = 6$. 

Then $a = 7q + 5$ and $b = 7s + 6$ for some integers $q,s$. By substitution, 

$ab = (7q + 5)(7s+6) = 49qs + 42q + 35s + 30 = 7(7qs + 6q + 5s + 4) + 2$.

Because products and sums of integers are integers, $7qs + 6q + 5s + 4$ is an integer, and hence $ab = 7\cdot$ (an integer) $+ 2$. Thus, since $0 \leq 2 < 7$, the remainder obtained when $ab$ is divided by 7 is 2, and so $ab \mod 7 = 2$.
\end{proof}

\subsection{Exercise 26}
Prove that a necessary and sufficient condition for an integer $n$ to be divisible by a positive integer $d$ is that $n \mod d = 0$.

{\it Hint:} You need to show that (1) for each integer $n$ and positive integer $d$, if $n$ is divisible by $d$ then $n \mod d = 0$; and (2) for each integer $n$ and positive integer $d$, if $n \mod d = 0$ then $n$ is divisible by $d$.

\begin{proof}
Assume $n$ is any integer and $d$ is any positive integer.

(1) Assume $n$ is divisible by $d$. {\it Want to prove $n \mod d = 0$.}

By definition of divisible, $n = ad$ for some integer $a$. So $n = ad + 0$. This means that dividing $n$ by $d$ results in a remainder of $0$ where $0 \leq 0 < d$. By definition of $\mod$, we have $n \mod d = 0$.

(2) Assume $n \mod d = 0$. {\it Want to prove $n$ is divisible by $d$.}

By definition of $\mod$, $n = qd + 0$ for some integer $q$. So $n = qd$, where $q$ is an integer. By definition of divisible, $n$ is divisible by $d$.
\end{proof}

\subsection{Exercise 27}
Use the quotient-remainder theorem with divisor equal to 2 to prove that the square of any integer can be written in one of the two forms $4k$ or $4k + 1$ for some integer $k$.

{\it Hint:} Given any integer $n$, by the quotient-remainder theorem with divisor equal to 2, $n = 2q$, or $n = 2q + 1$ for some integer $q$.

\begin{proof}
Assume $n$ is any integer. {\it [Want to prove $n^2 = 4k$ or $n^2 = 4k+1$ for some integer $k$.]}

By the quotient-remainder theorem $n = 2q+r$ for some integers $q, r$ where $0 \leq r < 2$.

{\bf Case 1:} $r = 0$.

Then $n = 2q$, so $n^2 = (2q)^2 = 4q^2$. Let $k = q^2$ which is an integer (square of an integer). So $n^2 = 4k$ for some integer $k$.

{\bf Case 2:} $r = 1$.

Then $n = 2q+1$, so $n^2 = (2q+1)^2 = 4q^2+4q+1 = 4(q^2+q)+1$. Let $k = q^2+q$ which is an integer (sum and product of integers). So $n^2 = 4k+1$ for some integer $k$.

Since these two cases exhaust all the possibilities, we proved what was to be shown.
\end{proof}

\subsection{Exercise 28}

\subsubsection{(a)}
Prove: Given any set of three consecutive integers, one of the integers is a multiple of 3. 

\begin{proof}
(using the {\it Hint}:)

Suppose that $n, n + 1$, and $n + 2$ are any three consecutive integers. Then by the quotient-remainder theorem $n = 3q+r$ for some integers $q,r$ where $0 \leq r < 3$.

There are three cases:

{\bf Case 1 ($n = 3q$ for some integer $q$).} In this case $n = 3q$ is a multiple of 3.

{\bf Case 2 ($n = 3q + 1$ for some integer $q$).} In this case $n + 2 = (3q+1)+2 = 3q+3 = 3(q+1)$ is a multiple of 3 (because $q+1$ is an integer).

{\bf Case 3 ($n = 3q + 2$ for some integer $q$).} In this case $n + 1 = (3q + 2) + 1 = 3q+3 = 3(q+1)$ is a multiple of 3 (because $q+1$ is an integer).

Since these cases exhaust all the possibilities, we conclude that in all possible cases one of the integers is a multiple of 3.
\end{proof}

\subsubsection{(b)}
Use the result of part (a) to prove that any product of three consecutive integers is a multiple of 3.

\begin{proof}
Assume $n, n+1, n+2$ are any three consecutive integers. By part (a) one of them is a multiple of 3. There are 3 cases:

{\bf Case 1 ($n$) is a multiple of 3.} Then $n = 3k$ for some integer $k$. Then $n(n+1)(n+2) = (3k)(n+1)(n+2) = 3[k(n+1)(n+2)]$. Let $t = k(n+1)(n+2)$ which is a product of integers, therefore an integer. So $n(n+1)(n+2) = 3t$ for some integer $t$, so $n(n+1)(n+2)$ is a multiple of 3.

{\bf Case 2 ($n+1$) is a multiple of 3.} Then $n+1 = 3k$ for some integer $k$. Then $n(n+1)(n+2) = n(3k)(n+2) = 3[kn(n+2)]$. Let $t = kn(n+2)$ which is a product of integers, therefore an integer. So $n(n+1)(n+2) = 3t$ for some integer $t$, so $n(n+1)(n+2)$ is a multiple of 3.

{\bf Case 3 ($n+2$) is a multiple of 3.} Then $n+2 = 3k$ for some integer $k$. Then $n(n+1)(n+2) = n(n+1)(3k) = 3[kn(n+1)]$. Let $t = kn(n+1)$ which is a product of integers, therefore an integer. So $n(n+1)(n+2) = 3t$ for some integer $t$, so $n(n+1)(n+2)$ is a multiple of 3.

Since these cases exhaust all possibilities, in all possible cases their product is a multiple of 3.
\end{proof}

\subsection{Exercise 29}

\subsubsection{(a)}
Use the quotient-remainder theorem with divisor equal to 3 to prove that the square of any integer has the form $3k$ or $3k + 1$ for some integer $k$.

\begin{proof}
(using the {\it Hint}) 

Assume $n$ is any integer. By the quotient-remainder theorem $n = 3q+r$ for some integers $q,r$ with $0 \leq r < 3$. There are 3 cases depending on the value of $r$:

{\bf Case 1:} $n = 3q$.

Then $n^2 = (3q)^2 = 9q^2 = 3(3q^2)$. Let $k = 3q^2$, which is an integer since it's a product of integers. Therefore $n = 3k$ for some integer $k$.

{\bf Case 2:} $n = 3q + 1$.

Then $n^2 = (3q+1)^2 = 9q^2 + 6q + 1 = 3(3q^2 + 2q) + 1$. Let $k = 3q^2 + 2q$, which is an integer since it's a sum and product of integers. Therefore $n = 3k + 1$ for some integer $k$.

{\bf Case 3:} $n = 3q + 2$

Then $n^2 = (3q+2)^2 = 9q^2 + 12q + 4 = 3(3q^2 + 4q + 1) + 1$. Let $k = 3q^2 + 4q+1$, which is an integer since it's a sum and product of integers. Therefore $n = 3k + 1$ for some integer $k$.

Since these exhaust all the possibilities, in all cases $n = 3k$ or $n = 3k+1$ for some integer $k$.
\end{proof}

\subsubsection{(b)}
Use the mod notation to rewrite the result of part (a).

\begin{proof}
The square of any integer $\mod 3$ is either 0 or 1.

If $n$ is any integer, then $n^2 = 0 \mod 3$ or $n^2 = 1 \mod 3$.
\end{proof}

\subsection{Exercise 30}

\subsubsection{(a)}
Use the quotient-remainder theorem with divisor equal to 3 to prove that the product of any two consecutive integers has the form $3k$ or $3k + 2$ for some integer $k$.

\begin{proof}
Assume $n$ and $n+1$ are any two consecutive integers. By the quotient-remainder theorem $n = 3q+r$ for some integers $q,r$ with $0 \leq r < 3$. There are 3 cases depending on the value of $r$:

{\bf Case 1:} $n = 3q$.

Then $n(n+1) = (3q)(3q+1) = 9q^2 + 3q = 3(3q^2 + q)$. Let $k = 3q^2 + q$, which is an integer since it's a sum and product of integers. Therefore $n(n+1) = 3k$ for some integer $k$.

{\bf Case 2:} $n = 3q + 1$.

Then $n(n+1) = (3q+1)(3q+2) = 9q^2 + 9q + 2 = 3(3q^2 + 3q) + 2$. Let $k = 3q^2 + 3q$, which is an integer since it's a sum and product of integers. Therefore $n = 3k + 2$ for some integer $k$.

{\bf Case 3:} $n = 3q + 2$

Then $n(n+1) = (3q+2)(3q+3) = 9q^2 + 15q + 6 = 3(3q^2 + 5q + 2)$. Let $k = 3q^2 + 5q + 2$, which is an integer since it's a sum and product of integers. Therefore $n = 3k$ for some integer $k$.

Since these exhaust all the possibilities, in all cases $n(n+1) = 3k$ or $n(n+1) = 3k+2$ for some integer $k$.
\end{proof}

\subsubsection{(b)}
Use the$ $mod$ $notation to rewrite the result of part (a).

\begin{proof}
For any integer $n$, $n(n+1) = 0 \mod 3$ or $n(n+1) = 2 \mod 3$.
\end{proof}

{\bf \cy In $31-33$, you may use the properties listed in example 4.3.3.}

\subsection{Exercise 31}

\subsubsection{(a)}
Prove that for all integers $m$ and $n$, $m + n$ and $m - n$ are either both odd or both even.

\begin{proof}
Assume $m,n$ are any two integers. There are 4 cases.

{\bf Case 1.} $m,n$ are both odd.

Then $m = 2r+1, n = 2s+1$ for some integers $r,s$. 

So $m+n = 2r + 2s + 2 = 2(r+s+1)$ is even, and $m-n = 2r - 2s = 2(r-s)$ is even.

{\bf Case 2.} $m,n$ are both even.

Then $m = 2r, n = 2s$ for some integers $r,s$.

So $m+n = 2r + 2s  = 2(r+s)$ is even, and $m-n = 2r - 2s = 2(r-s)$ is even.

{\bf Case 3.} $m$ is odd, $n$ is even.

Then $m = 2r+1, n = 2s$ for some integers $r,s$.

So $m+n = 2r + 2s + 1 = 2(r+s) +1$ is odd, and $m-n = 2r - 2s + 1 = 2(r-s) + 1$ is odd.

{\bf Case 4.} $m$ is even, $n$ is odd.

Then $m = 2r, n = 2s+1$ for some integers $r,s$.

So $m+n = 2r + 2s + 1 = 2(r+s)+1$ is even, and $m-n = 2r - 2s - 1 = 2r - 2s - 2 + 1 = 2(r-s - 1) + 1$ is odd.

Since these cases exhaust all possibilities, we proved that $m+n$ and $m-n$ are either both odd or both even for all integers $m,n$.
\end{proof}

\subsubsection{(b)}
Find all solutions to the equation $m^2 - n^2 = 56$ for which both $m$ and $n$ are positive integers.

\begin{proof}
If $m^2 - n^2 = 56$, then $56 = (m + n)(m - n)$. Now $56 = 2^3 \cdot 7$, and by the unique factorization theorem, this factorization is unique. Hence the only representation of 56 as a product of two positive integers are $56 = 7\cdot8 = 14\cdot4 = 28\cdot2 = 56\cdot1$. By part (a), $m$ and $n$ must both be odd or both be even. Thus the only solutions are either $m + n = 14$ and $m - n = 4$ or $m + n = 28$ and $m - n = 2$. It follows that the only solutions are either $m = 9$ and $n = 5$ or $m = 15$ and $n = 13$.
\end{proof}

\subsubsection{(c)}
Find all solutions to the equation $m^2 - n^2 = 88$ for which both $m$ and $n$ are positive integers.

\begin{proof}
If $m^2 - n^2 = 88$, then $88 = (m + n)(m - n)$. Now $88 = 2^3 \cdot 11$, and by the unique factorization theorem, this factorization is unique. Hence the only representation of 88 as a product of two positive integers are $88 = 11\cdot8 = 22\cdot4 = 44\cdot2 = 88\cdot1$. By part (a), $m$ and $n$ must both be odd or both be even. Thus the only solutions are either $m + n = 22$ and $m - n = 4$ or $m + n = 44$ and $m - n = 2$. It follows that the only solutions are either $m = 13$ and $n = 9$ or $m = 23$ and $n = 21$.
\end{proof}

\subsection{Exercise 32}
Given any integers $a, b$, and $c$, if $a - b$ is even and $b - c$ is even, what can you say about the parity of $2a - (b + c)$? Prove your answer.

\begin{proof}
Under the given conditions, $2a - (b + c)$ is even.

Suppose $a, b$, and $c$ are any integers such that $a - b$
is even and $b - c$ is even. [We must show that $2a - (b + c)$ is even.] 

Note first that $2a - (b + c) = (a - b) + (a - c)$. Also note that $(a - b) + (b - c)$ is a sum of two even integers and hence is even by Example 4.3.3 \#1. But $(a - b) + (b - c) = a - c$, and so $a - c$ is even. Hence $2a - (b + c)$ is a sum of two even integers, and thus it is even {\it [as was to be shown].}
\end{proof}

\subsection{Exercise 33}
Given any integers $a, b$, and $c$, if $a - b$ is odd and
$b - c$ is even, what can you say about the parity of
$a - c$? Prove your answer.

\begin{proof}
Under the given conditions, $a - c$ is even.

Suppose $a, b$, and $c$ are any integers such that $a - b$
is odd and $b - c$ is even. [We must show that $a - c$ is even.] 

Note first that $a - c = (a - b) + (b - c)$. Also note that $(a - b) + (b - c)$ is a sum of an odd integer and an even integer and hence is odd by Example 4.3.3 \#5. Hence $a - c$ is odd {\it [as was to be shown].}
\end{proof}

\subsection{Exercise 34}
Given any integer $n$, if $n > 3$, could $n, n + 2$, and
$n + 4$ all be prime? Prove or give a counterexample.

{\it Hint:} Express $n$ using the quotient-remainder theorem with $d = 3$.

\begin{proof}
Assume $n>3$ is any integer. By the quotient-remainder theorem, $n = 3q+r$ for some integers $q,r$ where $0 \leq r < 3$.

{\bf Case 1.} $r = 0$.

Then $n = 3q$ is not prime because it is divisible by 3.

{\bf Case 2.} $r = 1$.

Then $n+2 = (3q+1)+2 = 3q+3 = 3(q+1)$ is not prime because it is divisible by 3.

{\bf Case 3.} $r = 2$.

Then $n+4 = (3q+2)+4 = 3q+6 = 3(q+2)$ is not prime because it is divisible by 3.

Since these cases exhaust all possibilities, it is impossible for $n, n+2, n+4$ to be all prime for any integer $n>3$.
\end{proof}

{\bf \cy Prove each of the statements in $35-43$.}

\subsection{Exercise 35}
The fourth power of any integer has the form $8m$ or $8m + 1$ for some integer $m$.

\begin{proof}
Assume $n$ is any integer. {\it [Want to prove $n^4 = 8m$ or $n^4 = 8m+1$ for some integer $m$.]}

{\bf Case 1.} $n$ is even.

By definition of even, $n = 2k$ for some integer $k$. So $n^4 = (2k)^4 = 16k^4 = 2(8k^4)$. 

Let $m = 8k^4$ which is an integer since it's a product of integers. So $n^4 = 8m$ where $m$ is an integer.

{\bf Case 2.} $n$ is odd.

By definition of odd, $n = 2k+1$ for some integer $k$. So 

$n^4 = (2k+1)^4 = (2k)^4 + 4 \cdot (2k)^3 + 6 \cdot (2k)^2 + 4 \cdot (2k) + 1$ 

$= 16k^4 + 32k^3 + 24k^2 + 8k + 1 = 8(2k^4 + 4k^3 + 3k^2 + k) + 1$.

Let $m = 2k^4 + 4k^3 + 3k^2 + k$ which is an integer since it's a sum and product of integers. So $n^4 = 8m + 1$ where $m$ is an integer.

Since these cases exhaust all possibilities, $n^4 = 8m$ or $n^4 = 8m+1$ for some integer $m$ {\it [as was to be shown.]}
\end{proof}

\subsection{Exercise 36}
The product of any four consecutive integers is divisible by 8.

{\it Hint:} Use the quotient-remainder theorem (as in Example 4.5.6) to say that $n = 4q, n = 4q + 1, n = 4q + 2$, or $n = 4q + 3$ and divide into cases accordingly.

\begin{proof}
Assume $n, n+1, n+2, n+3$ are any four consecutive integers. {\it [Want to prove that their product is equal to $8m$ for some integer $m$.]}

By the quotient-remainder theorem, $n = 4q+r$ for some integers $r,q$ with $0 \leq r < 4$.

{\bf Case 1:} $n = 4q + 0$.

Then $n(n+1)(n+2)(n+3) = 4q(4q+1)(4q+2)(4q+3) = 4q(4q+1)2(2q+1)(4q+3) = 8[q(4q+1)(2q+1)(4q+3)]$.

Let $m = q(4q+1)(2q+1)(4q+3)$, which is an integer because it's a sum and product of integers. Then $n(n+1)(n+2)(n+3) = 8m$ where $m$ is an integer.

{\bf Case 2:} $n = 4q + 1$.

Then $n(n+1)(n+2)(n+3) = (4q+1)(4q+2)(4q+3)(4q+4) = (4q+1)2(2q+1)(4q+3)4(q+1) = 8[(4q+1)(2q+1)(4q+3)(q+1)]$.

Let $m = (4q+1)(2q+1)(4q+3)(q+1)$, which is an integer because it's a sum and product of integers. Then $n(n+1)(n+2)(n+3) = 8m$ where $m$ is an integer.

{\bf Case 3:} $n = 4q + 2$.

Then $n(n+1)(n+2)(n+3) = (4q+2)(4q+3)(4q+4)(4q+5) = 2(2q+1)(4q+3)4(q+1)(4q+5) = 8[(2q+1)(4q+3)(q+1)(4q+5)]$.

Let $m = (2q+1)(4q+3)(q+1)(4q+5)$, which is an integer because it's a sum and product of integers. Then $n(n+1)(n+2)(n+3) = 8m$ where $m$ is an integer.

{\bf Case 4:} $n = 4q + 3$.

Then $n(n+1)(n+2)(n+3) = (4q+3)(4q+4)(4q+5)(4q+6) = (4q+3)4(q+1)(4q+5)2(2q+3) = 8[(4q+3)(q+1)(4q+5)(2q+3)]$.

Let $m = (4q+3)(q+1)(4q+5)(2q+3)$, which is an integer because it's a sum and product of integers. Then $n(n+1)(n+2)(n+3) = 8m$ where $m$ is an integer.

Since these cases exhaust all possibilities, we have proved what was to be shown.
\end{proof}

\subsection{Exercise 37}
For any integer $n$, $n^2 + 5$ is not divisible by 4.

{\it Hint:} Given any integer $n$, consider the two cases where $n$ is even and where $n$ is odd.

\begin{proof}
Assume $n$ is any integer. {\it [Want to prove $n^2+5$ is not divisible by 4.]}

{\bf Case 1.} $n$ is even. Then $n = 2k$ for some integer $k$.

Then $n^2+5 = (2k)^2+5 = 4k^2+5 = 4(k^2+1)+1$ where $k^2+1$ is an integer. Therefore $n \mod 4 = 1$ which means $4 \nmid n$ by Exercise 26.

{\bf Case 2.} $n$ is odd. Then $n = 2k+1$ for some integer $k$.

Then $n^2+5 = (2k+1)^2+5 = 4k^2+4k+1+5 = 4(k^2+k+1)+2$ where $k^2+k+1$ is an integer. Therefore $n \mod 4 = 2$ which means $4 \nmid n$ by Exercise 26.

Since these cases exhaust all possibilities, we have proved that $n^2+5$ is not divisible by 4.
\end{proof}

\subsection{Exercise 38}
For every integer $m$, $m^2 = 5k$, or $m^2 = 5k + 1$, or $m^2 = 5k + 4$ for some integer $k$.

\begin{proof}
Assume $m$ is any integer. By the quotient-remainder theorem $m = 5q+r$ for some integers $q,r$ where $0 \leq r < 5$. There are 5 cases:

{\bf Case 1:} $r = 0$. So $m = 5q + 0$ and $m^2 = (5q+0)^2 = 25q^2$.

Then $m^2 = 25q^2 = 5(5q^2)$.

Let $k = 5q^2$ which is an integer because it's a product of integers. So $m^2 = 5k$ where $k$ is an integer.

{\bf Case 2:} $r = 1$. So $m = 5q + 1$ and $m^2 = (5q + 1)^2 = 25q^2 + 10q + 1$.

Then $m^2 = 25q^2 + 10q + 1 = 5(5q^2+2q)+1$.

Let $k = 5q^2+2q$ which is an integer because it's a sum and product of integers. So $m^2 = 5k + 1$ where $k$ is an integer.

{\bf Case 3:} $r = 2$. So $m = 5q + 2$ and $m^2 = (5q + 2)^2 = 25q^2 + 20q + 4$.

Then $m^2 = 25q^2 + 20q + 4 = 5(5q^2+4q)+4$.

Let $k = 5q^2+4q$ which is an integer because it's a sum and product of integers. So $m^2 = 5k +4$ where $k$ is an integer.

{\bf Case 4:} $r = 3$. So $m = 5q + 3$ and $m^2 = (5q + 3)^2 = 25q^2 + 30q+9$.

Then $m^2 = 25q^2 + 30q+9 = 5(5q^2+6q+1)+4$.

Let $k = 5q^2+6q+1$ which is an integer because it's a sum and product of integers. So $m^2 = 5k + 4$ where $k$ is an integer.

{\bf Case 5:} $r = 4$. So $m = 5q + 4$ and $m^2 = (5q + 4)^2 = 25q^2 + 40q + 16$.

Then $m^2 = 25q^2 + 40q + 16 = 5(5q^2+8q+3)+1$.

Let $k = 5q^2+8q+3$ which is an integer because it's a sum and product of integers. So $m^2 = 5k + 1$ where $k$ is an integer.

Since these cases exhaust all possibilities, we have proved for any integer $m$, $m^2 = 5k$ or $m^2 = 5k+1$ or $m^2 = 5k+4$ for some integer $k$.
\end{proof}

\subsection{Exercise 39}
Every prime number except 2 and 3 has the form $6q + 1$ or $6q + 5$ for some integer $q$.

{\it Hint:} Use the quotient-remainder theorem to say that $p$ must have one of the forms $6q, 6q + 1, 6q + 2, 6q + 3, 6q + 4$, or $6q + 5$ for some integer $q$. Then use the fact that $p$ is prime and not equal to either 2 or 3 to show that you only need to consider two cases.

\begin{proof}
Assume $p$ is any prime number except 2 and 3. By the quotient-remainder theorem $p = 6q+r$ for some integers, where $0 \leq r < 6$. There are 6 cases.

{\bf Case 1:} $r = 0$. Then $p = 6q$ is divisible by $6$, contradicting the fact that $p$ is prime. So this case is impossible.

{\bf Case 2:} $r = 1$. Then $p = 6q+1$ for some integer $q$ as was to be shown.

{\bf Case 3:} $r = 2$. Then $p = 6q+2 = 2(3q+1)$ is divisible by $2$, contradicting the fact that $p$ is prime. So this case is impossible.

{\bf Case 4:} $r = 3$. Then $p = 6q+3 = 3(2q+1)$ is divisible by $3$, contradicting the fact that $p$ is prime. So this case is impossible.

{\bf Case 5:} $r = 4$. Then $p = 6q+4 = 2(3q+2)$ is divisible by $2$, contradicting the fact that $p$ is prime. So this case is impossible.

{\bf Case 6:} $r = 5$. Then $p = 6q+5$ for some integer $q$ as was to be shown.

Since these cases exhaust all possibilities, we proved that $p = 6q+1$ or $p = 6q+5$ for some integer $q$.
\end{proof}

\subsection{Exercise 40}
If $n$ is any odd integer, then $n^4 \mod 16 = 1$.

\begin{proof}
Assume $n$ is any odd integer. By definition of odd, $n = 2k+1$ for some integer $k$.

Then $n^4 = (2k+1)^4 = (2k)^4 + 4(2k)^3 + 6(2k)^2 + 4(2k) + 1 = 16k^4 + 32k^3 + 24k^2 + 8k + 1 = 16k^4 + 32k^3 + 16k^2 + 8k^2 + 8k + 1 = 16(k^4 + 2k^3+ k^2) + 8k^2 + 8k + 1$

$= 16(k^4 + 2k^3+ k^2) + 8k(k + 1) + 1$.

By an earlier Exercise, the product of any two consecutive integers is even. So $k(k+1)$ is even. So $k(k+1) = 2s$ for some integer $s$. Then

$n^4 = 16(k^4 + 2k^3+ k^2) + 8(2s) + 1 = 16(k^4 + 2k^3+ k^2 + s) + 1$.

Let $t = k^4 + 2k^3+ k^2 + s$ which is an integer because it's a sum and product of integers. Therefore $n^4 = 16t + 1$ for some integer $t$. Since $0 \leq 1 < 16$, the remainder of dividing $n^4$ by 16 is 1. So by definition of mod, $n^4 \mod = 1$.
\end{proof}

\subsection{Exercise 41}
For all real numbers $x$ and $y$, $|x| \cdot |y| = |xy|$.

\begin{proof}
Assume $x,y$ are any two real numbers. There are 2 cases.

{\bf Case 1:} $x \geq 0$. In this case $|x| = x$ by definition of absolute value. There are 2 subcases.

{\bf Subcase 1.1:} $y \geq 0$. In this subcase $|y| = y$ by definition of absolute value. Moreover notice that $xy \geq 0$ therefore $|xy| = xy$ by definition of absolute value. So $|x||y| = xy = |xy|$.

{\bf Subcase 1.2:} $y < 0$. In this subcase $|y| = -y$ by definition of absolute value. Moreover notice that $xy \leq 0$ therefore $|xy| = -xy$ by definition of absolute value. So $|x||y| = x(-y) = -xy = |xy|$.

{\bf Case 2:} $x < 0$. In this case $|x| = -x$ by definition of absolute value. There are 2 subcases.

{\bf Subcase 2.1:} $y \geq 0$. In this subcase $|y| = y$ by definition of absolute value. Moreover notice that $xy \leq 0$ therefore $|xy| = -xy$ by definition of absolute value. So $|x||y| = (-x)y = -xy = |xy|$.

{\bf Subcase 2.2:} $y < 0$. In this subcase $|y| = -y$ by definition of absolute value. Moreover notice that $xy \geq 0$ therefore $|xy| = xy$ by definition of absolute value. So $|x||y| = (-x)(-y) = xy = |xy|$.

In all cases we showed $|x||y| = |xy|$. Since these cases exhaust all possibilities, $|x||y| = |xy|$ for all real numbers $x,y$.
\end{proof}

\subsection{Exercise 42}
For all real numbers $r$ and $c$ with $c \geq 0$, $-c \leq r \leq c$ if, and only if, $|r| \leq c$. (Hint: Proving $A$ if, and only if, $B$ requires proving both if $A$ then $B$ and if $B$ then $A$.)

\begin{proof}
Assume $-c \leq r \leq c$. We want to prove $|r| \leq c$. There are two cases:

{\bf Case 1.} $r \geq 0$. Then $|r| = r$. Since we know $r \leq c$, we have by substitution $|r| \leq c$.

{\bf Case 2.} $r < 0$. Then $|r| = -r$. Since we know $-c \leq r$, multiplying this inequality by $-1$ we get $c \geq -r$, in other words $-r \leq c$. So we have by substitution $|r| \leq c$.

Now we want to prove the converse.

Assume $|r| \leq c$. We want to prove $-c \leq r \leq c$.

{\bf Case 1.} $r \geq 0$. Then $|r| = r$. Since we know $|r| \leq c$, by substitution we get $r \leq c$. 

We know $c \geq 0$ so $-c \leq 0$. We also know $0 \leq r$, so combining these we get $-c \leq 0 \leq r$, and by transitivity $-c \leq r$.

Combining these two results we get $-c \leq r \leq c$.

{\bf Case 2.} $r < 0$. Then $|r| = -r$.  Since we know $|r| \leq c$, by substitution we get $-r \leq c$. Multiplying this inequality by $-1$ gives us $r \geq -c$, in other words $-c \leq r$.

We know $c \geq 0$ in other words $0 \leq c$. We also know $r < 0$, so combining these we get $r < 0 \leq c$, and by transitivity $r \leq c$.

Combining these two results we get $-c \leq r \leq c$.
\end{proof}

\subsection{Exercise 43}
For all real numbers $a$ and $b$, $||a| - |b|| \leq |a - b|$.

{\it Hint:} Apply the triangle inequality with $x = a - b$ and $y = b$ and with $x = b - a$ and $y = a$. Then use the result of exercise 42.

\begin{proof}
(using the Hint)

Let $x = a-b, y = b$. By triangle inequality $|x+y| \leq |x|+ |y|$ we have: 

$|(a-b)+b| = |a| \leq |a-b|+|b|$. Therefore $|a| - |b| \leq |a-b|$.

Let $x = b-a, y = a$. By triangle inequality $|x+y| \leq |x|+ |y|$ we have: 

$|(b-a)+a| = |b| \leq |b-a|+|a|$. Therefore $-|b-a| \leq |a| - |b|$.

Notice $|b-a| = |a-b|$ so we have $-|a-b| \leq |a| - |b|$.

Combining these two results we have $-|a-b| \leq |a| - |b| \leq |a-b|$.

Then by Exercise 42 (where $c = |a-b|$ and $r = |a| - |b|$), we have $||a| - |b|| \leq |a-b|$.
\end{proof}

\subsection{Exercise 44}
A matrix {\bf M} has 3 rows and 4 columns.
$$
\left[
\begin{array}{cccc}
a_{11}&a_{12}&a_{13}&a_{14}\\
a_{21}&a_{22}&a_{23}&a_{24}\\
a_{31}&a_{32}&a_{33}&a_{34}\\
\end{array}
\right]
$$
The 12 entries in the matrix are to be stored in {\it row major} form in locations 7,609 to 7,620 in a computer’s memory. This means that the entries in the first row (reading left to right) are stored first, then the entries in the second row, and finally the entries in the third row.

\subsubsection{(a)}
Which location will $a_{22}$ be stored in?

\begin{proof}
$7,609 + 5 = 7,614$
\end{proof}

\subsubsection{(b)}
Write a formula (in $i$ and $j$) that gives the integer $n$ so that $a_{ij}$ is stored in location $7,609 + n$.

\begin{proof}
$n = 4\cdot (i-1) + j - 1$
\end{proof}

\subsubsection{(c)}
Find formulas (in $n$) for $r$ and $s$ so that $a_{rs}$ is stored in location $7,609 + n$.

\begin{proof}
$r = 1 + (n$ div $4), s = 1 + (n \mod 4)$
\end{proof}

\subsection{Exercise 45}
Let {\bf M} be a matrix with $m$ rows and $n$ columns, and suppose that the entries of {\bf M} are stored in a computer’s memory in row major form (see exercise 44) in locations $N, N + 1, N + 2, \ldots, N + mn - 1$. Find formulas in $k$ for $r$ and $s$ so that $a_{rs}$ is stored in location $N + k$.

\begin{proof}
$r = 1 + (k$ div $n), s = 1 + (k \mod n)$
\end{proof}

\subsection{Exercise 46}
If $m, n$, and $d$ are integers, $d > 0$, and $m \mod d = n \mod d$, does it necessarily follow that $m = n$? That $m - n$ is divisible by $d$? Prove your answers.

\begin{proof}
{\it Answer to first question:} No. Counterexample: Let $m = 1, n = 3$, and $d = 2$. Then $m \mod d = 1$ and $n \mod d = 1$ but $m \neq n$.

{\it Answer to second question:} Yes. Proof: Suppose $m, n$, and $d$ are integers such that $m \mod d = n \mod d$. Let $r = m \mod d = n mod d$. By definition of mod, $m = dp + r$ and $n = dq + r$ for some integers $p$ and $q$. Then $m - n = (dp + r) - (dq + r) = d(p - q)$. But $p - q$ is an integer (being a difference of integers), and so $m - n$ is divisible by $d$ by definition of divisible.
\end{proof}

\subsection{Exercise 47}
If $m$, $n$, and $d$ are integers, $d > 0$, and $d \mid (m - n)$, what is the relation between \\ $m \mod d$ and $n \mod d$? Prove your answer.

\begin{proof}
{\it Answer:} They are equal. Let's prove the answer.

Let $r = m \mod d$ and $s = n \mod d$.

By definition of mod, $m = da+r$ and $n = db+s$ for some integers $a,b$, where $0 \leq r < d$ and $0 \leq s < d$.

Since $d \mid (m - n)$, by definition of divides, $m-n = dq$ for some integer $q$. 

Substituting, we get $m-n = (da+r) - (db+s) = dq$.

So $da+r-db-s-dq = 0$. Organizing, we get $d(a-b-q) + r-s = 0$. So $r-s = d(q-a+b)$. This means $r-s$ is a multiple of $d$.

Since $0 \leq r < d$ and $0 \leq s < d$, the range of possible values of $r-s$ is: $-d+1, -d+2, \ldots, -2, -1, 0, 1, 2, \ldots, d-2, d-1$. Among these values, the only multiple of $d$ is 0. 

Therefore $r-s = 0$. So $r = s$, in other words $m \mod d = n \mod d$.
\end{proof}

\subsection{Exercise 48}
If $m, n, a, b$, and $d$ are integers, $d > 0$, and $m \mod d = a$ and $n \mod d = b$, is $(m + n) \mod d = a + b$? Is $(m + n) \mod d = (a + b) \mod d$? Prove your answers.

\begin{proof}
{\it Answer to the first question:} No. Let $m = 1, n = 2, d = 3, a = 1, b = 2$. Then $1 \mod 3 = 1$ and $2 \mod 3 = 2$ but $(1+2) \mod 3 = 3 \mod 3 = 0 \neq 1+2$.

{\it Answer to the first question:} Yes. Let $r = (a+b) \mod d$. By definition of mod, $m = pd+a$ and $n = qd+b$ and $(a+b) = sd+r$ for some integers $p,q,s$, where $0 \leq a < d$ and $0 \leq b < d$ and $0 \leq r < d$. Then 

$m+n = pd+a+qd+b = d(p+q)+(a+b) = d(p+q) + sd+r = d(p+q+s)+r$.

Let $t = p+q+s$ which is an integer (being a sum of integers). So $m+n = dt + r$ where $t$ is an integer and $0 \leq r < d$. Therefore by definition of mod, $(m+n) \mod d = r$, in other words, $(m + n) \mod d = (a + b) \mod d$.
\end{proof}

\subsection{Exercise 49}
If $m, n, a, b$, and $d$ are integers, $d > 0$, and $m \mod d = a$ and $n \mod d = b$, is $(mn) \mod d = ab$? Is $(mn) \mod d = ab \mod d$? Prove your answers.

\begin{proof}
{\it Answer to the first question:} No. Let $m = 2, n = 3, d = 6, a = 2, b = 3$. Then $2 \mod 6 = 2$ and $3 \mod 6 = 3$ but $(2\cdot3) \mod 6 = 6 \mod 6 = 0 \neq 6 = 2\cdot3$.

{\it Answer to the first question:} Yes. Let $r = ab \mod d$. By definition of mod, $m = pd+a$ and $n = qd+b$ and $(ab) = sd+r$ for some integers $p,q,s$, where $0 \leq a < d$ and $0 \leq b < d$ and $0 \leq r < d$. Then 

$mn = (pd+a)(qd+b) = pqd^2 + pbd + qad + ab = pqd^2 + pbd + qad + (sd + r)$

$ = d(pqd+pb+qa+s) + r$

Let $t = pqd+pb+qa+s$ which is an integer (being a sum and product of integers). So $mn = dt + r$ where $t$ is an integer and $0 \leq r < d$. Therefore by definition of mod, $(mn) \mod d = r$, in other words, $(mn) \mod d = (ab) \mod d$.
\end{proof}

\subsection{Exercise 50}
Prove that if $m, d$, and $k$ are integers and $d > 0$, then $(m + dk) \mod d = m \mod d$.

\begin{proof}
Let $r = m \mod d$. By definition of mod, $m = qd+r$ for some integer $q$ where $0 \leq r < d$. 

Then $m + dk = qd+r+dk = d(q+k)+r$.

Let $t = q+k$ which is an integer (being a sum of integers). Then $m+dk = dt + r$ where $t,r$ are integers with $0 \leq r < d$. So by definition of mod, $(m+dk) \mod d = r$. In other words, $(m + dk) \mod d = m \mod d$.
\end{proof}

\section{Exercise Set 4.6}

\subsection{Exercise 1}

\begin{proof}

\end{proof}

\subsection{Exercise 2}

\begin{proof}

\end{proof}

\subsection{Exercise 3}

\begin{proof}

\end{proof}

\subsection{Exercise 4}

\begin{proof}

\end{proof}

\subsection{Exercise 5}

\begin{proof}

\end{proof}

\subsection{Exercise 6}

\begin{proof}

\end{proof}

\subsection{Exercise 7}

\begin{proof}

\end{proof}

\subsection{Exercise 8}

\begin{proof}

\end{proof}

\subsection{Exercise 9}

\begin{proof}

\end{proof}

\subsection{Exercise 10}

\subsubsection{(a)}

\begin{proof}

\end{proof}

\subsubsection{(b)}

\begin{proof}

\end{proof}

\subsection{Exercise 11}

\begin{proof}

\end{proof}

\subsection{Exercise 12}

\begin{proof}

\end{proof}

\subsection{Exercise 13}

\begin{proof}

\end{proof}

\subsection{Exercise 14}

\begin{proof}

\end{proof}

\subsection{Exercise 15}

\begin{proof}

\end{proof}

\subsection{Exercise 16}

\begin{proof}

\end{proof}

\subsection{Exercise 17}

\begin{proof}

\end{proof}

\subsection{Exercise 18}

\begin{proof}

\end{proof}

\subsection{Exercise 19}

\begin{proof}

\end{proof}

\subsection{Exercise 20}

\begin{proof}

\end{proof}

\subsection{Exercise 21}

\begin{proof}

\end{proof}

\subsection{Exercise 22}

\begin{proof}

\end{proof}

\subsection{Exercise 23}

\begin{proof}

\end{proof}

\subsection{Exercise 24}

\begin{proof}

\end{proof}

\subsection{Exercise 25}

\begin{proof}

\end{proof}

\subsection{Exercise 26}

\begin{proof}

\end{proof}

\subsection{Exercise 27}

\begin{proof}

\end{proof}

\subsection{Exercise 28}

\begin{proof}

\end{proof}

\subsection{Exercise 29}

\begin{proof}

\end{proof}

\subsection{Exercise 30}

\begin{proof}

\end{proof}

\subsection{Exercise 31}

\begin{proof}

\end{proof}

\subsection{Exercise 32}

\begin{proof}

\end{proof}

\subsection{Exercise 33}

\begin{proof}

\end{proof}

\section{Exercise Set 4.7}

\subsection{Exercise 1}

\begin{proof}

\end{proof}

\subsection{Exercise 2}

\begin{proof}

\end{proof}

\subsection{Exercise 3}

\begin{proof}

\end{proof}

\subsection{Exercise 4}

\begin{proof}

\end{proof}

\subsection{Exercise 5}

\begin{proof}

\end{proof}

\subsection{Exercise 6}

\begin{proof}

\end{proof}

\subsection{Exercise 7}

\begin{proof}

\end{proof}

\subsection{Exercise 8}

\begin{proof}

\end{proof}

\subsection{Exercise 9}

\subsubsection{(a)}

\begin{proof}

\end{proof}

\subsubsection{(b)}

\begin{proof}

\end{proof}

\subsection{Exercise 10}

\begin{proof}

\end{proof}

\subsection{Exercise 11}

\begin{proof}

\end{proof}

\subsection{Exercise 12}

\subsubsection{(a)}

\begin{proof}

\end{proof}

\subsubsection{(b)}

\begin{proof}

\end{proof}

\subsection{Exercise 13}

\subsubsection{(a)}

\begin{proof}

\end{proof}

\subsubsection{(b)}

\begin{proof}

\end{proof}

\subsection{Exercise 14}

\subsubsection{(a)}

\begin{proof}

\end{proof}

\subsubsection{(b)}

\begin{proof}

\end{proof}

\subsection{Exercise 15}

\begin{proof}

\end{proof}

\subsection{Exercise 16}

\begin{proof}

\end{proof}

\subsection{Exercise 17}

\begin{proof}

\end{proof}

\subsection{Exercise 18}

\begin{proof}

\end{proof}

\subsection{Exercise 19}

\begin{proof}

\end{proof}

\subsection{Exercise 20}

\begin{proof}

\end{proof}

\subsection{Exercise 21}

\subsubsection{(a)}

\begin{proof}

\end{proof}

\subsubsection{(b)}

\begin{proof}

\end{proof}

\subsection{Exercise 22}

\subsubsection{(a)}

\begin{proof}

\end{proof}

\subsubsection{(b)}

\begin{proof}

\end{proof}

\subsection{Exercise 23}

\begin{proof}

\end{proof}

\subsection{Exercise 24}

\begin{proof}

\end{proof}

\subsection{Exercise 25}

\begin{proof}

\end{proof}

\subsection{Exercise 26}

\begin{proof}

\end{proof}

\subsection{Exercise 27}

\begin{proof}

\end{proof}

\subsection{Exercise 28}

\begin{proof}

\end{proof}

\subsection{Exercise 29}

\begin{proof}

\end{proof}

\subsection{Exercise 30}

\subsubsection{(a)}

\begin{proof}

\end{proof}

\subsubsection{(b)}

\begin{proof}

\end{proof}

\subsection{Exercise 31}

\subsubsection{(a)}

\begin{proof}

\end{proof}

\subsubsection{(b)}

\begin{proof}

\end{proof}

\subsubsection{(c)}

\begin{proof}

\end{proof}

\subsection{Exercise 32}

\begin{proof}

\end{proof}

\subsection{Exercise 33}

\begin{proof}

\end{proof}

\subsection{Exercise 34}

\subsubsection{(a)}

\begin{proof}

\end{proof}

\subsubsection{(b)}

\begin{proof}

\end{proof}

\subsubsection{(c)}

\begin{proof}

\end{proof}

\subsubsection{(d)}

\begin{proof}

\end{proof}

\subsection{Exercise 35}

\begin{proof}

\end{proof}

\subsection{Exercise 36}

\begin{proof}

\end{proof}

\section{Exercise Set 4.8}

\subsection{Exercise 1}

\begin{proof}

\end{proof}

\subsection{Exercise 2}

\begin{proof}

\end{proof}

\subsection{Exercise 3}

\begin{proof}

\end{proof}

\subsection{Exercise 4}

\begin{proof}

\end{proof}

\subsection{Exercise 5}

\begin{proof}

\end{proof}

\subsection{Exercise 6}

\begin{proof}

\end{proof}

\subsection{Exercise 7}

\begin{proof}

\end{proof}

\subsection{Exercise 8}

\begin{proof}

\end{proof}

\subsection{Exercise 9}

\begin{proof}

\end{proof}

\subsection{Exercise 10}

\begin{proof}

\end{proof}

\subsection{Exercise 11}

\begin{proof}

\end{proof}

\subsection{Exercise 12}

\begin{proof}

\end{proof}

\subsection{Exercise 13}

\begin{proof}

\end{proof}

\subsection{Exercise 14}

\begin{proof}

\end{proof}

\subsection{Exercise 15}

\begin{proof}

\end{proof}

\subsection{Exercise 16}

\begin{proof}

\end{proof}

\subsection{Exercise 17}

\begin{proof}

\end{proof}

\subsection{Exercise 18}

\subsubsection{(a)}

\begin{proof}

\end{proof}

\subsubsection{(b)}

\begin{proof}

\end{proof}

\subsection{Exercise 19}

\subsubsection{(a)}

\begin{proof}

\end{proof}

\subsubsection{(b)}

\begin{proof}

\end{proof}

\subsubsection{(c)}

\begin{proof}

\end{proof}

\subsection{Exercise 20}

\begin{proof}

\end{proof}

\subsection{Exercise 21}

\begin{proof}

\end{proof}

\subsection{Exercise 22}

\begin{proof}

\end{proof}

\subsection{Exercise 23}

\begin{proof}

\end{proof}

\subsection{Exercise 24}

\begin{proof}

\end{proof}

\subsection{Exercise 25}

\begin{proof}

\end{proof}

\subsection{Exercise 26}

\begin{proof}

\end{proof}

\subsection{Exercise 27}

\begin{proof}

\end{proof}

\subsection{Exercise 28}

\begin{proof}

\end{proof}

\subsection{Exercise 29}

\begin{proof}

\end{proof}

\subsection{Exercise 30}

\subsubsection{(a)}

\begin{proof}

\end{proof}

\subsubsection{(b)}

\begin{proof}

\end{proof}

\subsection{Exercise 31}

\begin{proof}

\end{proof}

\subsection{Exercise 32}

\begin{proof}

\end{proof}

\subsection{Exercise 33}

\begin{proof}

\end{proof}

\subsection{Exercise 34}

\subsubsection{(a)}

\begin{proof}

\end{proof}

\subsubsection{(b)}

\begin{proof}

\end{proof}

\subsection{Exercise 35}

\begin{proof}

\end{proof}

\subsection{Exercise 36}

\begin{proof}

\end{proof}

\subsection{Exercise 37}

\begin{proof}

\end{proof}

\subsection{Exercise 38}

\begin{proof}

\end{proof}

\section{Exercise Set 4.9}

\subsection{Exercise 1}

\begin{proof}

\end{proof}

\subsection{Exercise 2}

\begin{proof}

\end{proof}

\subsection{Exercise 3}

\begin{proof}

\end{proof}

\subsection{Exercise 4}

\begin{proof}

\end{proof}

\subsection{Exercise 5}

\begin{proof}

\end{proof}

\subsection{Exercise 6}

\begin{proof}

\end{proof}

\subsection{Exercise 7}

\begin{proof}

\end{proof}

\subsection{Exercise 8}

\begin{proof}

\end{proof}

\subsection{Exercise 9}

\begin{proof}

\end{proof}

\subsection{Exercise 10}

\begin{proof}

\end{proof}

\subsection{Exercise 11}

\begin{proof}

\end{proof}

\subsection{Exercise 12}

\begin{proof}

\end{proof}

\subsection{Exercise 13}

\begin{proof}

\end{proof}

\subsection{Exercise 14}

\subsubsection{(a)}

\begin{proof}

\end{proof}

\subsubsection{(b)}

\begin{proof}

\end{proof}

\subsection{Exercise 15}

\subsubsection{(a)}

\begin{proof}

\end{proof}

\subsubsection{(b)}

\begin{proof}

\end{proof}

\subsection{Exercise 16}

\subsubsection{(a)}

\begin{proof}

\end{proof}

\subsubsection{(b)}

\begin{proof}

\end{proof}

\subsection{Exercise 17}

\begin{proof}

\end{proof}

\subsection{Exercise 18}

\begin{proof}

\end{proof}

\subsection{Exercise 19}

\begin{proof}

\end{proof}

\subsection{Exercise 20}

\subsubsection{(a)}

\begin{proof}

\end{proof}

\subsubsection{(b)}

\begin{proof}

\end{proof}

\subsection{Exercise 21}

\subsubsection{(a)}

\begin{proof}

\end{proof}

\subsubsection{(b)}

\begin{proof}

\end{proof}

\subsubsection{(c)}

\begin{proof}

\end{proof}

\subsection{Exercise 22}

\begin{proof}

\end{proof}

\subsection{Exercise 23}

\subsubsection{(a)}

\begin{proof}

\end{proof}

\subsubsection{(b)}

\begin{proof}

\end{proof}

\subsubsection{(c)}

\begin{proof}

\end{proof}

\subsubsection{(d)}

\begin{proof}

\end{proof}

\subsubsection{(e)}

\begin{proof}

\end{proof}

\subsubsection{(f)}

\begin{proof}

\end{proof}

\subsection{Exercise 24}

\subsubsection{(a)}

\begin{proof}

\end{proof}

\subsubsection{(b)}

\begin{proof}

\end{proof}

\subsubsection{(c)}

\begin{proof}

\end{proof}

\subsubsection{(d)}

\begin{proof}

\end{proof}

\subsubsection{(e)}

\begin{proof}

\end{proof}

\subsubsection{(f)}

\begin{proof}

\end{proof}

\subsection{Exercise 25}

\begin{proof}

\end{proof}

\section{Exercise Set 4.10}

\subsection{Exercise 1}

\begin{proof}

\end{proof}

\subsection{Exercise 2}

\begin{proof}

\end{proof}

\subsection{Exercise 3}

\subsubsection{(a)}

\begin{proof}

\end{proof}

\subsubsection{(b)}

\begin{proof}

\end{proof}

\subsection{Exercise 4}

\begin{proof}

\end{proof}

\subsection{Exercise 5}

\begin{proof}

\end{proof}

\subsection{Exercise 6}

\begin{proof}

\end{proof}

\subsection{Exercise 7}

\begin{proof}

\end{proof}

\subsection{Exercise 8}

\subsubsection{(a)}

\begin{proof}

\end{proof}

\subsubsection{(b)}

\begin{proof}

\end{proof}

\subsection{Exercise 9}

\begin{proof}

\end{proof}

\subsection{Exercise 10}

\begin{proof}

\end{proof}

\subsection{Exercise 11}

\begin{proof}

\end{proof}

\subsection{Exercise 12}

\begin{proof}

\end{proof}

\subsection{Exercise 13}

\begin{proof}

\end{proof}

\subsection{Exercise 14}

\begin{proof}

\end{proof}

\subsection{Exercise 15}

\begin{proof}

\end{proof}

\subsection{Exercise 16}

\begin{proof}

\end{proof}

\subsection{Exercise 17}

\begin{proof}

\end{proof}

\subsection{Exercise 18}

\begin{proof}

\end{proof}

\subsection{Exercise 19}

\begin{proof}

\end{proof}

\subsection{Exercise 20}

\begin{proof}

\end{proof}

\subsection{Exercise 21}

\begin{proof}

\end{proof}

\subsection{Exercise 22}

\begin{proof}

\end{proof}

\subsection{Exercise 23}

\subsubsection{(a)}

\begin{proof}

\end{proof}

\subsubsection{(b)}

\begin{proof}

\end{proof}

\subsection{Exercise 24}

\begin{proof}

\end{proof}

\subsection{Exercise 25}

\subsubsection{(a)}

\begin{proof}

\end{proof}

\subsubsection{(b)}

\begin{proof}

\end{proof}

\subsection{Exercise 26}

\subsubsection{(a)}

\begin{proof}

\end{proof}

\subsubsection{(b)}

\begin{proof}

\end{proof}

\subsection{Exercise 27}

\subsubsection{(a)}

\begin{proof}

\end{proof}

\subsubsection{(b)}

\begin{proof}

\end{proof}

\subsubsection{(c)}

\begin{proof}

\end{proof}

\subsection{Exercise 28}

\subsubsection{(a)}

\begin{proof}

\end{proof}

\subsubsection{(b)}

\begin{proof}

\end{proof}

\subsubsection{(c)}

\begin{proof}

\end{proof}

\subsection{Exercise 29}

\begin{proof}

\end{proof}

\subsection{Exercise 30}

\begin{proof}

\end{proof}

\subsection{Exercise 31}

\begin{proof}

\end{proof}

\subsection{Exercise 32}

\begin{proof}

\end{proof}

\end{document}
